%Trang bìa
\chapmoi{TÍNH TẢI LẠNH BẰNG PHƯƠNG PHÁP \textit{CARRIER}}

%Phần nội dung
<<<<<<< HEAD
\section{XÂY DỰNG MÔ HÌNH 3D BẰNG \emph{REVIT}}
\subsection{Giới thiệu}
	Toà khách sạn Grand Plaza được xây dựng 27 tầng, mỗi tầng có nhiệm vụ và chức năng khác nhau, dưới đây là hình toà khách sạn được dựng bởi mô hình 3D bằng Revit}.

\begin{figure}[H]
  \centering
  \includegraphics[width=0.9\textwidth]{revit}
  \caption{\textbf{Bản vẽ 3D Revit khách sạn Plaza}}
  \label{revit}
\end{figure}

\begin{figure}[H]
	\centering
	\includegraphics[width=0.9\textwidth]{cadtongthe}
	\caption{\textbf{Bản vẽ Cad khách sạn Plaza}}
	\label{cadtongthe}
\end{figure}

\subsection{Bản vẽ từng tầng}
\textbf{Tầng 1}: Được sử dụng làm phòng cấp cứu, coffee, phòng máy và sảnh văn phòng.

\begin{figure}[H]
	\centering
	\includegraphics[width=1\textwidth]{1ex}
	\caption{\textbf{Thông số tầng 1}}
	\label{1ex}
\end{figure}

\begin{figure}[H]
  \centering
  \includegraphics[width=1\textwidth]{1stcad}
  \caption{\textbf{Mặt bằng Cad tầng 1}}
  \label{1stcad}
\end{figure}

\begin{figure}[H]
  \centering
  \includegraphics[width=1\textwidth]{1strevit}
  \caption{\textbf{Mặt bằng Revit tầng 1}}
  \label{1strevit}
\end{figure}

\begin{figure}[H]
	\centering
	\includegraphics[width=0.8\textwidth]{r1strevit}
	\caption{Mặt bên phải mặt bằng Revit tầng 1}
	\label{l1strevit}
\end{figure}

\begin{figure}[H]
	\centering
	\includegraphics[width=0.8\textwidth]{l1strevit}
	\caption{Mặt bên trái mặt bằng Revit tầng 1}
	\label{l1strevit}
\end{figure}

\newpage
\textbf{Tầng M}: Được sử dụng làm phòng điều khiển.
\begin{figure}[H]
	\centering
	\includegraphics[width=1\textwidth]{mex}
	\caption{\textbf{Thông số tầng M}}
	\label{mex}
\end{figure}

\begin{figure}[H]
  \centering
  \includegraphics[width=0.8\textwidth]{mcad}
  \caption{Tầng M}
  \label{mcad}
\end{figure}

\begin{figure}[H]
  \centering
  \includegraphics[width=0.7\textwidth]{mrevit}
  \caption{Tầng M bằng Revit}
  \label{mrevit}
\end{figure} 

\begin{figure}[H]
	\centering
	\includegraphics[width=0.8\textwidth]{rmrevit}
	\caption{Mặt bên phải mặt bằng Revit tầng M}
	\label{mrevit}
\end{figure} 

\begin{figure}[H]
	\centering
	\includegraphics[width=0.8\textwidth]{lmrevit}
	\caption{Mặt bên trái mặt bằng Revit tầng M}
	\label{lrevit}
\end{figure} 

\newpage
\textbf{Tầng 2 - 3}: Được sử dụng làm cửa hàng.
\begin{figure}[H]
	\centering
	\includegraphics[width=0.8\textwidth]{23ex}
	\caption{\textbf{Thông số tầng 2 - 3}}
	\label{23ex}
\end{figure}

\begin{figure}[H]
  \centering
  \includegraphics[width=0.7\textwidth]{23cad}
  \caption{Tầng 2 - 3 bằng Cad}
  \label{23cad}
\end{figure}

\begin{figure}[H]
  \centering
  \includegraphics[width=0.7\textwidth]{23revit}
  \caption{Tầng 2 - 3 bằng Revit}
  \label{23revit}
\end{figure} 

\begin{figure}[H]
	\centering
	\includegraphics[width=0.8\textwidth]{r23revit}
	\caption{Mặt bên phải mặt bằng Revit tầng 2 - 3}
	\label{23revit}
\end{figure} 

\begin{figure}[H]
	\centering
	\includegraphics[width=0.7\textwidth]{l23revit}
	\caption{Mặt bên trái mặt bằng Revit tầng 2 - 3}
	\label{23revit}
\end{figure} 

\newpage
\textbf{Tầng 4}: Được sử dụng làm văn phòng.
\begin{figure}[H]
	\centering
	\includegraphics[width=0.8\textwidth]{4ex}
	\caption{\textbf{Thông số tầng 4}}
	\label{4ex}
\end{figure}

\begin{figure}[H]
  \centering
  \includegraphics[width=0.8\textwidth]{4cad}
  \caption{Tầng 4 bằng Cad}
  \label{4cad}
\end{figure}

\begin{figure}[H]
  \centering
  \includegraphics[width=0.8\textwidth]{4revit}
  \caption{Tầng 4 bằng Revit}
  \label{4revit}
\end{figure} 

\begin{figure}[H]
	\centering
	\includegraphics[width=0.7\textwidth]{r4revit}
	\caption{Mặt bên phải mặt bằng Revit tầng 4}
	\label{4revit}
\end{figure} 

\begin{figure}[H]
	\centering
	\includegraphics[width=0.7\textwidth]{l4revit}
	\caption{Mặt bên trái mặt bằng Revit tầng 4}
	\label{4revit}
\end{figure} 

\newpage
\textbf{Tầng 5 - 27}: Được sử dụng làm văn phòng.
\begin{figure}[H]
	\centering
	\includegraphics[width=0.8\textwidth]{527ex}
	\caption{\textbf{Thông số tầng 5 - 27}}
	\label{527ex}
\end{figure}

\begin{figure}[H]
  \centering
  \includegraphics[width=0.8\textwidth]{527cad}
  \caption{Tầng 5 - 27 bằng Cad}
  \label{527cad}
\end{figure}

\begin{figure}[H]
  \centering
  \includegraphics[width=0.8\textwidth]{527revit}
  \caption{Tầng 5 - 27 bằng Revit}
  \label{527revit}
\end{figure} 

\begin{figure}[H]
	\centering
	\includegraphics[width=0.7\textwidth]{r527revit}
	\caption{Mặt bên phải mặt bằng Revit tầng 5 - 27}
	\label{527revit}
\end{figure} 

\begin{figure}[H]
	\centering
	\includegraphics[width=0.7\textwidth]{l527revit}
	\caption{Mặt bên trái mặt bằng Revit tầng 5 - 27}
	\label{527revit}
\end{figure} 

\section{PHƯƠNG PHÁP TRUYỀN THỐNG}
\subsection{ĐẠI CƯƠNG}
Các bước chủ yếu để tính toán cân bằng nhiệt ẩm truyền thống gồm \textbf{7 bước} để tính toán như sau:
\subsubsection{Xác định các nguồn nhiệt toả ra}
- Các nguồn nhiệt này có thể xuất phát từ nhiều nguồn khác nhau, điển hình như: Do người, do máy móc, chiếu sáng, rò rỉ không khí, bức xạ mặt trời, thẩm thấu qua kết cấu,...

- Phương trình cân bằng nhiệt tổng quát có dạng:

\begin{equation}
	Q_{t} = Q_{t1} + Q_{t2}
\end{equation}

\begin{itemize}[leftmargin = 3cm, label = $\star$]
	\item $Q_{t}$ - nhiệt thừa trong phòng, \textit{W};
	\item $Q_{t1}$ - nhiệt toả ra trong phòng, \textit{W};
	\item $Q_{t2}$ - nhiệt thẩm thấu từ ngoài vào qua kết cấu bao che do chênh lệch nhiệt độ, \textit{W};
=======
\section{THÔNG SỐ THIẾT KẾ}
\subsection{NHIỆT ĐỘ BÊN NGOÀI}
-- Các thông số bên ngoài toà nhà được thể hiện trong bảng dưới đây:
\begin{table}[H]
	\vspace{-0.3cm}
	\centering
	\caption{Thông số ngoài trời}
	\begin{tabular}{|l|r|}
		\hline
		Nhiệt độ bên ngoài -- t{\scriptsize N} & \textcolor[rgb]{ 1,  0,  0}{\textbf{32} $ ^{\circ} $C} \bigstrut\\
		\hline
		Độ ẩm tương đối -- $\varphi$ & \textcolor[rgb]{ 1,  0,  0}{\textbf{75}\%} \bigstrut\\
		\hline
		Nhiệt độ bầu ướt -- t{\scriptsize ư} & \textcolor[rgb]{ 1,  0,  0}{\textbf{18.76} $ ^{\circ} $C} \bigstrut\\
		\hline
	\end{tabular}
	\label{b:tsnt}
\end{table}

\subsection{NHIỆT ĐỘ BÊN TRONG}
-- Các thông số yêu cầu bên trong toà nhà được thể hiện trong bảng dưới đây:
\begin{table}[H]
	\vspace{-0.3cm}
	\centering
	\caption{Thông số trong nhà}
	\begin{tabular}{|l|r|}
		\hline
		Nhiệt độ trong nhà -- t{\scriptsize N} & \textcolor[rgb]{ 1,  0,  0}{\textbf{25} $ ^{\circ} $C} \bigstrut\\
		\hline
		Độ ẩm tương đối -- $\varphi$ & \textcolor[rgb]{ 1,  0,  0}{\textbf{55}\%} \bigstrut\\
		\hline
	\end{tabular}
	\label{b:tstn}
\end{table}

\section{TÍNH TOÁN TẢI LẠNH}
\subsection{NHIỆT HIỆN BỨC XẠ QUA KÍNH --- Q{\scriptsize 1}}
\subsubsection{Xác định Q$_{1}$}
Với điều kiện nhiệt độ và độ ẩm của không khí bên ngoài như đã chọn để thiết kế điều hoà không khí cho tòa nhà: t$_{T}$ = 32$^{\circ}$C; {\large $\varphi$}$_{N}$ = 75\%.

Tra đồ thị t – d ta được nhiệt độ đọng sương: {\large t}{\scriptsize đs} = 28.2619$^{\circ}$C.

Từ đó ta xác định được:
\begin{itemize}
	\item {\Large $\varepsilon$}{\scriptsize đs} $= 1 -\dfrac{t_{s} - 20}{10}\times0.13 = 1 -\dfrac{28.2619 - 20}{10}\times0.13 = 0.909$
	
	\item {\Large $\varepsilon$}{\scriptsize c} $= 1$ (do Tp.Hà Nội có độ cao gần mực nước biển).
	
	\item {\Large $\varepsilon$}{\scriptsize mm} $= 1$ (tính vào lúc bức xạ mặt trời lớn nhất).
	
	\item {\Large $\varepsilon$}{\scriptsize kh} $= 1.17$ (khung kính kim loại).
	
	\item {\Large $\varepsilon$}{\scriptsize m} $= 0.7$ (dùng kính Calorex màu xanh).
	
	\item {\Large $\varepsilon$}{\scriptsize r} $= 0.56$ (hệ số mặt trời kể đến ảnh hưởng của kính khác kính cơ bản khi có màn che bên trong).
>>>>>>> 9f4d729aebfed9639d1515f92e5cad78d680a649
\end{itemize}

Toàn bộ tòa nhà sử dụng loại kính Calorex, màu xanh, dày 6 mm và có sử dụng màn che màu sáng.
\begin{table}[H]
	\vspace{-0.3cm}
	\centering
	\caption{Các hệ số kính và màn che}
	\begin{adjustbox}{width=\textwidth}
		\begin{tabular}{|c|c|c|c|c|c|}
			\hline
			& \textbf{Hệ số hấp thụ} & \textbf{Hệ số phản xạ} & \textbf{Hệ số xuyên qua} & \textbf{Hệ số kính} & \textbf{Hệ số mặt trời} \bigstrut\\
			\hline
			\multirow{2}[2]{*}{Kính Carolex màu xanh} & \multirow{2}[2]{*}{{\large $\alpha$}$_{K}$ = 0.75} & \multirow{2}[2]{*}{{\large $\beta$}$_{K}$ = 0.05} & \multirow{2}[2]{*}{{\large $\tau$}$_{K}$ = 0.20} & \multirow{2}[2]{*}{{\large $\varepsilon$}$_{m}$ = 0.57} & \multirow{2}[2]{*}{} \bigstrut[t]\\
			&       &       &       &       &  \bigstrut[b]\\
			\hline
			\multirow{2}[2]{*}{Màn che màu sáng} & \multirow{2}[2]{*}{{\large $\alpha$}$_{m}$ = 0.37} & \multirow{2}[2]{*}{{\large $\beta$}$_{m}$ = 0.51} & \multirow{2}[2]{*}{{\large $\tau$}$_{m}$ = 0.12} & \multirow{2}[2]{*}{} & \multirow{2}[2]{*}{{\large $\varepsilon$}$_{r}$ = 0.56} \bigstrut[t]\\
			&       &       &       &       &  \bigstrut[b]\\
			\hline
		\end{tabular}%
		\label{tab:adadlabel}%
	\end{adjustbox}
\end{table}%

Nhiệt bức xạ mặt trời qua cửa kính khác kính cơ bản vào phòng theo các hướng của các phòng trong tòa nhà:
\begin{equation*}
	\begin{split}
		R_{K} &= [0.4\alpha_{K} + \tau_{K}(\alpha_{m} + \tau_{m} + \rho_{K}\rho_{m} + 0.4\alpha_{K}\alpha_{m})]R_{N} \\
		&= [0.4\alpha_{K} + \tau_{K}(\alpha_{m} + \tau_{m} + \rho_{K}\rho_{m} + 0.4\alpha_{K}\alpha_{m})]\dfrac{R}{0.88} \\
		&= [0.4\times0.75 + 0.2\times(0.37 + 0.12 + 0.05\times0.51 + 0.4\times0.75\times0.37)]\dfrac{R}{0.88} \\
		&= 0.4833\times R
	\end{split}
\end{equation*}
\begin{table}[H]
	\vspace{-0.3cm}
	\centering
	\caption{Bức xạ Mặt Trời qua kính lớn nhất theo hướng}
	\begin{tabular}{|c|c|c|}
		\hline
		\textbf{Hướng kính} & \textbf{RTmax ( W/m2 )} & \textbf{Rk ( W/m2 )} \bigstrut\\
		\hline
		Đông  & 520   & 251.32 \bigstrut\\
		\hline
		Tây   & 520   & 251.32 \bigstrut\\
		\hline
		Nam   & 470   & 227.15 \bigstrut\\
		\hline
		Bắc   & 82    & 39.63 \bigstrut\\
		\hline
	\end{tabular}
	\label{b:bxmtln}
\end{table}
Vậy: $Q_{1} = n_{t}\times F_{1}\times R_{K}\times $

\subsubsection{Hệ số tác dụng tức thời n{\scriptsize t}}
Để xác định hệ số tác dụng tức thời, ta phải xác định tổng khối lượng của các bề mặt tạo nên không gian điều hoà tính trên 1m$^2$:

\begin{center}
	G $= \rho \times \delta \times F$
\end{center}

Trong đó:
\begin{itemize}
	\item $F$ là diện tích tường, đơn vị m$^2$.
	\item $\rho$ là khối lượng riêng của vật liệu, đơn vị kg/m$^3$.
	\item $\delta$ là bề dày của vật liệu khảo sát, đơn vị m.
\end{itemize}

Cấu trúc kết cấu bao che như bảng dưới đây:

\begin{wraptable}[13]{l}{0.4\textwidth}
	\caption{Kết cấu bao che}
	\begin{tabular}{|r|r|r|r|r|}
		\hline
		\multicolumn{4}{|c|}{\textbf{Tường dày 200mm}} \bigstrut\\
		\hline
		\textbf{{\large $\delta_{g}$}} & 180 mm & {\large $\rho_{g}$} & 1800 kg/m$^3$ \bigstrut\\
		\hline
		{\large $\delta_{v}$} & 10 mm & \textbf{{\large $\rho_{v}$}} & 1800 kg/m$^3$ \bigstrut\\
		\hline
		\multicolumn{4}{|c|}{\textbf{Sàn}} \bigstrut\\
		\hline
		\textbf{{\large $\delta_{bt}$}} & 250 mm & {\large $\rho_{bt}$} & 2400 kg/m$^3$ \bigstrut\\
		\hline
		\textbf{{\large $\delta_{v}$}} & 50 mm & \textbf{{\large $\rho_{v}$}} & 1800 kg/m$^3$ \bigstrut\\
		\hline
		\multicolumn{4}{|c|}{\textbf{Trần}} \bigstrut\\
		\hline
		\textbf{{\large $\delta$}} & 50 mm & \textbf{{\large $\rho$}} & 1000 kg/m$^3$ \bigstrut\\
		\hline
	\end{tabular}
	\label{b:kcbc}%
\end{wraptable}%

\vspace{0.5cm}
Khối lượng bình quân của kết cấu bao che được xác định như sau:
\begin{equation*}
	g_{s} = \dfrac{G^{'}+0.5G^{''}}{F_{s}}, kg/m^2 
\end{equation*}

Trong đó:
\begin{itemize}
	\item $G^{'}$ -- khối lượng tường có mặt ngoài tiếp xúc với bức xạ mặt trời và của sàn nằm trên mặt đất, đơn vị là kg.
	\item $G^{''}$ -- khối lượng tường có mặt ngoài không tiếp xúc với bức xạ mặt trời và của sàn không nằm trên mặt đất, đơn vị là kg.
	\item $F_{s}$ -- diện tích sàn, đơn vị là m$^2$.
\end{itemize}

Bảng ở trang sau đây thể hiện khối lượng bình quân của kết cấu bao che:
\begin{landscape}
\begin{table}[H]
	\vspace{0.5cm}
	\centering
	\begin{tabular}{|l|l|r|r|r|r|r|r|r|}
		\hline
		\multicolumn{1}{|c|}{\textbf{ TẦNG}} & \multicolumn{1}{c|}{\textbf{TÊN PHÒNG}} & \multicolumn{1}{c|}{\textbf{DIỆN TÍCH (m²) }} & \multicolumn{1}{c|}{\textbf{F{\scriptsize kính}}} & \multicolumn{1}{c|}{\textbf{F{\scriptsize ng} (m$^2$)}} & \multicolumn{1}{c|}{\textbf{F{\scriptsize tr} (m$^2$)}} & \multicolumn{1}{c|}{\textbf{G'}} & \multicolumn{1}{c|}{\textbf{G''}} & \multicolumn{1}{c|}{\textbf{g$_{s}$}} \bigstrut\\
		\hline
		\textbf{Tầng 1} & sảnh văn phòng & 509.579 & 86    & 0     & 310.149 & 0     & 457680.5 & 449.0771 \bigstrut\\
		\hline
		& coffe & 717.5506 & 123   & 0     & 138.672 & 0     & 542535.7 & 378.047 \bigstrut\\
		\hline
		& phòng máy & 57.9477 & 0     & 37.71 & 99.945 & 12896.82 & 74165.1 & 862.491 \bigstrut\\
		\hline
		& phòng cấp cứu & 57.9477 & 0     & 37.71 & 99.945 & 12896.82 & 74165.1 & 862.491 \bigstrut\\
		\hline
		\textbf{Tầng M} & phòng điều khiển 1 & 266.8014 & 0     & 152.6895 & 31.1175 & 52219.81 & 194735.1 & 560.6694 \bigstrut\\
		\hline
		& phòng điều khiển 2 & 662.885 & 0     & 256.05 & 96.9795 & 87569.1 & 490557.6 & 502.1202 \bigstrut\\
		\hline
		\textbf{Tầng 2 - 3} & cửa hàng 1 & 218.2908 & 119.6478 & 0     & 226.0494 & 0     & 227929.5 & 522.0778 \bigstrut\\
		\hline
		& cửa hàng 2 & 167.484 & 91.8  & 0     & 198.2016 & 0     & 183348.9 & 547.3625 \bigstrut\\
		\hline
		& cửa hàng 3 & 166.2525 & 144.3258 & 0     & 144.3258 & 0     & 164073.6 & 493.4472 \bigstrut\\
		\hline
		& cửa hàng 4 & 166.2525 & 144.3258 & 0     & 144.3258 & 0     & 164073.6 & 493.4472 \bigstrut\\
		\hline
		& cửa hàng 5 & 167.484 & 91.8  & 0     & 198.2016 & 0     & 183348.9 & 547.3625 \bigstrut\\
		\hline
		& cửa hàng 6 & 218.2908 & 119.6478 & 0     & 226.0494 & 0     & 227929.5 & 522.0778 \bigstrut\\
		\hline
		\textbf{Tầng 4} & khối văn phòng 1 & 724.812 & 416   & 0     & 374.544 & 0     & 628214.3 & 433.3636 \bigstrut\\
		\hline
		& khối văn phòng 2 & 724.812 & 416   & 0     & 374.544 & 0     & 628214.3 & 433.3636 \bigstrut\\
		\hline
		\textbf{Tầng 5 - 27} & khối văn phòng & 2040.78 & 783.2 & 0     & 271.2 & 0     & 1500889 & 367.7243 \bigstrut\\
		\hline
	\end{tabular}%
	\caption{Khối lượng bình quân của kết cấu bao che}
	\label{b:klkcbc}%
\end{table}%
\end{landscape}


\begin{table}[H]
	\centering
	\caption{Nhiệt do bức xạ Mặt Trời}
	\begin{adjustbox}{width=\textwidth}
	\begin{tabular}{|c|r|r|l|r|r|r|r|}
		\hline
		\textbf{TẦNG} & \multicolumn{1}{c|}{\textbf{TÊN PHÒNG}} & \multicolumn{1}{c|}{\textbf{DIỆN TÍCH (m²) }} & \multicolumn{1}{c|}{\textbf{HƯỚNG}} & \multicolumn{1}{c|}{\textbf{gs}} & \multicolumn{1}{c|}{\textbf{Rk}} & \multicolumn{1}{c|}{\textbf{nt}} & \multicolumn{1}{c|}{\textbf{Q1}} \bigstrut\\
		\hline
		\multirow{4}[8]{*}{\textbf{Tầng 1}} & \multicolumn{1}{l|}{sảnh văn phòng} & 86    & B     & 449.0771 & 39.6306 & 0.91  & 1052.837 \bigstrut\\
		\cline{2-8}          & \multicolumn{1}{l|}{coffe} & 123   & Đ     & 378.047 & 251.316 & 0.8   & 8394.713 \bigstrut\\
		\cline{2-8}          & \multicolumn{1}{l|}{phòng máy} & 0     & N     & 862.491 & 227.151 & 0.67  & 0 \bigstrut\\
		\cline{2-8}          & \multicolumn{1}{l|}{phòng cấp cứu} & 0     & N     & 862.491 & 227.151 & 0.67  & 0 \bigstrut\\
		\hline
		\multirow{2}[4]{*}{\textbf{Tầng M}} & \multicolumn{1}{l|}{phòng điều khiển 1} & 0     & Đ     & 560.6694 & 251.316 & 0.65  & 0 \bigstrut\\
		\cline{2-8}          & \multicolumn{1}{l|}{phòng điều khiển 2} & 0     & N     & 502.1202 & 227.151 & 0.71  & 0 \bigstrut\\
		\hline
		\multirow{6}[12]{*}{\textbf{Tầng 2 - 3}} & \multicolumn{1}{l|}{cửa hàng 1} & 119.6478 & N     & 522.0778 & 227.151 & 0.71  & 6550.408 \bigstrut\\
		\cline{2-8}          & \multicolumn{1}{l|}{cửa hàng 2} & 91.8  & N     & 547.3625 & 227.151 & 0.71  & 5025.813 \bigstrut\\
		\cline{2-8}          & \multicolumn{1}{l|}{cửa hàng 3} & 144.3258 & N     & 493.4472 & 227.151 & 0.71  & 7901.464 \bigstrut\\
		\cline{2-8}          & \multicolumn{1}{l|}{cửa hàng 4} & 144.3258 & B     & 493.4472 & 39.6306 & 0.91  & 1766.878 \bigstrut\\
		\cline{2-8}          & \multicolumn{1}{l|}{cửa hàng 5} & 91.8  & B     & 547.3625 & 39.6306 & 0.91  & 1123.842 \bigstrut\\
		\cline{2-8}          & \multicolumn{1}{l|}{cửa hàng 6} & 119.6478 & B     & 522.0778 & 39.6306 & 0.91  & 1464.763 \bigstrut\\
		\hline
		\multirow{2}[4]{*}{\textbf{Tầng 4}} & \multicolumn{1}{l|}{khối văn phòng 1} & 416   & B     & 433.3636 & 39.6306 & 0.91  & 5092.793 \bigstrut\\
		\cline{2-8}          & \multicolumn{1}{l|}{khối văn phòng 2} & 416   & N     & 433.3636 & 227.151 & 0.71  & 22774.92 \bigstrut\\
		\hline
		\multirow{4}[8]{*}{\textbf{Tầng 5 - 27}} & \multicolumn{1}{c|}{\multirow{4}[8]{*}{khối văn phòng}} & 120.4 & Đ     & 367.7243 & 251.316 & 0.8   & 188997.1 \bigstrut\\
		\cline{3-8}          &       & 120.4 & T     & 367.7243 & 251.316 & 0.85  & 200809.4 \bigstrut\\
		\cline{3-8}          &       & 271.2 & N     & 367.7243 & 227.151 & 0.88  & 423258.3 \bigstrut\\
		\cline{3-8}          &       & 271.2 & B     & 367.7243 & 39.6306 & 0.99  & 83075.69 \bigstrut\\
		\hline
		\textbf{Mái} &       & 2040.78 & Ngang &       & 382.7736 & 1     & 265172.7 \bigstrut\\
		\hline
	\end{tabular}%
\end{adjustbox}
	\label{b:ndbxmt}%
\end{table}%

\textbf{Vậy tổng nhiệt do bức xạ:} $Q_{1} = 1222.462(KW)$.

\subsection{NHIỆT TRUYỀN KẾT CẤU BAO CHE -- Q{\scriptsize 2}}
Nhiệt truyền qua vách Q2 gồm hai thành phần:

-- Nhiệt do bức xạ vào tường được bỏ qua trong quá trình tính toán.

-- Nhiệt do chênh nhiệt độ giữa không khí trong phòng và ngoài nhà.
\begin{equation*}
	\Delta t = t_{n} - t_{T}
\end{equation*}

Vậy: $Q_{2} = Q_{21} + Q_{22} + Q_{23} + Q_{24} + Q_{25}$, W 

Nhiệt truyền qua kết cấu bao che được xác định bằng công thức:
\begin{equation*}
	Q_{2X} = k_{2X}\times F_{2X}\times\Delta t
\end{equation*}

Trong đó:
\begin{itemize}
	\item $k_{2X}$ : Hệ số truyền nhiệt qua kế cấu bao che W/(m$^2$.K).
	\item $F_{2X}$ : Diện tích kết cấu bao che (m²).
	\item $\Delta t$ : Độ chênh lệch nhiệt độ với không gian không diều hòa ($^{\circ}C$).
\end{itemize}

\subsubsection{Nhiệt truyền qua tường -- Q{\scriptsize 21}}

Xác định hệ số tryền nhiệt qua vách tường:

-- Đối với tường ngoài dày 200 mm:
\begin{table}[H]
	\centering
	\begin{tabular}{|r|r|r|r|r|}
		\hline
		\multicolumn{4}{|c|}{\textbf{Tường ngoài dày 200mm}} \bigstrut\\
		\hline
		{\large $\delta_{v}$} & 10 mm & \textbf{{\large $\lambda_{v}$}} & 0.93 W/mK \bigstrut\\
		\hline
		\textbf{{\large $\delta_{g}$}} & 180 mm & {\large $\lambda_{g}$} & 0.81 W/mK \bigstrut\\
		\hline
		{\large $\delta_{v}$} & 10 mm & \textbf{{\large $\lambda_{v}$}} & 0.93 W/mK \bigstrut\\
		\hline
	\end{tabular}
\end{table}
\vspace{-0.5cm}
{\Large \begin{equation*}
	\begin{split}
	k_{21} &= \dfrac{1}{\frac{1}{\alpha_{N}} + \frac{\delta_{g}}{\lambda_{g}} + 2\times\frac{\delta_{v}}{\lambda_{v}} + \frac{1}{\alpha_{T}}}\\
	&=\dfrac{1}{\frac{1}{20} + \frac{0.18}{0.81} + 2\times\frac{0.01}{0.93} + \frac{1}{10}}\\
	&={\scriptstyle 2.54, W/m^2 K}
	\end{split}
\end{equation*}}

-- Đối với tường trong dày 200 mm:
\begin{table}[H]
	\centering
	\begin{tabular}{|r|r|r|r|r|}
		\hline
		\multicolumn{4}{|c|}{\textbf{Tường trong dày 200mm}} \bigstrut\\
		\hline
		{\large $\delta_{v}$} & 10 mm & \textbf{{\large $\lambda_{v}$}} & 0.93 W/mK \bigstrut\\
		\hline
		\textbf{{\large $\delta_{g}$}} & 180 mm & {\large $\lambda_{g}$} & 0.81 W/mK \bigstrut\\
		\hline
		{\large $\delta_{v}$} & 10 mm & \textbf{{\large $\lambda_{v}$}} & 0.93 W/mK \bigstrut\\
		\hline
	\end{tabular}
\end{table}
\vspace{-0.5cm}
{\Large \begin{equation*}
	\begin{split}
		k_{21} &= \dfrac{1}{\frac{1}{\alpha_{T}} + \frac{\delta_{g}}{\lambda_{g}} + 2\times\frac{\delta_{v}}{\lambda_{v}} + \frac{1}{\alpha_{T}}}\\
		&=\dfrac{1}{\frac{1}{10} + \frac{0.18}{0.81} + 2\times\frac{0.01}{0.93} + \frac{1}{10}}\\
		&={\scriptstyle 2.25,  W/m^2 K}
	\end{split}
\end{equation*}}

\begin{table}[H]
	\vspace{-1cm}
	\centering
	\caption{Nhiệt truyền qua tường}
	\begin{adjustbox}{width=\textwidth}
	\begin{tabular}{|c|l|r|r|r|r|r|}
		\hline
		\textbf{ TẦNG} & \multicolumn{1}{c|}{\textbf{TÊN PHÒNG}} & \multicolumn{1}{c|}{\textbf{K kính}} & \multicolumn{1}{p{6.145em}|}{\textbf{DIỆN TÍCH (m²) }} & \multicolumn{1}{c|}{\textbf{t$_{N}$ ($^{\circ}C$)}} & \multicolumn{1}{c|}{\textbf{t$_{N}$ ($^{\circ}C$)}} & \multicolumn{1}{c|}{\textbf{Q$_{21}$(W)}} \bigstrut\\
		\hline
		\multirow{4}[8]{*}{\textbf{Tầng 1}} & sảnh văn phòng & 2.57  & 86    & 32    & 25    & 1547.14 \bigstrut\\
		\cline{2-7}          & coffe & 2.57  & 123   & 32    & 25    & 2212.77 \bigstrut\\
		\cline{2-7}          & phòng máy & 2.57  & 0     & 32    & 25    & 0 \bigstrut\\
		\cline{2-7}          & phòng cấp cứu & 2.57  & 0     & 32    & 25    & 0 \bigstrut\\
		\hline
		\multirow{2}[4]{*}{\textbf{Tầng M}} & phòng điều khiển 1 & 2.57  & 0     & 32    & 25    & 0 \bigstrut\\
		\cline{2-7}          & phòng điều khiển 2 & 2.57  & 0     & 32    & 25    & 0 \bigstrut\\
		\hline
		\multirow{6}[12]{*}{\textbf{Tầng 2 - 3}} & cửa hàng 1 & 2.57  & 120   & 32    & 25    & 2152.464 \bigstrut\\
		\cline{2-7}          & cửa hàng 2 & 2.57  & 92    & 32    & 25    & 1651.482 \bigstrut\\
		\cline{2-7}          & cửa hàng 3 & 2.57  & 144   & 32    & 25    & 2596.421 \bigstrut\\
		\cline{2-7}          & cửa hàng 4 & 2.57  & 144   & 32    & 25    & 2596.421 \bigstrut\\
		\cline{2-7}          & cửa hàng 5 & 2.57  & 92    & 32    & 25    & 1651.482 \bigstrut\\
		\cline{2-7}          & cửa hàng 6 & 2.57  & 120   & 32    & 25    & 2152.464 \bigstrut\\
		\hline
		\multirow{2}[4]{*}{\textbf{Tầng 4}} & khối văn phòng 1 & 2.57  & 416   & 32    & 25    & 7483.84 \bigstrut\\
		\cline{2-7}          & khối văn phòng 2 & 2.57  & 416   & 32    & 25    & 7483.84 \bigstrut\\
		\hline
		\textbf{Tầng 5 - 27} & khối văn phòng & 2.57  & 783   & 32    & 25    & 14089.77 \bigstrut\\
		\hline
	\end{tabular}%
	\end{adjustbox}
	\label{b:ntqvt}%
\end{table}%

\textbf{Vậy tổng nhiệt lượng truyền qua vách tường:} Q$_{21} = 45618.09(W)$

\subsubsection{Nhiệt truyền qua cửa ra vào -- Q{\scriptsize 22}}

Vì các phòng trong tòa nhà có sử dụng cửa gỗ (dày 40mm). Tra hệ số truyền nhiệt theo bảng 4.12\footnote{Theo sách Thiết kế hệ thống điều hoà không khí - Nguyễn Đức Lợi}.

Ta có được:
\begin{equation*}
	K_{22} = 2.23, W/m^2K
\end{equation*}
\newpage
\begin{table}[H]
	\centering
	\caption{Nhiệt lượng truyền qua cửa ra vào}
	\begin{adjustbox}{width=\textwidth}
	\begin{tabular}{|c|l|r|r|r|r|r|}
		\hline
		\textbf{ TẦNG} & \multicolumn{1}{c|}{\textbf{TÊN PHÒNG}} & \multicolumn{1}{c|}{\textbf{K kính}} & \multicolumn{1}{p{6.145em}|}{\textbf{DIỆN TÍCH (m²) }} & \multicolumn{1}{c|}{\textbf{t$_{N}$ ($^{\circ}C$)}} & \multicolumn{1}{c|}{\textbf{t$_{N}$ ($^{\circ}C$)}} & \multicolumn{1}{c|}{\textbf{Q$_{22}$(W)}} \bigstrut\\
		\hline
		\multirow{4}[8]{*}{\textbf{Tầng 1}} & sảnh văn phòng & 2.57  & 7.92  & 32    & 25    & 142.4808 \bigstrut\\
		\cline{2-7}          & coffe & 2.57  & 11.88 & 32    & 25    & 213.7212 \bigstrut\\
		\cline{2-7}          & phòng máy & 2.57  & 3.96  & 32    & 25    & 71.2404 \bigstrut\\
		\cline{2-7}          & phòng cấp cứu & 2.57  & 3.96  & 32    & 25    & 71.2404 \bigstrut\\
		\hline
		\multirow{2}[4]{*}{\textbf{Tầng M}} & phòng điều khiển 1 & 2.57  & 3.96  & 32    & 25    & 71.2404 \bigstrut\\
		\cline{2-7}          & phòng điều khiển 2 & 2.57  & 3.96  & 32    & 25    & 71.2404 \bigstrut\\
		\hline
		\multirow{6}[12]{*}{\textbf{Tầng 2 - 3}} & cửa hàng 1 & 2.57  & 7.92  & 32    & 25    & 142.4808 \bigstrut\\
		\cline{2-7}          & cửa hàng 2 & 2.57  & 7.92  & 32    & 25    & 142.4808 \bigstrut\\
		\cline{2-7}          & cửa hàng 3 & 2.57  & 7.92  & 32    & 25    & 142.4808 \bigstrut\\
		\cline{2-7}          & cửa hàng 4 & 2.57  & 7.92  & 32    & 25    & 142.4808 \bigstrut\\
		\cline{2-7}          & cửa hàng 5 & 2.57  & 7.92  & 32    & 25    & 142.4808 \bigstrut\\
		\cline{2-7}          & cửa hàng 6 & 2.57  & 7.92  & 32    & 25    & 142.4808 \bigstrut\\
		\hline
		\multirow{2}[4]{*}{\textbf{Tầng 4}} & khối văn phòng 1 & 2.57  & 15.84 & 32    & 25    & 284.9616 \bigstrut\\
		\cline{2-7}          & khối văn phòng 2 & 2.57  & 15.84 & 32    & 25    & 284.9616 \bigstrut\\
		\hline
		\textbf{Tầng 5 - 27} & khối văn phòng & 2.57  & 31.68 & 32    & 25    & 13108.23 \bigstrut\\
		\hline
	\end{tabular}%
	\end{adjustbox}
	\label{b:ntqcrv}%
\end{table}%

\textbf{Vậy tổng nhiệt lượng truyền qua vách tường:} Q$_{22} = 15174.21(W)$

\subsubsection{Nhiệt truyền qua cửa chiếu sáng -- Q{\scriptsize 23}}

Vì công trình không có cửa chiếu sáng nên lượng nhiệt truyền qua cửa chiếu sáng có thể xem như bằng 0.

\subsubsection{Nhiệt truyền qua nền sàn -- Q{\scriptsize 24}}

Với kết cấu sàn tầng 1 (tiếp xúc với không gian không điều hòa) như sau:

\begin{table}[H]
	\centering
	\begin{tabular}{|r|r|r|r|r|}
		\hline
		\multicolumn{4}{|c|}{\textbf{Sàn}} \bigstrut\\
		\hline
		\textbf{{\large $\delta_{g}$}} & 250 mm & {\large $\lambda_{g}$} & 0.81 kg/mK \bigstrut\\
		\hline
		\textbf{{\large $\delta_{v}$}} & 50 mm & \textbf{{\large $\lambda_{v}$}} & 0.93 W/mK \bigstrut\\	
		\hline
	\end{tabular}
	\label{b:kcs}
\end{table}

{\Large \begin{equation*}
	\begin{split}
		k_{24} &= \dfrac{1}{\frac{1}{\alpha_{N}} + \frac{\delta_{g}}{\lambda_{g}} + 2\times\frac{\delta_{v}}{\lambda_{v}} + \frac{1}{\alpha_{T}}}\\
		&= \dfrac{1}{\frac{1}{20} + \frac{0.18}{0.81} + 2\times\frac{0.01}{0.93} + \frac{1}{10}}\\
		&={\scriptstyle 2.54, W/m^2 K}
	\end{split}
\end{equation*}}

Do các phòng ở tầng hầm 1 và tầng 1 có sàn tiếp xúc với tầng (Có không gian không điều hòa) nên ta phải tính toán giá trị Q24, còn các phòng từ tầng 2 đến tầng 27 có sàn tiếp xúc với phòng có điều hòa nên giá trị Q$ _{24} $ = 0.

\begin{table}[H]
	\centering
	\caption{Nhiệt truyền qua sàn}
	\begin{tabular}{|c|l|r|r|r|r|r|}
		\hline
		\textbf{ TẦNG} & \multicolumn{1}{c|}{\textbf{TÊN PHÒNG}} & \multicolumn{1}{c|}{\textbf{K$ _{24} $}} & \multicolumn{1}{p{6.145em}|}{\textbf{DIỆN TÍCH (m²) }} & \multicolumn{1}{c|}{\textbf{t$_{N}$ ($^{\circ}C$)}} & \multicolumn{1}{c|}{\textbf{t$_{T}$ ($^{\circ}C$)}} & \multicolumn{1}{c|}{\textbf{Q$ _{24} $(W)}} \bigstrut\\
		\hline
		\multirow{4}[8]{*}{\textbf{Tầng 1}} & sảnh văn phòng & 2.54  & 510   & 32    & 25    & 9059.698 \bigstrut\\
		\cline{2-7}          & coffe & 2.54  & 718   & 32    & 25    & 12757.18 \bigstrut\\
		\cline{2-7}          & phòng máy & 2.54  & 58    & 32    & 25    & 1030.24 \bigstrut\\
		\cline{2-7}          & phòng cấp cứu & 2.54  & 58    & 32    & 25    & 1030.24 \bigstrut\\
		\hline
	\end{tabular}%
	\label{tab:addlabel}%
\end{table}%
\textbf{Vậy tổng nhiệt lượng truyền qua vách tường:} Q$_{24} = 23877.36(W)$

\subsubsection{Nhiệt truyền qua mái -- Q{\scriptsize 25}:}

Các phòng ở tầng 27 có trần tiếp xúc phòng không điều hòa.
\begin{equation*}
	Q_{25} = k_{25}\times F\times \Delta t
\end{equation*}
\begin{table}[H]
	\centering
	\begin{tabular}{|r|r|r|r|r|}
		\hline
		\multicolumn{4}{|c|}{\textbf{Trần}} \bigstrut\\
		\hline
		\textbf{{\large $\delta_{g}$}} & 250 mm & {\large $\lambda_{g}$} & 0.81 kg/mK \bigstrut\\
		\hline
		\textbf{{\large $\delta_{v}$}} & 50 mm & \textbf{{\large $\lambda_{v}$}} & 0.93 W/mK \bigstrut\\	
		\hline
	\end{tabular}
	\label{b:kct}
\end{table}
\begin{table}[H]
	\centering
	\caption{Nhiệt truyền qua mái}
	\begin{tabular}{|c|l|r|r|r|r|r|}
		\hline
		\textbf{ TẦNG} & \multicolumn{1}{c|}{\textbf{TÊN PHÒNG}} & \multicolumn{1}{c|}{\textbf{K$ _{25} $}} & \multicolumn{1}{p{6.145em}|}{\textbf{DIỆN TÍCH (m²) }} & \multicolumn{1}{c|}{\textbf{t$_{N}$ ($^{\circ}C$)}} & \multicolumn{1}{c|}{\textbf{t$_{T}$ ($^{\circ}C$)}} & \multicolumn{1}{c|}{\textbf{Q$ _{25} $(W)}} \bigstrut\\
		\hline
		\textbf{Tầng 27} & khối văn phòng & 2.54     & 2041     & 32       & 25       & 36282.6 \bigstrut\\
		\hline
	\end{tabular}%
	\label{b:ntqm}%
\end{table}%
\textbf{Vậy tổng nhiệt lượng truyền qua vách tường:} Q$_{25} = 36282.6(W)$

\subsubsection{NHIỆT TOẢ DO ĐÈN \& CÁC THIẾT BỊ KHÁC -- Q{\scriptsize 3}}

Do chưa biết chính xác có bao nhiêu bòng đèn chiếu sáng thì ta lấy mật độ tải chiếu sáng theo tài liệu.

-- Tải do chiếu sáng lấy 11 W/m$^2$.

-- Tải do thiết bị điện lấy 25 W/m$^2$.

Ta có công thức:
\begin{equation*}
	Q = A\times F
\end{equation*}

Trong đó:
\begin{itemize}
	\item F là diện tích sàn, đơn vị m$^2$.
	\item A là mật độ tải do chiếu sáng /mật độ tải thiết bị.
\end{itemize}

\begin{table}[H]
	\centering
	\caption{Tải do chiếu sáng}
	\begin{adjustbox}{width=\textwidth, height=.2\textheight}
	\begin{tabular}{|c|l|r|r|r|r|r|}
		\hline
		\textbf{SỐ TẦNG} & \multicolumn{1}{c|}{\textbf{TÊN PHÒNG}} & \multicolumn{1}{c|}{\textbf{DIỆN TÍCH (m²) }} & \multicolumn{1}{c|}{\textbf{A (W/m²)}} & \multicolumn{1}{c|}{\textbf{n$ _{t} $}} & \multicolumn{1}{c|}{\textbf{n{\scriptsize đ}}} & \multicolumn{1}{c|}{\textbf{Q$_{3}$ (W)}} \bigstrut\\
		\hline
		\multirow{4}[8]{*}{\textbf{Tầng 1}} & sảnh văn phòng & 509.579  & 1        & 0.87     & 0.85     & 376.8337 \bigstrut\\
		\cline{2-7}             & coffe    & 717.5506 & 1        & 0.84     & 0.85     & 512.3311 \bigstrut\\
		\cline{2-7}             & phòng máy & 57.9477  & 1        & 0.87     & 0.85     & 42.85232 \bigstrut\\
		\cline{2-7}             & phòng cấp cứu & 57.9477  & 1        & 0.87     & 0.85     & 42.85232 \bigstrut\\
		\hline
		\multirow{2}[4]{*}{\textbf{Tầng M}} & phòng điều khiển 1 & 266.8014 & 1        & 0.87     & 0.85     & 197.2996 \bigstrut\\
		\cline{2-7}             & phòng điều khiển 2 & 662.885  & 1        & 0.87     & 0.85     & 490.2035 \bigstrut\\
		\hline
		\multirow{6}[12]{*}{\textbf{Tầng 2 - 3}} & cửa hàng 1 & 218.2908 & 1        & 0.87     & 0.85     & 161.426 \bigstrut\\
		\cline{2-7}             & cửa hàng 2 & 167.484  & 1        & 0.87     & 0.85     & 123.8544 \bigstrut\\
		\cline{2-7}             & cửa hàng 3 & 166.2525 & 1        & 0.87     & 0.85     & 122.9437 \bigstrut\\
		\cline{2-7}             & cửa hàng 4 & 166.2525 & 1        & 0.87     & 0.85     & 122.9437 \bigstrut\\
		\cline{2-7}             & cửa hàng 5 & 167.484  & 1        & 0.87     & 0.85     & 123.8544 \bigstrut\\
		\cline{2-7}             & cửa hàng 6 & 218.2908 & 1        & 0.87     & 0.85     & 161.426 \bigstrut\\
		\hline
		\multirow{2}[4]{*}{\textbf{Tầng 4}} & khối văn phòng 1 & 724.812  & 1        & 0.84     & 0.85     & 517.5158 \bigstrut\\
		\cline{2-7}             & khối văn phòng 2 & 724.812  & 1        & 0.84     & 0.85     & 517.5158 \bigstrut\\
		\hline
		\textbf{Tầng 5 - 27} & khối văn phòng & 2040.78  & 1        & 0.84     & 0.85     & 33513.69 \bigstrut\\
		\hline
	\end{tabular}%
	\end{adjustbox}
	\label{b:tcs}%
\end{table}%
\textbf{Vậy tổng nhiệt lượng do chiếu sáng:} Q$_{3} = 37028(W)$

\begin{table}[H]
	\centering
	\caption{Tải do thiết bị}
	\begin{tabular}{|c|l|r|r|r|}
		\hline
		\textbf{SỐ TẦNG} & \multicolumn{1}{c|}{\textbf{TÊN PHÒNG}} & \multicolumn{1}{c|}{\textbf{DIỆN TÍCH (m²) }} & \multicolumn{1}{c|}{\textbf{A (W/m²)}} & \multicolumn{1}{c|}{\textbf{Q3 (w)}} \bigstrut\\
		\hline
		\multirow{4}[8]{*}{\textbf{Tầng 1}} & sảnh văn phòng & 509.579  & 25       & 12739.47 \bigstrut\\
		\cline{2-5}             & coffe    & 717.5506 & 25       & 17938.76 \bigstrut\\
		\cline{2-5}             & phòng máy & 57.9477  & 25       & 1448.693 \bigstrut\\
		\cline{2-5}             & phòng cấp cứu & 57.9477  & 25       & 1448.693 \bigstrut\\
		\hline
		\multirow{2}[4]{*}{\textbf{Tầng M}} & phòng điều khiển 1 & 266.8014 & 25       & 6670.035 \bigstrut\\
		\cline{2-5}             & phòng điều khiển 2 & 662.885  & 25       & 16572.13 \bigstrut\\
		\hline
		\multirow{6}[12]{*}{\textbf{Tầng 2 - 3}} & cửa hàng 1 & 218.2908 & 25       & 5457.269 \bigstrut\\
		\cline{2-5}             & cửa hàng 2 & 167.484  & 25       & 4187.1 \bigstrut\\
		\cline{2-5}             & cửa hàng 3 & 166.2525 & 25       & 4156.313 \bigstrut\\
		\cline{2-5}             & cửa hàng 4 & 166.2525 & 25       & 4156.313 \bigstrut\\
		\cline{2-5}             & cửa hàng 5 & 167.484  & 25       & 4187.1 \bigstrut\\
		\cline{2-5}             & cửa hàng 6 & 218.2908 & 25       & 5457.269 \bigstrut\\
		\hline
		\multirow{2}[4]{*}{\textbf{Tầng 4}} & khối văn phòng 1 & 724.812  & 25       & 18120.3 \bigstrut\\
		\cline{2-5}             & khối văn phòng 2 & 724.812  & 25       & 18120.3 \bigstrut\\
		\hline
		\textbf{Tầng 5 - 27} & khối văn phòng & 2040.78  & 25       & 1173449 \bigstrut\\
		\hline
	\end{tabular}%
	\label{b:ttb}%
\end{table}%
\textbf{Vậy tổng nhiệt lượng do thiết bị:} Q$_{3} = 1294.108(KW)$

\subsection{NHIỆT TOẢ DO NGƯỜI -- Q{\scriptsize 4} }
\subsubsection{Nhiệt hiện do người toả ra Q{\scriptsize 4h}}
-- Lượng nhiệt hiện do người toả ra:
\begin{equation*}
	Q_{4h} = n\times q_{h}, W
\end{equation*}

Trong đó:
\begin{itemize}
	\item n - là số người trong không gian cần điều hòa.
	\item $q_{h}$ - là nhiệt hiện tỏa ra từ một người, W/người.
\end{itemize}

Trong trường hợp số lượng người quá đông như : hội trường, rạp hát, vũ trường, sân khấu, phòng thi đấu thể thao …cần kể đến sự hấp thụ của kết cấu bao che . Do đó, ta cần kể đến hệ số tác động tức thời nt (hệ số tác động tức thời do chiếu sáng và nhiệt hiện của người), tra bảng 4-8.

Đối với các tòa nhà lớn ta cần tính thêm hệ số tác dụng không đồng thời n{\scriptsize đ}:
\begin{itemize}[label={-}]
	\item Cửa hàng bách hóa: n{\scriptsize đ} = 0,75 $ \div $ 0,9.
	\item Nhà cao tầng khách sạn: n{\scriptsize đ} = 0,8 $ \div $ 0,9.
	\item Nhà cao tầng công sở: n{\scriptsize đ} = 0,8 $ \div $ 0,9.
\end{itemize}

Vậy: $Q_{4h} = n_{t}\times n_{d}\times n\times q_{h}$

\subsubsection{Nhiệt ẩn do người toả ra -- Q{\scriptsize 4a}}
Trong không gian cần điều hoà ngoài sự hiện diện của thành phần nhiệt hiện còn có thành phần khác cũng phải kể đến là nhiệt ẩn. Nhiệt ẩn trong không gian điều hoà có thể do ngưới toả ra (như mồ hôi, do thở), do thức ăn toả ra (nơi ăn uống).

Nhiệt ẩn của phòng điều hoà được xác định như sau:
\begin{equation*}
	Q_{3} = n\times q_{a}, W
\end{equation*}

Trong đó:
\begin{itemize}
	\item n - số người trong phòng cần điều hoà.
	\item $q_{a}$ -  nhiệt ẩn do một người toả ra , W/người.
\end{itemize}

Đối với khu ăn uống cộng thêm 10 W/người do thức ăn toả ra .

Chọn hệ số tác dụng đồng thời n{\scriptsize đ} = 0,9.

Do số người hiện diện trong không gian cần điều hoà không cố định nên xem sự hiện diện của nam và nữ trong không gian điều hoà là như nhau để thuận lợi cho việc tính toán.

Ta lấy:
\begin{itemize}[label={-}]
	\item Nhiệt ẩn do một người tỏa ra: $q_{a}$ = 60 W.
	\item Nhiệt hiện do một người tỏa ra: $q_{h}$ = 70 W.
\end{itemize}

\begin{table}[H]
	\centering
	\caption{Mật độ người trong các loại phòng}
	\begin{adjustbox}{width=\textwidth}
	\begin{tabular}{|p{10.785em}|c|c|c|c|c|}
		\hline
		\textbf{Loại không  gian} & \multicolumn{1}{p{7.285em}|}{\textbf{Văn phòng làm việc}} & \multicolumn{1}{p{4.07em}|}{\textbf{Cà phê}} & \multicolumn{1}{p{4.07em}|}{\textbf{Sảnh}} & \multicolumn{1}{p{4.07em}|}{\textbf{Hội nghị}} & \multicolumn{1}{p{6.785em}|}{\textbf{Phòng phụ trợ}} \bigstrut\\
		\hline
		Mật độ , m2/người & 8        & 3        & 10       & 2        & 5 \bigstrut\\
		\hline
	\end{tabular}%
	\end{adjustbox}
	\label{b:mdn}%
\end{table}%

\begin{landscape}
\begin{table}[H]
	\vspace{1cm}
	\centering
	\begin{adjustbox}{width=1.5\textheight, height=0.4\textwidth}
	\begin{tabular}{|c|l|r|r|r|r|r|r|r|r|r|}
		\hline
		\textbf{TẦNG} & \multicolumn{1}{c|}{\textbf{TÊN PHÒNG}} & \multicolumn{1}{c|}{\textbf{DIỆN TÍCH (m²) }} & \multicolumn{1}{c|}{\textbf{MẬT ĐỘ NGƯỜI m²/người}} & \multicolumn{1}{c|}{\textbf{n}} & \multicolumn{1}{c|}{\textbf{qh (w)}} & \multicolumn{1}{c|}{\textbf{qn (w)}} & \multicolumn{1}{c|}{\textbf{nđ}} & \multicolumn{1}{c|}{\textbf{Qh (w)}} & \multicolumn{1}{c|}{\textbf{Qa (w)}} & \multicolumn{1}{c|}{\textbf{Q4 (w)}} \bigstrut\\
		\hline
		\multirow{4}[8]{*}{\textbf{tầng 1}} & sảnh văn phòng & 509.579  & 10       & 50.9579  & 70       & 60       & 0.9      & 3210.348 & 3057.474 & 6267.822 \bigstrut\\
		\cline{2-11}             & coffe    & 717.5506 & 3        & 239.1835 & 70       & 60       & 0.9      & 15068.56 & 14351.01 & 29419.57 \bigstrut\\
		\cline{2-11}             & phòng máy & 57.9477  & 5        & 11.58954 & 70       & 60       & 0.9      & 730.141  & 695.3724 & 1425.513 \bigstrut\\
		\cline{2-11}             & phòng cấp cứu & 57.9477  & 5        & 11.58954 & 70       & 60       & 0.9      & 730.141  & 695.3724 & 1425.513 \bigstrut\\
		\hline
		\multirow{2}[4]{*}{\textbf{tầng M}} & phòng điều khiển 1 & 266.8014 & 5        & 53.36028 & 70       & 60       & 0.9      & 3361.697 & 3201.617 & 6563.314 \bigstrut\\
		\cline{2-11}             & phòng điều khiển 2 & 662.885  & 5        & 132.577  & 70       & 60       & 0.9      & 8352.351 & 7954.62  & 16306.97 \bigstrut\\
		\hline
		\multirow{6}[12]{*}{\textbf{tầng 2 - 3}} & cửa hàng 1 & 218.2908 & 8        & 27.28635 & 70       & 60       & 0.9      & 1719.04  & 1637.181 & 3356.22 \bigstrut\\
		\cline{2-11}             & cửa hàng 2 & 167.484  & 8        & 20.9355  & 70       & 60       & 0.9      & 1318.937 & 1256.13  & 2575.067 \bigstrut\\
		\cline{2-11}             & cửa hàng 3 & 166.2525 & 8        & 20.78156 & 70       & 60       & 0.9      & 1309.238 & 1246.894 & 2556.132 \bigstrut\\
		\cline{2-11}             & cửa hàng 4 & 166.2525 & 8        & 20.78156 & 70       & 60       & 0.9      & 1309.238 & 1246.894 & 2556.132 \bigstrut\\
		\cline{2-11}             & cửa hàng 5 & 167.484  & 8        & 20.9355  & 70       & 60       & 0.9      & 1318.937 & 1256.13  & 2575.067 \bigstrut\\
		\cline{2-11}             & cửa hàng 6 & 218.2908 & 8        & 27.28635 & 70       & 60       & 0.9      & 1719.04  & 1637.181 & 3356.22 \bigstrut\\
		\hline
		\multirow{2}[4]{*}{\textbf{tầng 4}} & khối văn phòng 1 & 724.812  & 8        & 90.6015  & 70       & 60       & 0.9      & 5707.895 & 5436.09  & 11143.98 \bigstrut\\
		\cline{2-11}             & khối văn phòng 2 & 724.812  & 8        & 90.6015  & 70       & 60       & 0.9      & 5707.895 & 5436.09  & 11143.98 \bigstrut\\
		\hline
		\textbf{tầng 5 - 27} & khối văn phòng & 2040.78  & 8        & 255.0975 & 70       & 60       & 0.9      & 16071.14 & 15305.85 & 721670.8 \bigstrut\\
		\hline
	\end{tabular}%
	\end{adjustbox}
	\caption{Nhiệt lượng do người toả ra}
	\label{b:ndn}%
\end{table}%
\end{landscape}
\textbf{Vậy tổng nhiệt lượng do người toả ra:} Q$_{4} = 822342(W)$

\subsection{NHỆT HIỆN \& ẨN DO GIÓ TƯƠI MANG VÀO -- Q{\scriptsize 5}}
\subsubsection{Các khu vực sử dụng FCU:}
Lượng nhiệt do gió tươi mang vào được xác định bằng biểu thức sau:
\begin{equation*}
	\begin{split}
	Q_{5} &= Q_{5h} + Q_{5a} \\
	Q_{5h} &= 1.2 \times n \times l \times (t_{N} - t_{T}), W \\
	Q_{5a} &= 3 \times n \times l \times (d_{N} - d_{T}), W
	\end{split}
\end{equation*}

Trong đó:
\begin{itemize}
	\item $t_{N}$, $t_{T}$ - nhiệt độ của không khí tươi bên ngoài và không khí trong không gian điều hoà, $^{\circ}C$.
	\item $d_{N}$, $d_{T}$ - độ chứa hơi của không khí tươi bên ngoài và không khí trong không gian điều hoà, g /kg.
	\item $ n $ - số người trong không gian điều hoà.
	\item $l$ - lượng không khí tươi từ ngoài trời cần đưa vào phòng cho một người trong một giây, l/s.
\end{itemize}

Yêu cầu gió tươi:
\begin{itemize}[label={-}]
	\item Phòng làm việc: 25 m$^3$/h.người
	\item Hành lang: 25 m$^3$/h.m$^2$sàn
	\item Bếp: 30 m$^3$/h.người
	\item Cà phê: 30 m$^3$/h.người
	\item Phòng hội nghị: 30 m$^3$/h.người
\end{itemize}

-- Tuỳ thuộc vào không gian mà nhiệt độ và độ chứa hơi khác nhau.

-- Độ chứa hơi của không khí được xác định theo phần mềm DAIKIN.

\begin{table}[H]
	\centering
	\caption{Độ chứa hơi của không khí}
	\begin{tabular}{|p{7.645em}|c|c|c|}
		\hline
		\textbf{Thông số} & \multicolumn{1}{p{5.43em}|}{\textbf{t ($^{\circ}C$)}} & \multicolumn{1}{p{4.645em}|}{\textbf{j (\%)}} & \multicolumn{1}{p{5.785em}|}{\textbf{d (g/kg kk)}} \bigstrut\\
		\hline
		Bên trong & 25       & 55       & 10.94 \bigstrut\\
		\hline
		Bên ngoài & 32       & 75       & 22.84 \bigstrut\\
		\hline
	\end{tabular}%
	\label{b:dch}%
\end{table}%

\begin{landscape}
	\begin{table}[H]
		\vspace{1.3cm}
		\centering
		\begin{adjustbox}{width=1.5\textheight, height=0.4\textwidth}
		\begin{tabular}{|c|l|r|r|r|r|r|r|r|r|r|r|r|}
			\hline
			\textbf{TẦNG} & \multicolumn{1}{c|}{\textbf{TÊN PHÒNG}} & \multicolumn{1}{p{5.285em}|}{\textbf{THỂ TÍCH (m3) }} & \multicolumn{1}{p{4.07em}|}{\textbf{n}} & \multicolumn{1}{p{5.5em}|}{\textbf{l (l/s/người)}} & \multicolumn{1}{p{6.645em}|}{\textbf{LN (l/s)}} & \multicolumn{1}{p{4.07em}|}{\textbf{tN}} & \multicolumn{1}{p{4.07em}|}{\textbf{tT}} & \multicolumn{1}{p{4.07em}|}{\textbf{dT}} & \multicolumn{1}{p{4.07em}|}{\textbf{dN}} & \multicolumn{1}{p{5.855em}|}{\textbf{Q5h (W)}} & \multicolumn{1}{p{6em}|}{\textbf{Q5a (W)}} & \multicolumn{1}{p{6.715em}|}{\textbf{Q5 (W)}} \bigstrut\\
			\hline
			\multirow{4}[8]{*}{\textbf{Tầng 1}} & sảnh văn phòng & 2293.11  & 51       & 5.5      & 280.27   & 32       & 25       & 10.94    & 22.84    & 2354.25  & 10003.19 & 12357.45 \bigstrut\\
			\cline{2-13}             & coffe    & 3228.98  & 239      & 4.7      & 1124.16  & 32       & 25       & 10.94    & 22.84    & 9442.97  & 40123.02 & 49565.98 \bigstrut\\
			\cline{2-13}             & phòng máy & 260.76   & 12       & 4.5      & 52.15    & 32       & 25       & 10.94    & 22.84    & 438.08   & 1861.41  & 2299.50 \bigstrut\\
			\cline{2-13}             & phòng cấp cứu & 260.76   & 12       & 4.5      & 52.15    & 32       & 25       & 10.94    & 22.84    & 438.08   & 1861.41  & 2299.50 \bigstrut\\
			\hline
			\multirow{2}[4]{*}{\textbf{Tầng M}} & phòng điều khiển 1 & 1200.61  & 53       & 4.5      & 240.12   & 32       & 25       & 10.94    & 22.84    & 2017.02  & 8570.28  & 10587.30 \bigstrut\\
			\cline{2-13}             & phòng điều khiển 2 & 2982.98  & 133      & 4.5      & 596.60   & 32       & 25       & 10.94    & 22.84    & 5011.41  & 21293.41 & 26304.82 \bigstrut\\
			\hline
			\multirow{6}[12]{*}{\textbf{Tầng 2 - 3}} & cửa hàng 1 & 1178.77  & 27       & 5        & 136.43   & 32       & 25       & 10.94    & 22.84    & 1146.03  & 4869.45  & 6015.48 \bigstrut\\
			\cline{2-13}             & cửa hàng 2 & 904.41   & 21       & 5        & 104.68   & 32       & 25       & 10.94    & 22.84    & 879.29   & 3736.09  & 4615.39 \bigstrut\\
			\cline{2-13}             & cửa hàng 3 & 897.76   & 21       & 5        & 103.91   & 32       & 25       & 10.94    & 22.84    & 872.83   & 3708.62  & 4581.45 \bigstrut\\
			\cline{2-13}             & cửa hàng 4 & 897.76   & 21       & 5        & 103.91   & 32       & 25       & 10.94    & 22.84    & 872.83   & 3708.62  & 4581.45 \bigstrut\\
			\cline{2-13}             & cửa hàng 5 & 904.41   & 21       & 5        & 104.68   & 32       & 25       & 10.94    & 22.84    & 879.29   & 3736.09  & 4615.39 \bigstrut\\
			\cline{2-13}             & cửa hàng 6 & 1178.77  & 27       & 5        & 136.43   & 32       & 25       & 10.94    & 22.84    & 1146.03  & 4869.45  & 6015.48 \bigstrut\\
			\hline
			\multirow{2}[4]{*}{\textbf{Tầng 4}} & khối văn phòng 1 & 3913.98  & 91       & 8.5      & 770.11   & 32       & 25       & 10.94    & 22.84    & 6468.95  & 27486.46 & 33955.41 \bigstrut\\
			\cline{2-13}             & khối văn phòng 2 & 3913.98  & 91       & 8.5      & 770.11   & 32       & 25       & 10.94    & 22.84    & 6468.95  & 27486.46 & 33955.41 \bigstrut\\
			\hline
			\textbf{Tầng 5 - 27} & khối văn phòng & 8163.12  & 255      & 8.5      & 2168.33  & 32       & 25       & 10.94    & 22.84    & 18213.96 & 77390.85 & 2198910.68 \bigstrut\\
			\hline
		\end{tabular}%
		\end{adjustbox}
		\caption{Nhiệt do gió tươi mang vào}
		\label{b:ndgt}%
	\end{table}%
\end{landscape}
\textbf{Vậy tổng nhiệt lượng do gió tươi mang vào:} Q$_{5} = 2 400 660.6(W)$

\subsection{NHIỆT HIỆN \& ẨN DO LỌT GIÓ -- Q{\scriptsize 6}}
Để tiết kiệm năng lượng, phòng cần điều hoà phải được làm kín để ta chủ động cấp lượng không khí tươi cho phòng. Tuy nhiên vẫn có hiện tượng không khí tươi lọt vào phòng qua cửa ra vào, qua khe cửa sổ,…Mức độ rò rỉ phụ thuộc vào nhiều yếu tố: độ chênh áp bên trong và bên ngoài, tốc độ gió, số lần đóng mở cửa,..Lượng nhiệt đó được xác định như sau :
\begin{equation*}
	\begin{split}
	Q_{6} &= Q_{6h} + Q_{6a} \\
	Q_{6h} &= 0.39 \times \xi \times V \times (t_{N} - t_{T}), W \\
	Q_{6a} &= 0.84 \times \xi \times V \times (d_{N} - d_{T}), W		
	\end{split}
\end{equation*}

Trong đó:
\begin{itemize}
	\item $\xi$ - hệ số kinh nghiệm.
	\item $V$ - thể tích phòng, m$^3$.
\end{itemize}

Nếu ở không gian cần điều hoà có số người ra vào nhiều, cửa đóng mở nhiều lần thì cần phải bổ sung thêm vào lượng nhiệt hiện và ẩn sau:

\begin{equation*}
	\begin{split}
		Q_{bsh} &= 1.23\times L_{bs}\times (t_{N} - t_{T}), W\\
		Q_{bsa} &= 3\times L_{bs}\times (d_{N} - d_{T}), W\\
	\end{split}
\end{equation*}

Trong đó:
\begin{itemize}
	\item $t_{N}$, $t_{T}$ - nhiệt độ của không khí tươi bên ngoài và không khí trong không gian điều hoà, $^{\circ}C$.
	\item $d_{N}$, $d_{T}$ - độ chứa hơi của không khí tươi bên ngoài và không khí trong không gian điều hoà, g /kg.
	\item $ n $ - số người qua cửa trong một giờ.
	\item $L_{bs}$ - $= 0.28\times n\times L_{c}$, l/s.
	\item $L_{c}$ - Lượng không khí lọt qua mỗi lần mở cửa, m$^3$/người. 
\end{itemize}

\begin{landscape}
	\begin{table}[H]
		\vspace{0.5cm}
		\centering
	\begin{adjustbox}{width=1.5\textheight}
		\begin{tabular}{|c|l|r|r|r|r|r|r|r|r|r|}
			\hline
			\textbf{TẦNG} & \multicolumn{1}{c|}{\textbf{TÊN PHÒNG}} & \multicolumn{1}{p{6.645em}|}{\textbf{THỂ TÍCH (m3) }} & \multicolumn{1}{p{6.855em}|}{\textbf{HỆ SỐ KINH NGHIỆM}} & \multicolumn{1}{c|}{\textbf{tN}} & \multicolumn{1}{c|}{\textbf{tT}} & \multicolumn{1}{c|}{\textbf{dT}} & \multicolumn{1}{c|}{\textbf{dN}} & \multicolumn{1}{c|}{\textbf{Q6h}} & \multicolumn{1}{c|}{\textbf{Q6a}} & \multicolumn{1}{c|}{\textbf{Q6}} \bigstrut\\
			\hline
			\multirow{4}[8]{*}{\textbf{Tầng 1}} & sảnh văn phòng & 2293.11  & 0.42     & 32       & 25       & 10.94    & 22.84    & 2629.3   & 9624.9   & 12254 \bigstrut\\
			\cline{2-11}             & coffe    & 3228.98  & 0.35     & 32       & 25       & 10.94    & 22.84    & 3085.3   & 11294    & 14379 \bigstrut\\
			\cline{2-11}             & phòng máy & 260.76   & 0.7      & 32       & 25       & 10.94    & 22.84    & 498.32   & 1824.2   & 2322.5 \bigstrut\\
			\cline{2-11}             & phòng cấp cứu & 260.76   & 0.7      & 32       & 25       & 10.94    & 22.84    & 498.32   & 1824.2   & 2322.5 \bigstrut\\
			\hline
			\multirow{2}[4]{*}{\textbf{Tầng M}} & phòng điều khiển 1 & 1200.61  & 0.55     & 32       & 25       & 10.94    & 22.84    & 1802.7   & 6599.1   & 8401.8 \bigstrut\\
			\cline{2-11}             & phòng điều khiển 2 & 2982.98  & 0.4      & 32       & 25       & 10.94    & 22.84    & 3257.4   & 11924    & 15182 \bigstrut\\
			\hline
			\multirow{6}[12]{*}{\textbf{Tầng 2 - 3}} & cửa hàng 1 & 1178.77  & 0.55     & 32       & 25       & 10.94    & 22.84    & 1769.9   & 6479.1   & 8249 \bigstrut\\
			\cline{2-11}             & cửa hàng 2 & 904.41   & 0.6      & 32       & 25       & 10.94    & 22.84    & 1481.4   & 5423     & 6904.4 \bigstrut\\
			\cline{2-11}             & cửa hàng 3 & 897.76   & 0.6      & 32       & 25       & 10.94    & 22.84    & 1470.5   & 5383.1   & 6853.7 \bigstrut\\
			\cline{2-11}             & cửa hàng 4 & 897.76   & 0.6      & 32       & 25       & 10.94    & 22.84    & 1470.5   & 5383.1   & 6853.7 \bigstrut\\
			\cline{2-11}             & cửa hàng 5 & 904.41   & 0.6      & 32       & 25       & 10.94    & 22.84    & 1481.4   & 5423     & 6904.4 \bigstrut\\
			\cline{2-11}             & cửa hàng 6 & 1178.77  & 0.55     & 32       & 25       & 10.94    & 22.84    & 1769.9   & 6479.1   & 8249 \bigstrut\\
			\hline
			\multirow{2}[4]{*}{\textbf{Tầng 4}} & khối văn phòng 1 & 3913.98  & 0.35     & 32       & 25       & 10.94    & 22.84    & 3739.8   & 13690    & 17430 \bigstrut\\
			\cline{2-11}             & khối văn phòng 2 & 3913.98  & 0.35     & 32       & 25       & 10.94    & 22.84    & 3739.8   & 13690    & 17430 \bigstrut\\
			\hline
			\textbf{Tầng 5 - 27} & khối văn phòng & 8163.12  & 0.35     & 32       & 25       & 10.94    & 22.84    & 7799.9   & 28553    & 836108 \bigstrut\\
			\hline
		\end{tabular}%
	\end{adjustbox}
		\caption{Nhiệt do lọt gió}
		\label{b:ndlg}%
	\end{table}%
\end{landscape}
\textbf{Vậy tổng nhiệt lượng do gió tươi mang vào:} Q$_{6} = 969845(W)$

$\Rightarrow$ \textbf{Vậy tổng tải lạnh Q{\scriptsize 0} công trình là}: \textbf{6821.779267, KW}






%\section{XÂY DỰNG MÔ HÌNH 3D BẰNG \emph{REVIT}}
%Toà khách sạn Grand Plaza được xây dựng 27 tầng, mỗi tầng có nhiệm vụ và chức năng khác nhau, dưới đây là hình toà khách sạn được dựng bởi mô hình 3D bằng Revit.
%
%\begin{figure}[H]
%	\centering
%	\includegraphics[width=1\textwidth]{revit}
%	\caption{Mô hình 3D Revit}
%	\label{revit}
%\end{figure}
%
%\newpage
%\textbf{Tầng 1}: Được sử dụng làm phòng cấp cứu, coffee, phòng máy và sảnh văn phòng.
%
%\begin{figure}[H]
%	\centering
%	\includegraphics[width=1\textwidth]{1stcad}
%	\caption{Tầng 1 được vẽ bằng AutoCAD}
%	\label{1stcad}
%\end{figure}
%
%\begin{figure}[H]
%	\centering
%	\includegraphics[width=1\textwidth]{1strevit}
%	\caption{Tầng 1 được mô hình hoá bằng phần mềm Revit}
%	\label{1strevit}
%\end{figure}
%
%\newpage
%\textbf{Tầng M}: Được sử dụng làm phòng điều khiển.
%
%\begin{figure}[H]
%	\centering
%	\includegraphics[width=1\textwidth]{mcad}
%	\caption{Tầng M được vẽ bằng AutoCAD}
%	\label{mcad}
%\end{figure}
%
%\begin{figure}[H]
%	\centering
%	\includegraphics[width=1\textwidth]{mrevit}
%	\caption{Tầng M được mô hình hoá bằng phần mềm Revit}
%	\label{mrevit}
%\end{figure} 
%
%\newpage
%\textbf{Tầng 2 - 3}: Được sử dụng làm cửa hàng.
%
%\begin{figure}[H]
%	\centering
%	\includegraphics[width=1\textwidth]{23cad}
%	\caption{Tầng 2 - 3 được vẽ bằng AutoCAD}
%	\label{23cad}
%\end{figure}
%
%\begin{figure}[H]
%	\centering
%	\includegraphics[width=1\textwidth]{23revit}
%	\caption{Tầng 2 - 3 được mô hình hoá bằng phần mềm Revit}
%	\label{23revit}
%\end{figure} 
%
%\newpage
%\textbf{Tầng 4}: Được sử dụng làm văn phòng.
%
%\begin{figure}[H]
%	\centering
%	\includegraphics[width=1\textwidth]{4cad}
%	\caption{Tầng 4 được vẽ bằng AutoCAD}
%	\label{4cad}
%\end{figure}
%
%\begin{figure}[H]
%	\centering
%	\includegraphics[width=1\textwidth]{4revit}
%	\caption{Tầng 4 được mô hình hoá bằng phần mềm Revit}
%	\label{4revit}
%\end{figure} 
%
%\newpage
%\textbf{Tầng 5 - 27}: Được sử dụng làm văn phòng.
%
%\begin{figure}[H]
%	\centering
%	\includegraphics[width=0.9\textwidth]{527cad}
%	\caption{Tầng 5 - 27 được vẽ bằng AutoCAD}
%	\label{527cad}
%\end{figure}
%
%\begin{figure}[H]
%	\centering
%	\includegraphics[width=0.9\textwidth]{527revit}
%	\caption{\textbf{Tầng 5 - 27} được mô hình hoá bằng phần mềm Revit}
%	\label{527revit}
%\end{figure} 


\section{TÍNH TẢI LẠNH BẰNG \emph{REVIT MEP}}
