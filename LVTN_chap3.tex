\newpage
\cfoot{CHƯƠNG 3}
\lhead{TÍNH TOÁN PHỤ TẢI LẠNH CHO ĐHKK}

%Trang bìa
\newpage
\chapter{\textbf{TÍNH TOÁN PHỤ TẢI LẠNH CHO ĐHKK}}
\newpage

%Phần nội dung
\section{XÂY DỰNG MÔ HÌNH 3D BẰNG \emph{REVIT}}
\hspace{1cm}$ \clubsuit $ Nhân làm phần này. Sẽ add vào sau. $ \clubsuit $

+++++++++++++++++++++++++++++++++++++++++

\section{PHƯƠNG PHÁP TRUYỀN THỐNG}
\subsection{ĐẠI CƯƠNG}
\hspace{1cm}Các bước chủ yếu để tính toán cân bằng nhiệt ẩm truyền thống gồm \textbf{7 bước} để tính toán như sau:
\subsubsection{Xác định các nguồn nhiệt toả ra}
\hspace{1cm}- Các nguồn nhiệt này có thể xuất phát từ nhiều nguồn khác nhau, điển hình như: Do người, do máy móc, chiếu sáng, rò rỉ không khí, bức xạ mặt trời, thẩm thấu qua kết cấu,...

- Phương trình cân bằng nhiệt tổng quát có dạng:

\begin{center}
	\textit{Q{\footnotesize t}} = \textit{Q{\footnotesize toả} + Q{\footnotesize tt}}
\end{center}

\begin{itemize}[leftmargin = 3cm, label = $\star$]
	\item \textit{Q{\footnotesize t}} - nhiệt thừa trong phòng, \textit{W};
	
	\item \textit{Q{\footnotesize toả}} - nhiệt toả ra trong phòng, \textit{W};
	
	\item \textit{Q{\footnotesize tt}} - nhiệt thẩm thấu từ ngoài vào qua kết cấu bao che do chênh lệch nhiệt độ, \textit{W};
\end{itemize}


- Trong đó, nhiệt lượng \textbf{Q{\footnotesize toả}} có thể được phân thành \textit{8 phần nhiệt} như sau:

\begin{center}
	\textit{Q{\footnotesize toả}} = \textit{Q$ _{1} $ + Q$ _{2} $ + Q$ _{3} $ + Q$ _{4} $ + Q$ _{5} $ + Q$ _{6} $ + Q$ _{7} $ + Q$ _{8} $}
\end{center}

\begin{itemize}[leftmargin = 3cm, label = $\ast$]
	\item \textit{Q$ _{1} $} - Nhiệt toả từ máy móc;
	
	\item \textit{Q$ _{2} $} - Nhiệt toả từ đèn chiếu;
	
	\item \textit{Q$ _{3} $} - Nhiệt toả từ người;
	
	\item \textit{Q$ _{4} $} - Nhiệt toả từ bán thành phẩm;
	
	\item \textit{Q$ _{5} $} - Nhiệt toả từ bề mặt thiết bị trao đổi nhiệt;
	
	\item \textit{Q$ _{6} $} - Nhiệt toả do bức xạ mặt trời qua cửa kính;
	
	\item \textit{Q$ _{7} $} - Nhiệt toả do bức xạ mặt trời qua bao che;
	
	\item \textit{Q$ _{8} $} - Nhiệt toả do lò rọt không khí qua cửa;
\end{itemize}

- Lượng nhiệt từ \textbf{Q{\footnotesize tt}} có thể phân được phân thành 4 lượng nhiệt sau:

\begin{center}
	\textit{Q{\footnotesize tt}} = \textit{Q$ _{9} $ + Q$ _{10} $ + Q$ _{11} $ + Q{\footnotesize bs}}, \textit{W}
\end{center}

\begin{itemize}[leftmargin = 3cm, label = $\ast$]
	\item \textit{Q$ _{9} $} - Nhiệt thẩm thấu qua vách;
	
	\item \textit{Q$ _{10} $} - Nhiệt thẩm thấu qua trần (mái);
	
	\item \textit{Q$ _{11} $} - Nhiệt thẩm thấu qua nền;
	
	\item \textit{Q{\footnotesize bs}} - Nhiệt tổn thất bổ sung do gió và hướng vách;
\end{itemize}


\subsubsection{Xác định nguồn ẩm thừa trong phòng điều hoà W{\footnotesize t}:}

\begin{center}
	\textit{W{\footnotesize t}} = \textit{W$ _{1} $ + W$ _{2} $ + W$ _{3} $ + W$ _{4} $}, kg/s
\end{center}

\begin{itemize}[leftmargin = 3cm, label = $\ast$]
	\item \textit{W$ _{9} $} - Lượng ẩm do người toả vào phòng, \textit{kg/s};
	
	\item \textit{W$ _{10} $} - Lượng ẩm bay hơi từ bán thành phẩm, \textit{kg/s};

	\item \textit{W$ _{11} $} - Lượng ẩm do bay hơi từ sàn ẩm, \textit{kg/s};
	
	\item \textit{W{\footnotesize bs}} - Lượng ẩm do hơi nước nóng toả vào phòng, \textit{kg/s};	
\end{itemize}
	
\subsubsection{Xác định tia quá trình {\Large $\varepsilon$} (còn gọi là hệ số góc tia quá trình)}
\begin{center}
	{\Large $\varepsilon_{t}$} = $ \dfrac{Q_{t}}{W_{t}} $, \textit{kJ/kg}
\end{center}

\subsubsection{Xác định sơ đồ điều hoà không khí}
\hspace{1cm}- Trong bước này, chúng ta cần phải xác định được sơ đồ điều hoà không khí với các thông số trạng thái không khí trong nhà T, ngoài nhà N, hoà trộn H và thổi vào V ví dụ như entanpi I$_{T}$, I$_{N}$, I$_{H}$, I$_{V}$, nhiệt độ t$_{T}$, t$_{N}$, t$_{H}$, t$_{V}$, lưu lượng không khí G$_{T}$, G$_{N}$, G$_{H}$, G$_{V}$ (\textit{kg/s}), L$_{T}$, L$_{N}$, L$_{H}$, L$_{V}$ (\textit{$m^3/s$}), khối lượng riêng không khí $\rho_{T}$, $\rho_{N}$, $\rho_{H}$, $\rho_{V}$, ẩm dung của không khí d$_{T}$, d$_{N}$, d$_{H}$, d$_{V}$...

\subsubsection{Xác định năng suất gió của hệ thống}
\hspace{1cm}- Để có thể tải được hết nhiệt thừa ra khỏi phòng điều hoà cần một lượng gió G$_{t}$ là:

\begin{center}
	$ G_{t} = \dfrac{Q_{t}}{I_{T} - I_{V}} $, \textit{kg/s}
\end{center}

- Để có thể tải được hết ẩm thừa ra khỏi phòng điều hoà cần một lượng gió G$_{W}$ là:

\begin{center}
	$ G_{W} = \dfrac{W_{t}}{d_{T} - d_{V}} $, \textit{kg/s}
\end{center}

Năng suất gió của hệ thống G phải bằng G$_{t}$ và G$_{M}$ do đó:

\begin{center}
	G = G$_{t}$ = G$_{M}$
\end{center}

hay:

\begin{center}
	$ \dfrac{Q_{t}}{I_{T} - I_{V}} $ = $\dfrac{W_{t}}{d_{T} - d_{V}} $
\end{center}

rút ra:

\begin{center}
	$ \dfrac{Q_{t}}{W_{t}} $ = $ \dfrac{I_{T} - I_{V}}{d_{T} - d_{V}} $ = {\Large $\varepsilon_{t}$}
\end{center}

{\Large $\varepsilon_{t}$} chính là hệ số góc tia quá trình.

\subsubsection{Tính năng suất lạnh}
\hspace{1cm}- Năng suất lạnh của hệ thống điều hoà không khí Q$ _{0} $ có thể được tính như sau:

\begin{center}
	$Q _{0} = G_{V}(I_{H} - I_{V})$, \textit{kW}
\end{center}
\begin{center}
	$ Q _{0} = Q_{t}\times\dfrac{I_{H} - I_{V}}{I_{T} - I_{V}}$, \textit{kW}
\end{center}

\subsubsection{Lượng ẩm ngưng tụ trên dàn bay hơi W}
\begin{center}
	$W = G_{V}(d_{H}-d_{V})$, \textit{kg/s}
\end{center}

\subsection{TÍNH TOÁN NHIỆT ẨM CHO TOÀ NHÀ}




\section{PHƯƠNG PHÁP CARRIER}
\hspace{1cm}- 

\section{TÍNH TẢI LẠNH BẰNG \emph{REVIT MEP}}
\hspace{1cm}- 
