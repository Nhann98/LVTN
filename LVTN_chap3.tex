%Trang bìa
\chapmoi{TÍNH TOÁN PHỤ TẢI LẠNH CHO ĐHKK}

%Phần nội dung
\section{XÂY DỰNG MÔ HÌNH 3D BẰNG \emph{REVIT}}
	Toà khách sạn Grand Plaza được xây dựng 27 tầng, mỗi tầng có nhiệm vụ và chức năng khác nhau, dưới đây là hình toà khách sạn được dựng bởi mô hình 3D bằng Revit.

\begin{figure}[H]
  \centering
  \includegraphics[scale=1]{revit}
  \caption{Mô hình khách sạn Grand Plaza}
\end{figure} 

\newpage
\textbf{Tầng 1}: Được sử dụng làm phòng cấp cứu, coffee, phòng máy và sảnh văn phòng.
\begin{figure}[H]
  \centering
  \includegraphics[scale=0.5]{1stcad}
  \caption{Mặt bằng tầng 1 bằng Cad}
\end{figure} 

\begin{figure}[H]
  \centering
  \includegraphics[scale=0.8]{1strevit}
  \caption{Mặt bằng tầng 1 bằng Revit}
\end{figure} 

\newpage
\textbf{Tầng M}: Được sử dụng làm phòng điều khiển.

\begin{figure}[H]
  \centering
  \includegraphics[scale=0.5]{mcad}
  \caption{Mặt bằng tầng M bằng Cad}
\end{figure} 

\begin{figure}[H]
  \centering
  \includegraphics[scale=0.7]{mrevit}
  \caption{Mặt bằng tầng M bằng Revit}
\end{figure} 

\newpage
\textbf{Tầng 2 - 3}: Được sử dụng làm cửa hàng.

\begin{figure}[H]
  \centering
  \includegraphics[scale=0.5]{23cad}
  \caption{Mặt bằng tầng 2 - 3 bằng Cad}
\end{figure} 

\begin{figure}[H]
  \centering
  \includegraphics[scale=1]{23revit}
  \caption{Mặt bằng tầng 2 - 3 bằng Revit}
\end{figure} 

\newpage
\textbf{Tầng 4}: Được sử dụng làm văn phòng.

\begin{figure}[H]
  \centering
  \includegraphics[scale=0.5]{4cad}
  \caption{Mặt bằng tầng 4 bằng Cad}
\end{figure} 

\begin{figure}[H]
  \centering
  \includegraphics[scale=0.9]{4revit}
  \caption{Mặt bằng tầng 4 bằng Revit}
\end{figure} 

\newpage
\textbf{Tầng 5 - 27}: Được sử dụng làm văn phòng.

\begin{figure}[H]
  \centering
  \includegraphics[scale=0.5]{527cad}
  \caption{Mặt bằng tầng 5 - 27 bằng Cad}
\end{figure} 

\begin{figure}[H]
  \centering
  \includegraphics[scale=0.8]{527revit}
  \caption{Mặt bằng tầng 5 - 27 bằng Revit}
\end{figure} 

\section{PHƯƠNG PHÁP TRUYỀN THỐNG}
\subsection{ĐẠI CƯƠNG}
Các bước chủ yếu để tính toán cân bằng nhiệt ẩm truyền thống gồm \textbf{7 bước} để tính toán như sau:
\subsubsection{Xác định các nguồn nhiệt toả ra}
- Các nguồn nhiệt này có thể xuất phát từ nhiều nguồn khác nhau, điển hình như: Do người, do máy móc, chiếu sáng, rò rỉ không khí, bức xạ mặt trời, thẩm thấu qua kết cấu,...

- Phương trình cân bằng nhiệt tổng quát có dạng:

\begin{equation}
	Q_{t} = Q_{t1} + Q_{t2}
\end{equation}

\begin{itemize}[leftmargin = 3cm, label = $\star$]
	\item $Q_{t}$ - nhiệt thừa trong phòng, \textit{W};
	\item $Q_{t1}$ - nhiệt toả ra trong phòng, \textit{W};
	\item $Q_{t2}$ - nhiệt thẩm thấu từ ngoài vào qua kết cấu bao che do chênh lệch nhiệt độ, \textit{W};
\end{itemize}


- Trong đó, nhiệt lượng $ Q_{t1} $ có thể được phân thành \textit{8 phần nhiệt} như sau:

\begin{equation}
		Q_{t1} = Q_{1} + Q_{2} + Q_{3} + Q_{4} + Q_{5} + Q_{6} + Q_{7} + Q_{8}
\end{equation}

\begin{itemize}[leftmargin = 3cm, label = $\ast$]
	\item $Q_{1}$ - Nhiệt toả từ máy móc;
	\item $Q_{2}$ - Nhiệt toả từ đèn chiếu;
	\item $Q_{3}$ - Nhiệt toả từ người;
	\item $Q_{4}$ - Nhiệt toả từ bán thành phẩm;
	\item $Q_{5}$ - Nhiệt toả từ bề mặt thiết bị trao đổi nhiệt;
	\item $Q_{6}$ - Nhiệt toả do bức xạ mặt trời qua cửa kính;
	\item $Q_{7}$ - Nhiệt toả do bức xạ mặt trời qua bao che;
	\item $Q_{8}$ - Nhiệt toả do lò rọt không khí qua cửa;
\end{itemize}

- Lượng nhiệt từ $ Q_{t2} $ có thể phân được phân thành 4 lượng nhiệt sau:
\begin{equation}
	Q_{t2} = Q_{9} + Q_{10} + Q_{11} + Q_{bs}, \textit{W}
\end{equation}

\begin{itemize}[leftmargin = 3cm, label = $\ast$]
	\item $ Q_{9} $ - Nhiệt thẩm thấu qua vách;
	\item $ Q_{10} $ - Nhiệt thẩm thấu qua trần (mái);
	\item $ Q_{11} $ - Nhiệt thẩm thấu qua nền;
	\item $ Q_{bs} $ - Nhiệt tổn thất bổ sung do gió và hướng vách;
\end{itemize}


\subsubsection{Xác định nguồn ẩm thừa trong phòng điều hoà W$ _{t} $:}
\begin{equation}
	W_{t} = W_{1} + W_{2} + W_{3} + W_{4}, kg/s
\end{equation}

\begin{itemize}[leftmargin = 3cm, label = $\ast$]
	\item $ W_{1} $ - Lượng ẩm do người toả vào phòng, \textit{kg/s};
	\item $ W_{2} $ - Lượng ẩm bay hơi từ bán thành phẩm, \textit{kg/s};
	\item $ W_{3} $ - Lượng ẩm do bay hơi từ sàn ẩm, \textit{kg/s};
	\item $ W_{4} $ - Lượng ẩm do hơi nước nóng toả vào phòng, \textit{kg/s};	
\end{itemize}
	
\subsubsection{Xác định tia quá trình {\Large $\varepsilon$} (còn gọi là hệ số góc tia quá trình)}
\begin{equation}
	{\Large \varepsilon_{t}} =  \dfrac{Q_{t}}{W_{t}}, \textit{kJ/kg}
\end{equation}

\subsubsection{Xác định sơ đồ điều hoà không khí}
- Trong bước này, chúng ta cần phải xác định được sơ đồ điều hoà không khí với các thông số trạng thái không khí trong nhà T, ngoài nhà N, hoà trộn H và thổi vào V ví dụ như entanpi I$_{T}$, I$_{N}$, I$_{H}$, I$_{V}$, nhiệt độ t$_{T}$, t$_{N}$, t$_{H}$, t$_{V}$, lưu lượng không khí G$_{T}$, G$_{N}$, G$_{H}$, G$_{V}$ (\textit{kg/s}), L$_{T}$, L$_{N}$, L$_{H}$, L$_{V}$ (\textit{$m^3/s$}), khối lượng riêng không khí $\rho_{T}$, $\rho_{N}$, $\rho_{H}$, $\rho_{V}$, ẩm dung của không khí d$_{T}$, d$_{N}$, d$_{H}$, d$_{V}$...

\subsubsection{Xác định năng suất gió của hệ thống}
- Để có thể tải được hết nhiệt thừa ra khỏi phòng điều hoà cần một lượng gió G$_{t}$ là:

\begin{equation}
	G_{t} = \dfrac{Q_{t}}{I_{T} - I_{V}} , \textit{kg/s}
\end{equation}

- Để có thể tải được hết ẩm thừa ra khỏi phòng điều hoà cần một lượng gió G$_{W}$ là:

\begin{equation}
	 G_{W} = \dfrac{W_{t}}{d_{T} - d_{V}} , \textit{kg/s}
\end{equation}

Năng suất gió của hệ thống G phải bằng G$_{t}$ và G$_{M}$ do đó:

\begin{equation}
	G = G_{t} = G_{M}
\end{equation}

hay:

\begin{equation}
	 \dfrac{Q_{t}}{I_{T} - I_{V}}  = \dfrac{W_{t}}{d_{T} - d_{V}} 
\end{equation}

rút ra:

\begin{equation}
	 \dfrac{Q_{t}}{W_{t}}  =  \dfrac{I_{T} - I_{V}}{d_{T} - d_{V}}  = {\LARGE \varepsilon_{t}}
\end{equation}

{\Large $\varepsilon_{t}$} chính là hệ số góc tia quá trình.

\subsubsection{Tính năng suất lạnh}
- Năng suất lạnh của hệ thống điều hoà không khí Q$ _{0} $ có thể được tính như sau:
\begin{equation}
	Q _{0} = G_{V}(I_{H} - I_{V}), \textit{kW}
\end{equation}
\begin{equation}
	 Q _{0} = Q_{t}\times\dfrac{I_{H} - I_{V}}{I_{T} - I_{V}}, \textit{kW}
\end{equation}


\subsubsection{Lượng ẩm ngưng tụ trên dàn bay hơi W}
\begin{equation}
	W = G_{V}(d_{H}-d_{V}), \textit{kg/s}
\end{equation}

\subsection{ÁP DỤNG TÍNH TOÁN NHIỆT ẨM CHO TOÀ NHÀ}
\subsubsection{Các thông số đầu vào}
- Những thông số đầu vào để áp dụng cho tính toán nhiệt ẩm bao gồm các thông số sau:

\begin{enumerate}[leftmargin=2.2cm]
	\item Thông số từng tầng \& từng phòng.
	\item Cấu trúc kết cấu bao che.
	\item Nhiệt độ bên ngoài \& nhiệt độ yêu cầu của toà nhà.
	\item Hệ số kính; hướng kính \& diện tích kính của toà nhà.
	\item Hệ số thông gió.
\end{enumerate}

- Như đã nói ở trong phần giới thiệu, toà nhà được chia làm \textbf{27 tầng + 1 tầng lửng}. Dưới đây sẽ là bảng kích thước cụ thể cũng như từng tầng cũng như công năng từng phòng trong mỗi tầng.


\begin{table}[H]
	\centering
	\caption{BẢNG KÍCH THƯỚC PHÒNG}
	\begin{adjustbox}{width=\textwidth}
	\begin{tabular}{|c|l|r|r|r|r|r|r|}
		\toprule
		\textbf{Tầng} & \multicolumn{1}{c|}{\textbf{Tên phòng}} & \multicolumn{1}{p{4.715em}|}{\textbf{Dài - X\newline{}(mm)}} & \multicolumn{1}{p{4.57em}|}{\textbf{Rộng - Y\newline{}(mm) }} & \multicolumn{1}{p{4.855em}|}{\textbf{Cao - Z\newline{}(mm) }} & \multicolumn{1}{p{5.5em}|}{\textbf{S sàn + trần\newline{}(m$ ^{2} $)}} & \multicolumn{1}{p{4.93em}|}{\textbf{S tường 13\newline{}(m$ ^{2} $)}} & \multicolumn{1}{p{6.43em}|}{\textbf{S tường 24\newline{} (m$ ^{2} $)}} \\
		\midrule
		\multicolumn{1}{|c|}{\multirow{4}[8]{*}{\textbf{Tầng 1}}} & Sảnh văn phòng & 47438 & 10742 & 4500  & 510   & 213   & 48 \\
		\cmidrule{2-8}          & Cafeteria & 23285 & 30816 & 4500  & 718   & 105   & 139 \\
		\cmidrule{2-8}          & Phòng máy & 8380  & 6915  & 4500  & 58    & 38    & 31 \\
		\cmidrule{2-8}          & Phòng cấp cứu & 8380  & 6915  & 4500  & 58    & 38    & 31 \\
		\midrule
		\multirow{2}[4]{*}{\textbf{Tầng M}} & Phòng điều khiển 1 & 12380 & 21551 & 4500  & 267   & 56    & 97 \\
		\cmidrule{2-8}          & Phòng điều khiển 2 & 56900 & 11650 & 4500  & 663   & 256   & 52 \\
		\midrule
		\multirow{6}[12]{*}{\textbf{Tầng 2 - 3}} & Cửa hàng 1 & 22157 & 9852  & 5400  & 218   & 120   & 53 \\
		\cmidrule{2-8}          & Cửa hàng 2 & 17000 & 9852  & 5400  & 167   & 92    & 53 \\
		\cmidrule{2-8}          & Cửa hàng 3 & 16875 & 9852  & 5400  & 166   & 91    & 53 \\
		\cmidrule{2-8}          & Cửa hàng 4 & 16875 & 9852  & 5400  & 166   & 91    & 53 \\
		\cmidrule{2-8}          & Cửa hàng 5 & 17000 & 9852  & 5400  & 167   & 92    & 53 \\
		\cmidrule{2-8}          & Cửa hàng 6 & 22157 & 9852  & 5400  & 218   & 120   & 53 \\
		\midrule
		\multirow{2}[4]{*}{\textbf{Tầng 4}} & Khối văn phòng 1 & 69360 & 10450 & 5400  & 725   & 375   & 56 \\
		\cmidrule{2-8}          & Khối văn phòng 2 & 69360 & 10450 & 5400  & 725   & 375   & 56 \\
		\midrule
		\textbf{Tầng 5 - 27} & Khối văn phòng & 67800 & 30100 & 4000  & 2041  & 271   & 120 \\
		\bottomrule
	\end{tabular}
	\label{B:tsphong}
	\end{adjustbox}
\end{table}





\begin{minipage}[H]{.5\linewidth}
	\centering
	\includegraphics[width=\textwidth]{Wall_section.png}
	\captionof{figure}{Mặt cắt tường}
	\label{H:tstuong}
\end{minipage}
\begin{minipage}[H]{.5\linewidth}
	\begin{table}[H]
		\centering
		\begin{tabular}{|r|r|r|r|r|}
			\toprule
			\multicolumn{5}{|c|}{\textbf{Tường dày 200mm}} \\
			\midrule
			{\LARGE $\delta_{g}$} & 180   &       & {\LARGE $\rho_{g}$}  & 1800 \\
			\midrule
			{\LARGE $\delta_{v}$} & 10    &       & {\LARGE $\rho_{v}$}  & 1800 \\
			\bottomrule
		\end{tabular}%
		\caption{Bảng thông số tường}
		\label{B:tstuong}
	\end{table}
\end{minipage}

\vspace{0.5cm}
Và kết quả là Virk đã có hẳn một cuốn sách nói về giả thuyết cho rằng thế giới của chúng ta là một mô phỏng. Anh cũng đã vạch ra một con đường phát triển của công nghệ, để một ngày nào đó, chúng ta sẽ tiến được tới một "Điểm mô phỏng", ở đó, con người có thể tạo ra được những thế giới mô phỏng mới giống hệt thế giới của mình, không thể phân biệt được giống như Ma trận.

Trong khuôn khổ một buổi phỏng vấn với Vox, đây là những gì mà bạn có thể tìm hiểu về giả thuyết mô phỏng từ Rizwan Virk. Các câu trả lời phỏng vấn của anh đã được chỉnh sửa nhẹ để mọi thứ trở nên dễ hiểu hơn.



\begin{figure}[H]
		\centering
		\includegraphics[width=0.8\textwidth]{Floor_Section.png}
		\caption{Mặt cắt sàn}
		\label{H:tssan}
\end{figure}

\begin{wraptable}{l}{0.35\textwidth}
		\centering
		\begin{tabular}{|r|r|r|r|r|}
			\toprule
			\multicolumn{5}{|c|}{\textbf{Sàn}} \\
			\midrule
			{\LARGE $\delta_{bt}$} & 250 &      & {\LARGE $\rho_{bt}$} & 2400 \\
			\midrule
			{\LARGE $\delta_{v}$} & 50   &      & {\LARGE $\rho_{v}$} & 1800 \\
			\bottomrule
		\end{tabular}%
		\caption{Bảng thông số sàn}
		\label{B:tssan}%
\end{wraptable}


Có phải tất cả chúng ta chỉ đang sống trong một mô phỏng máy tính, một trò chơi điện tử mà ai đó "ngoài kia" đã lập trình lên? Câu hỏi có vẻ vô lý. Tuy nhiên, có rất nhiều người thông minh, bao gồm Elon Musk, Nick Bostrom và John Wheeler, một nhà vật lý sống ở thời đại Einstein cho rằng điều này hoàn toàn có thể.

Nick Bostrom là một triết gia tại Đại học Oxford. Ông nổi tiếng với một lập luận về giả thuyết mô phỏng cho rằng trong số 3 điều dưới đây, có ít nhất một điều có khả năng là đúng:

1) Tất cả các nền văn minh giống với nền văn minh của loài người trong vũ trụ đều tuyệt chủng trước khi phát triển được một trình độ công nghệ có thể tạo ra các thực tại mô phỏng.

2) Nếu có bất kỳ nền văn minh nào đạt được đến trình độ công nghệ này, không ai trong số họ còn bận tâm đến việc tạo ra một thế giới mô phỏng nữa (Có thể họ bận tâm đến những thứ khác, vượt ra ngoài những gì chúng ta đang suy nghĩ)

Có phải tất cả chúng ta chỉ đang sống trong một mô phỏng máy tính, một trò chơi điện tử mà ai đó "ngoài kia" đã lập trình lên? Câu hỏi có vẻ vô lý. Tuy nhiên, có rất nhiều người thông minh, bao gồm Elon Musk, Nick Bostrom và John Wheeler, một nhà vật lý sống ở thời đại Einstein cho rằng điều này hoàn toàn có thể.


\begin{figure}[H]
	\centering
	\includegraphics[width=0.8\textwidth]{Ceiling_section.png}
	\caption{Mặt cắt trần}
	\label{H:tstran}	
\end{figure}

\begin{wraptable}{l}{0.35\textwidth}
		\centering
		\begin{tabular}{|r|r|r|r|r|}
			\toprule
			\multicolumn{5}{|c|}{\textbf{Trần }} \\
			\midrule
			{\LARGE $\delta$}& 50    &       &{\LARGE $\rho$}& 1000 \\
			\bottomrule
		\end{tabular}%
		\label{B:tstran}%
		\caption{Bảng thông số trần}
\end{wraptable}

Và kết quả là Virk đã có hẳn một cuốn sách nói về giả thuyết cho rằng thế giới của chúng ta là một mô phỏng. Anh cũng đã vạch ra một con đường phát triển của công nghệ, để một ngày nào đó, chúng ta sẽ tiến được tới một "Điểm mô phỏng", ở đó, con người có thể tạo ra được những thế giới mô phỏng mới giống hệt thế giới của mình, không thể phân biệt được giống như Ma trận.

Trong khuôn khổ một buổi phỏng vấn với Vox, đây là những gì mà bạn có thể tìm hiểu về giả thuyết mô phỏng từ Rizwan Virk. Các câu trả lời phỏng vấn của anh đã được chỉnh sửa nhẹ để mọi thứ trở nên dễ hiểu hơn.




\section{PHƯƠNG PHÁP CARRIER}
-

\section{TÍNH TẢI LẠNH BẰNG \emph{REVIT MEP}}
- 
