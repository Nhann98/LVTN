\fancyhead[L]{\leftmark}
%Trang bìa
\chapmoi{HỆ THỐNG PHÒNG CHÁY CHỮA CHÁY}
\section{KHÁI QUÁT}
-- Công trình Ha Noi Plaza là một công trình xây dựng có quy mô lớn, với mục đích sử dụng chủ yếu là nhà làm việc văn phòng, khách sạn 5 sao, trung tâm thương mại. 

-- Đặc biệt đây là công trình có kiến trúc nhiều tầng nên việc tác chiến chữa cháy có những khó khăn nhất định. Do mức độ quan trọng trên nên việc đầu tư trang thiết bị PCCC cho công trình là một mục tiêu rất quan trọng và thiết thực.

-- Thực hiện ý tưởng trên chúng em đã chọn phương án thiết kế hệ thống PCCC cho công trình. Căn cứ vào yêu cầu đặc tính của công trình, tiêu chuẩn Việt Nam về an toàn phòng cháy chữa cháy để thiết kế các hệ thống phòng cháy chữa cháy cho công trình gồm các hạng mục sau:
\begin{itemize}[label={-}]
	\item Hệ thống báo cháy tự động.
	\item Hệ thống phát hiện \& báo cháy.
	\item Hệ thống chữa cháy Sprinkler tự động.
	\item Hệ thống chữa cháy màn ngăn nước tự động.
	\item Hệ thống trụ chữa cháy và chữa cháy cuộn vòi.
	\item Bình chữa cháy xách tay.
\end{itemize}
\subsection{CÁC TIÊU CHUẨN ÁP DỤNG}
Các tiêu chuẩn áp dụng cho hệ thống được liệt kê theo danh sách dưới đây:
\begin{enumerate}
	\item QCVN 06-2010/BXD –- Quy chuẩn kỹ thuật quốc gia về an toàn cháy cho nhà và công trình.
	\item TCVN 2622-1995 –- Phòng cháy, chống cháy cho nhà và công trình –- Yêu cầu thiết kế.
	\item TCVN 3890-2009 –- Phương tiện phòng cháy và chữa cháy cho nhà và công trình –- Trang bị, bố trí, kiểm tra, bảo dưỡng.
	\item TCCN 5738-2001 -- Hệ thống báo cháy –- Yêu cầu kỹ thuật.
	\item TCVN 5760-1993 –- Hệ thống chữa cháy – Yêu cầu chung về thiết kế, lắp đặt và sử dụng.
	\item TCVN 6101-1996 –- Thiết bị chữa cháy –- Hệ thống chữa cháy Cacbon Dioxit –- Thiết kế và lắp đặt.
	\item TCVN 6379-1998 –- Thiết bị chữa cháy –- Trụ nước chữa cháy – Yêu câu kỹ thuật
	\item TCVN 7161-1-2009 –- Hệ thống chữa cháy bằng khí –- TÍnh chất vật lý và thiết kế -- Phần 1: Yêu cầu chung
	\item TCVN 7568-1 $\sim$ 6:2013 –- Hệ thống phát hiện và báo cháy – Thiết bị
	\item TCVN 5738:2001 -- Hệ thống báo cháy tự động - Yêu cầu kĩ thuật
	\item TCVN 7336:2003 -- Hệ thống chữa cháy tự động sprinker – Yêu cầu thiết kế và lắp đặt.
	\item TCVN 7435-1:2004 -– Phòng cháy, chữa cháy -– Bình chữa cháy xách tay và xe đẩy chữa cháy.
	\item Tham chiếu các tiêu chuẩn của NFPA (National Fire Protection Association -- Hiệp hội Phòng cháy Quốc gia của Hoa Kỳ):
	
	NFPA 13: Tiêu chuẩn thiết kế, lắp đặt hệ thống chữa cháy sprinkler.
\end{enumerate}

\subsection{YÊU CẦU ĐỐI VỚI CÁC HỆ THỐNG PCCC CHO CÔNG TRÌNH}
Căn cứ vào tính chất nguy hiểm cháy nổ của công trình hệ thống PCCC cho công trình phải đảm bảo yêu cầu sau:
\subsubsection{Yêu cầu về phòng cháy:}
-- Phải áp dụng cac giải pháp phòng cháy đảm bảo hạn chế tối đa khả năng xảy ra hoả hoạn. Trong trường hợp xảy ra hoả hoạn thì phải phát hiện đám cháy nhanh để cứu chữa kịp thời không để đám cháy lan ra các khu vực khác sinh ra cháy lớn khó cứu chữa gây ra hậu quả nghiêm trọng.

-- Biện pháp phòng cháy phải đảm bảo sao cho khi có cháy thì người và tài sản trong toà nhà dễ dàng sơ tán sang khu vực an toàn một cách nhanh chóng nhất.

-- Trong bất cứ điều kiện nào khi xảy ra cháy ở những vị trí dễ xảy ra cháy như các khu vực kỹ thuật, sảnh trong trường phải phát hiện được ngay ở nơi phát sinh cháy để tổ chức cứu chữa kịp thời.
\subsubsection{Yêu cầu về chữa cháy:}
Trang thiết bị chữa cháy của công trình phải đảm bảo các yêu cầu sau:
\begin{itemize}
	\item Trang thiết bị chữa cháy phải sẵn sàng ở chế độ thường trực, khi xảy ra cháy phải sử dụng ngay.
	\item Thiết bị chữa cháy phải là loại phù hợp và chữa cháy có hiệu quả đối với các đám cháy xảy ra trong công trình.
	\item Thiết bị chữa cháy trang bị cho công trình phải là loại dễ sử dụng, phù hợp với công trình và điều kiện nước ta.
	\item Thiết bị chữa cháy phải là loại chữa cháy không làm hư hỏng các dụng cụ, thiết bị khác tại khu vực chữa cháy thiệt hại thứ cấp.
	\item Trang thiết bị hệ thống PCCC được trang bị phải đảm bảo điều kiện đầu tư tối thiểu nhưng đạt hiệu quả tối đa.
\end{itemize}

\section{THIẾT KẾ HỆ THỐNG BÁO CHÁY TỰ ĐỘNG}
\subsection{MÔ TẢ CHUNG}
Phương án thiết kế bao gồm:
\begin{itemize}[label={-}]
	\item Thiết kế lắp đặt hệ thống báo cháy tự động địa chỉ cho toàn bộ công trình.
	\item Thiết kế lắp đặt hệ thống chuông, nút ấn báo cháy cho toàn bộ công trình.
	\item Thiết kế lắp đặt đèn chỉ dẫn lối thoát nạn (EXIT) đèn sự cố khi có sự cố xảy ra.
\end{itemize}

\subsection{HỆ THỐNG BÁO CHÁY ĐỊA CHỈ TỰ ĐỘNG}
Hệ thống báo cháy tự động bao gồm các bộ phận cơ bản như:
\begin{enumerate}
	\item Các đầu cảm biến (Detector) phát hiện sự cháy.
	\item Nút ấn báo cháy tay.
	\item Các modul phân tích, xử lý tín hiệu.
	\item Trung tâm điều khiển xử lý các thông tin từ đầu cảm biến và nút ấn báo cháy tay đưa về.
	\item Bộ phận báo động cháy gồm: còi, chuông.
	\item Hệ thống dây dẫn: gồm hệ thống dây dẫn tín hiệu và dây cấp nguồn.
	\item Nguồn điện.
\end{enumerate}

Các thiết bị điều khiển ngoại vi như máy in dữ liệu báo cháy, tủ ghép nối điều khiển hệ thống trên máy tính, tủ ghép nối tín hiệu điều khiển hệ thống chữa cháy, hệ thống thang máy, cũng như đóng mở thiết bị thông gió, cửa thoát nạn.

\subsection{PHƯƠNG ÁN THIẾT KẾ}
Phương án thiết kế hệ thống báo cháy tự động cho toà nhà được chọn là hệ thống báo cháy địa chỉ thông minh.

Tổ hợp chuông, nút ấn báo cháy trên các tầng được bố trí tại các vị trí nhiều người đi lại như khu vực gần cầu thang máy và cầu thang bộ để thuận tiện cho việc quan sát xử lý sự cố khi có đám cháy xảy ra. Các nút ấn báo cháy trên mỗi tầng đều là nút ấn báo cháy thường kết hợp với module địa chỉ và được lắp trên 01 địa chỉ cho từng tầng.

Thiết bị báo động được chọn là chuông báo cháy. Trên các tầng chuông báo cháy được lắp đặt trong tổ hợp cùng nút ấn báo cháy. Các chuông được lắp đặt trên cùng 01 tuyến dây cấp nguồn và đưa về module địa chỉ cho chuông báo cháy để điều khiển hoạt
động của tất cả các chuông trên cùng một tầng.

Hệ thống dây dẫn tín hiệu cho đầu báo cháy địa chỉ là loại cáp chống nhiễu chuyên dụng và có tiết diện 1.2 mm$^2$.

\section{PHƯƠNG TIỆN CHỮA CHÁY BAN ĐẦU \& HỆ THỐNG CHỮA CHÁY BẰNG NƯỚC}
\subsection{KHÁI QUÁT}
\subsubsection{Căn cứ thiết kế}
-- Các tiêu chuẩn áp dụng, tiêu chuẩn và tài liệu tham khảo trong phần II - Các tiêu chuẩn áp dụng.

-- Bản vẽ kiến trúc của toà nhà.
\subsubsection{Mô tả chung hệ thống}
Sau khi nghiên cứu đặc điểm kiến trúc, quy mô, tầm quan trọng, tính chất sử dụng và mức độ nguy hiểm của công trình, giải pháp thiết kế hệ thống chữa cháy bằng nước cho công trình gồm các bộ phận cơ bản sau:
\begin{itemize}[label={-}]
	\item Bể chứa nước chữa cháy.
	\item Hệ thống bơm chữa cháy chính, máy bơm dự phòng, máy bơm bù áp lực.
	\item Tủ điều khiển bơm chữa cháy.
	\item Bình áp lực, đồng hồ đo áp lực, công tắc áp lực.
	\item Hệ thống Alarm van và công tắc dòng chảy.
	\item Hệ thống van chặn, van một chiều, van hút lọc.
	\item Hệ thống đường ống dẫn nước.
	\item Các đầu phun Sprinkler.
	\item Họng nước chữa cháy vách tường.
	\item Họng nước chữa cháy ngoài nhà.
	\item Phương tiện chữa cháy ban đầu sử dụng các bình chữa cháy xách tay ABC , bình khí CO2 cho các phòng cơ điện, phòng điều khiển, phòng máy tính.
\end{itemize}

\subsection{HỆ THỐNG CHỮA CHÁY BẰNG NƯỚC}
-- Hệ thống chữa cháy bằng nước được cấp tới các đầu phun và Sprinker và các họng nước chữa cháy vách tường, bằng mạng đường ống có đường kính từ D25-D150. Đường ống chính D150 chạy từ trạm bơm đến hộp kỹ thuật. Đường ống đứng trục chính đi trong hộp kỹ thuật sử dụng ống thép tráng kẽm D150 và D100 chạy dọc theo hộp kỹ thuật và được nối với bể nước tại tầng mái. Tại mỗi tầng có hai trục đường ống chính.

-- Việc cấp nước và tạo áp cho mạng đường ống chữa cháy được sử dụng chung giữa hệ thống Sprinkler và họng nước chữa cháy vách tường bằng bộ bơm chữa cháy trục ngang.

-- Tại các tầng thiết kế Sprinker bố trí các đồng hồ đo áp suất nhằm kiểm tra áp lực của mạng đường ống.

-- Việc khởi động và tắt máy bơm có thể hoàn toàn tự động hoặc bằng tay. Tủ điều khiển chữa cháy nhận tín hiệu từ công tắc áp lực, công tắc dòng chảy để đưa ra tín hiệu điều khiển chữa cháy đến tủ khởi động bơm chữa cháy. Tuỳ theo các trạng thái mà tủ điểu khiển chữa cháy quyết định đưa ra tín hiệu điều khiển khởi động bơm chữa cháy chính, bơm chữa cháy dự phòng hay bơm bù áp lực. Ngoài ra tủ điều khiển chữa cháy còn đưa ra tín hiệu báo động chữa cháy ra chuông, đèn báo chữa cháy cũng như trung tâm báo cháy tự động.

-- Khi áp suất trong hệ thống tụt xuống còn 90\% so với mức cài đặt trước thì công tắc áp suất sẽ khởi động bơm bù áp suất ( Jockey pump). Một Zole khống chế thời gian chạy tối thiểu được gắn vào hệ thống điều khiển để tránh trường hợp máy bơm bù bị khởi động liên tục.

-- Nếu như áp suất của hệ thống tụt xuống còn 80\% so với mức cài đặt từ trước thì bơm bù áp suất sẽ dừng và máy bơm chữa cháy chính sẽ được khởi động ( 01 máy bơm thường trực đã được lựa chọn).

-- Nguồn điện cấp cho máy bơm lấy từ nguồn ưu tiên, đồng thời được cấp bằng nguồn điện máy phát của toà nhà.

-- Bể nước phục vụ cho toàn hệ thống chữa cháy được dùng chung với bể nước sinh hoạt được đặt chìm tại tầng trệt. Lượng nước dùng cho chữa cháy được đảm bảo bởi các thiết bị kiểm tra mức nước và điều khiển bơm sinh hoạt.

\subsection{HỆ THỐNG CHỮA CHÁY SPRINKLER}
-- Hệ thống chữa cháy sử dụng các đầu Sprinkler hướng lên (Upright) được lắp đặt cho tầng hầm và các tầng không có trần giả, sử dụng Sprinkler quay xuống (Pendent) được bố trí cho các tầng có bố trí trần giả. Khoảng cách giữa các đầu phun là 2.6m đến 4m, khoảng cách đến tường 1.2m đến 2.0m (xem bản vẽ thiết kế).

-- Thông số kỹ thuật cơ bản để tính toán, thiết kế hệ thống “Theo TCVN 7336:2003” như bảng dưới đây:
\begin{table}[H]
	\centering
	\begin{tabular}{|l|l|}
		\hline
		\textbf{Diện tích bảo vệ tối đa cho một đầu phun: } & \textcolor[rgb]{ 1,  0,  0}{9-12m$^2$} \bigstrut\\
		\hline
		\textbf{Mức độ nguy hiểm:} & \textcolor[rgb]{ 1,  0,  0}{Thông thường} \bigstrut\\
		\hline
		\textbf{Áp suất tại đầu phun: } & \textcolor[rgb]{ 1,  0,  0}{1at (10m.c.n)} \bigstrut\\
		\hline
		\textbf{Cường độ phun :} & \textcolor[rgb]{ 1,  0,  0}{14,4 lit/phút m$^2$} \bigstrut\\
		\hline
		\textbf{Thời gian phun: } & \textcolor[rgb]{ 1,  0,  0}{30 phút} \bigstrut\\
		\hline
		\textbf{Diện tích được bảo vệ tính toán :} & \textcolor[rgb]{ 1,  0,  0}{200 m$^2$} \bigstrut\\
		\hline
	\end{tabular}%
	\label{b:pcccts}%
\end{table}%

-- Nguồn nước được cấp sẽ đủ cung cấp cho cả hệ thống Sprinkler và họng nước chữa cháy trong nhà hoạt động đồng thời.

\begin{table}[H]
	\centering
	\caption{Thống kê đầu Sprinkler}
	\begin{tabular}{|c|l|r|r|r|}
		\hline
		\multirow{2}[2]{*}{\textbf{ TẦNG}} & \multicolumn{1}{c|}{\multirow{2}[2]{*}{\textbf{TÊN PHÒNG}}} & \multicolumn{1}{c|}{\multirow{2}[2]{*}{\textbf{DIỆN TÍCH (m²) }}} & \multicolumn{1}{c|}{\multirow{2}[2]{*}{\textbf{UPRIGHT}}} & \multicolumn{1}{c|}{\multirow{2}[2]{*}{\textbf{PENDANT}}} \bigstrut[t]\\
		&          &          &          &  \bigstrut[b]\\
		\hline
		\multirow{7}[14]{*}{\textbf{Hầm B1}} & Kho M\&E & 215.2    & 18       & 0 \bigstrut\\
		\cline{2-5}             & Phòng điều khiển & 52       & 4        & 0 \bigstrut\\
		\cline{2-5}             & Phòng máy phát điện & 170      & 14       & 0 \bigstrut\\
		\cline{2-5}             & Phòng nồi hơi & 650      & 54       & 0 \bigstrut\\
		\cline{2-5}             & Phòng máy chiller & 910      & 76       & 0 \bigstrut\\
		\cline{2-5}             & Phòng ắc quy & 43       & 4        & 0 \bigstrut\\
		\cline{2-5}             & Phòng máy biến áp & 400      & 33       & 0 \bigstrut\\
		\hline
		\multirow{4}[8]{*}{\textbf{Tầng 1}} & sảnh văn phòng & 510      & 42       & 0 \bigstrut\\
		\cline{2-5}             & coffe    & 718      & 0        & 60 \bigstrut\\
		\cline{2-5}             & phòng máy & 58       & 0        & 5 \bigstrut\\
		\cline{2-5}             & phòng cấp cứu & 58       & 0        & 5 \bigstrut\\
		\hline
		\multirow{2}[4]{*}{\textbf{Tầng M}} & phòng điều khiển 1 & 267      & 0        & 22 \bigstrut\\
		\cline{2-5}             & phòng điều khiển 2 & 663      & 0        & 55 \bigstrut\\
		\hline
		\multirow{6}[12]{*}{\textbf{Tầng 2 - 3}} & cửa hàng 1 & 218      & 0        & 18 \bigstrut\\
		\cline{2-5}             & cửa hàng 2 & 167      & 0        & 14 \bigstrut\\
		\cline{2-5}             & cửa hàng 3 & 166      & 0        & 14 \bigstrut\\
		\cline{2-5}             & cửa hàng 4 & 166      & 0        & 14 \bigstrut\\
		\cline{2-5}             & cửa hàng 5 & 167      & 0        & 14 \bigstrut\\
		\cline{2-5}             & cửa hàng 6 & 218      & 0        & 18 \bigstrut\\
		\hline
		\multirow{2}[4]{*}{\textbf{Tầng 4}} & khối văn phòng 1 & 725      & 0        & 60 \bigstrut\\
		\cline{2-5}             & khối văn phòng 2 & 725      & 0        & 60 \bigstrut\\
		\hline
		\textbf{Tầng 5 - 27} & khối văn phòng & 2041     & 0        & 170 \bigstrut\\
		\hline
	\end{tabular}%
	\label{tab:addlabel}%
\end{table}%


\subsubsection{Tính toán thông số cần thiết cho hệ thống Sprikler}
Lưu lượng cần thiết lấy từ nguồn cấp nước cơ bản để hệ thống làm việc:
\begin{equation*}
	Q = I_{b} \times F
\end{equation*}

\break
Trong đó:
\begin{itemize}
	\item $I_{b}$ - Cường độ phun tiêu chuẩn 4,8 lít/m$^2$.phút.
	\item $F$ - Diện tích bảo vệ cùng một lúc khi hệ thống làm việc 12 m$^2$.
\end{itemize}
\begin{equation*}
	\Rightarrow Q = 4.8 \times 200 = 9.6 (l/s)
\end{equation*}

\subsection{HỆ THỐNG CHỮA CHÁY HỌNG NƯỚC VÁCH TƯỜNG}
-- Hệ thống chữa cháy họng nước vách tường được thiết kế chung với mạng đường ống hệ thống Sprinkler. Đây là hệ thống chữa cháy bán tự động, Công trình được sử dụng cuộn vòi D50 -- L = 30 m, lăng phun có đường kính miệng d = 13mm với lưu lượng phun
là 2,5 l/s, số vòi phun cho một đám cháy xẩy ra đồng thời là 2 vậy lưu lượng cần thiết là 5l/s. Đường ống đến các họng nước được rẽ nhánh từ trục chính tại các tầng có đường kính là D150 mm, hoặc D100 mm.

-- Cuộn vòi phải được chấp thuận và phải tương đương với cuộn vòi đã được chấp thuận bởi cơ quan Phòng cháy chữa cháy địa phương. Áp lực làm việc của cuộn vòi trong điều kiện bình thường phải đạt 10bar.

-- Trừ khi có chỉ định khác, tâm họng nước đặt ở độ cao cách mặt sàn 1,25 m. Toàn bộ họng nước đặt trong hộp chữa cháy đặt chìm trong tường, những nơi hộp chữa cháy nằm ở vị trí vách kính, tường bêtông thì hệ thống ống và hộp đi nổi bên ngoài và ống được sơn màu đỏ. (chi tiết bố trí, lắp đặt xem bản vẽ thiết kế).

-- Van góc đường kính D50mm. Áp suất làm việc 16 bars. Các van góc phải đạt tiêu chuẩn an toàn PCCC về khớp nối, van chữa cháy.

-- Bảng hướng dẫn sử dụng được để ở vị trí ngay sát với cuộn vòi ở vị trí chính diện dễ thấy.

\subsection{LỰA CHỌN MÁY BƠM CHỮA CHÁY}
\subsubsection{Áp lực công tác bơm}
\begin{equation*}
	H_{ct} = H_{hh} + H_{td} + \sum H
\end{equation*}

Trong đó:
\begin{itemize}
	\item $H_{hh}$: chiều cao mực nước thấp nhất trong bể chứa đến vòi chữa cháy cao và xa nhất là 110m.
	\item $H_{td}$: áp lực tự do cần thiết cho hệ thống đầu phun là 10m.
	\item $\sum H$: tổng tổn thất áp lực từ miệng hút máy bơm đến vòi nước chữa cháy cao và xa nhất (bao gồm tổn thất áp lực theo chiều dài và tổn thất cục bao qua các thiết bị van, tê, cút, ống vải, lăng phun …) là 90m.
\end{itemize}

Như vậy, $H_{ct}$ có kết quả là:
\begin{equation*}
	\begin{split}
		H_{ct} &= 110 + 10 + 90\\ 
		&= 210 m
	\end{split}
\end{equation*}

\subsubsection{Lưu lượng của bơm}
\vspace{-0.5cm}
\begin{equation*}
	Q_{ct} = 48 + 1 = 49 = 176.4 m^3/h
\end{equation*}

\subsubsection{Chọn bơm}
Từ các thông số trên, chọn bơm như sau: Q = 185 m$^3$/h, H = 210m (gồm 2 bơm chữa cháy, 1 bơm dự phòng và 1 bơm bù áp).
\begin{figure}[H]
	\centering
	\caption{Sơ đồ nguyên lý hệ thống chữa cháy toà nhà}
	\includegraphics[width=\textwidth]{pccc.png}
\end{figure}

\section{ĐẦU BÁO KHÓI}
-- Là thiết bị giám sát trực tiếp, phát hiện ra dấu hiệu khói để chuyển các tín hiệu khói về trung tâm xử lý. Thời gian các đầu báo khói nhận và truyền thông tin đến trung tâm báo cháy không quá 30s. Mật độ môi trường từ 15\% đến 20\%. Nếu nồng độ của khói trong môi trường tại khu vực vượt qua ngưỡng cho phép (10\% -20\%) thì thiết bị sẽ phát tín hiệu báo động về trung tâm để xử lý.

-- Các đầu báo khói thường được bố trí tại các phòng làm việc, hội trường, các kho quỹ, các khu vực có mật độ không gian kín và các chất gây cháy thường tạo khói trước.

\break
-- Đầu báo khói được chia làm 2 loại như sau:
\subsection{ĐẦU BÁO KHÓI DẠNG ĐIỂM}
-- Được lắp tại các khu vực mà phạm vi giám sát nhỏ, trần nhà thấp (văn phòng, chung cư ...).
\begin{enumerate}
	\item Đầu báo khói Ion : Thiết bị tạo ra các dòng ion dương và ion âm chuyển động, khi có khói, khói sẽ làm cản trở chuyển động của các ion dương và ion âm, từ đó thiết
	bị sẽ gởi tín hiệu báo cháy về trung tâm xử lý.
	\item Đầu báo khói quang (photo): Thiết bị bao gồm một cặp đầu báo (một đầu phát tín hiệu, một đầu thu tín hiệu) bố trí đối nhau, khi có khói xen giữa 2 đầu báo, khói sẽ làm cản trở đường truyền tín hiệu giữa 2 đầu báo, từ đó đầu báo sẽ gởi tín hiệu báo cháy về trung tâm xử lý.
\end{enumerate}

\subsection{ĐẦU BÁO KHÓI DẠNG BEAM}
-- Gồm một cặp thiết bị được lắp ở hai đầu của khu vực cần giám sát. Thiết bị chiếu phát chiếu một chùm tia hồng ngoại, qua khu vực thuộc phạm vi giám sát rồi tới một thiết bị nhận có chứa một tế bào cảm quang có nhiệm vụ theo dõi sự cân bằng tín hiệu của chùm tia sáng. Đầu báo này hoạt động trên nguyên lý làm mờ ánh sáng đối nghịch với nguyên lý tán xạ ánh sáng (cảm ứng khói ngay tại đầu báo).

-- Đầu báo khói loại Beam có tầm hoạt động rất rộng (15m x 100m), sử dụng thích hợp tại những khu vực mà các loại đầu báo khói quang điện tỏ ra không thích hợp, chẳng hạn như tại những nơi mà đám khói tiên liệu là sẽ có khói màu đen.

-- Hơn nữa đầu báo loại Beam có thể đương đầu với tình trạng khắc nghiệt về nhiệt độ, bụi bặm, độ ẩm quá mức, nhiều tạp chất,... Do đầu báo dạng Beam có thể đặt đằng sau cửa sổ có kiếng trong, nên rất dễ lau chùi, bảo quản.

-- Đầu báo dạng Beam thường được lắp trong khu vực có phạm vi giám sát lớn, trần nhà quá cao không thể lắp các đầu báo điểm (các nhà xưởng...).

\section{ĐẦU BÁO NHIỆT}
-- Đầu báo nhiệt là loại dùng để dò nhiệt độ của môi trường trong phạm vi bảo vệ, khi nhiệt độ của môi trường không thỏa mãn những quy định của các đầu báo nhiệt do nhà sản xuất quy định, thì nó sẽ phát tín hiệu báo động gởi về trung tâm xử lý.

-- Các đầu báo nhiệt được lắp đặt ở những nơi không thể lắp được đầu báo khói (nơi chứa thiết bị máy móc, Garage, các buồng điện động lực, nhà máy, nhà bếp,...).

\begin{figure}[H]
	\centering
	\includegraphics[width=0.7\textwidth]{pccc_daubaonhiet.jpg}
	\caption{Đầu báo nhiệt}
\end{figure}

\subsection{ĐẦU BÁO NHIỆT CỐ ĐỊNH}
-- Là loại đầu báo bị kích hoạt và phát tín hiệu báo động khi cảm ứng nhiệt độ trong bầu không khí chung quanh đầu báo tăng lên ở mức độ nhà sản xuất quy định (57$^{\circ}$, 70$^{\circ}$, 100$^{\circ}$...).
\subsection{ĐẦU BÁO NHIỆT GIA TĂNG}
-- Là loại đầu báo bị kích hoạt và phát tín hiệu báo động khi cảm ứng hiện tượng bầu không khí chung quanh đầu báo gia tăng nhiệt độ đột ngột khoảng 9$^{\circ}$C/phút.

\section{ĐẦU BÁO LỬA}
-- Là thiết bị cảm ứng các tia cực tím phát ra từ ngọn lửa, nhận tín hiệu, rồi gởi tín hiệu báo động về trung tâm xử lý khi phát hiện lửa.

-- Được sử dụng chủ yếu ở các nơi xét thấy có sự nguy hiểm cao độ, những nơi mà ánh sáng của ngọn lửa là dấu hiệu tiêu biểu cho sự cháy (ví dụ như kho chứa chất lỏng dễ cháy).

-- Đầu báo lửa rất nhạy cảm đối với các tia cực tím và đã được nghiên cứu tỉ mỉ để tránh tình trạng báo giả. Đầu dò chỉ phát tín hiệu báo động về trung tâm báo cháy khi có 2 xung cảm ứng tia cực tím sau 2 khoảng thời gian, mỗi thời kỳ là 5s.

\break
\section{CÔNG TẮC KHẨN}
\begin{wrapfigure}[10]{l}{0.4\textwidth}
	\vspace{-1cm}
	\centering
	\includegraphics[width=0.48\textwidth]{CONG-TAC-KHAN.jpg}
	\caption{Công tắc vuông}
\end{wrapfigure}

-- Được lắp đặt tại những nơi dễ thấy của hành lang cầu thang để sử dụng khi cần thiết. Thiết bị này cho phép người sử dụng chủ động truyền thông tin báo cháy bằng cách nhấn hoặc kéo vào công tắc khẩn, báo động khẩn cấp cho mọi người đang hiện diện trong khu vực đó được biết để có biện pháp xử lý hỏa hoạn và di chuyển ra khỏi khu vực nguy hiểm bằng các lối thóat hiểm.

-- Gồm có các loại công tắc khẩn như sau:
\begin{enumerate}[leftmargin=2.2cm]
	\item Khẩn tròn, vuông.
	\item Khẩn kính vỡ.
	\item Khẩn giật.
\end{enumerate}

\section{CHUÔNG BÁO CHÁY}
-- Được lắp đặt tại phòng bảo vệ, các phòng có nhân viên trực ban, hành lang, cầu thang hoặc những nơi đông người qua lại nhằm thông báo cho những người xung quanh có thể biết được sự cố đang xảy ra để có phương án xử lý, di tản kịp thời.

-- Khi xảy ra sự cố hỏa hoạn, chuông báo động sẽ phát tín hiệu báo động giúp cho nhân viên bảo vệ nhận biết và thông qua thiết bị theo dõi sự cố hỏa hoạn (bảng hiển thị phụ) sẽ biết khu vực nào xảy ra hỏa hoạn, từ đó thông báo kịp thời đến các nhân viên có trách nhiệm phòng cháy chữa cháy khắc phục sự cố hoặc có biện pháp xử lý thích hợp.

\begin{figure}[H]
	\centering
	\includegraphics[width=0.5\textwidth]{bell_firing.png}
	\caption{Chuông báo cháy}
\end{figure}

\section{CÒI BÁO CHÁY}
-- Có tính năng và vị trí lắp đặt giống như chuông báo cháy, tuy nhiên còi được sử dụng khi khoảng cách giữa nơi phát thông báo đến nơi cần nhận thông báo báo động quá xa.
\begin{figure}[H]
	\vspace{-1cm}
	\centering
	\includegraphics[width=0.45\textwidth]{coibaochay.jpg}
	\caption{Còi báo cháy}
\end{figure}

\section{ĐÈN BÁO CHÁY}
-- Có công dụng phát tín hiệu báo động, mỗi lọai đèn có chức năng khác nhau và được lắp đặt ở tại các vị trí thích hợp để phát huy tối đa tính năng của thiết bị này.

-- Gồm có các lọai đèn:
\subsection{ĐÈN CHỈ LỐI THOÁT HIỂM}
- Đèn chỉ dẫn thoát nạn Exit lắp đặt ở độ cao 2,5m. Đèn chỉ dẫn thoát nạn Exit được cấp nguồn AC 220V. Để duy trì đèn Exit luôn luôn sáng có 1 nguồn DC dự phòng tự động chuyển nguồn khi nguồn AC không có. Tuỳ từng vị trí lắp đặt, các đèn Exit phải có mũi tên chỉ hướng thoát nạn.

\begin{wrapfigure}[10]{r}{0.4\textwidth}
	\vspace{-0.5cm}
	\centering
	\includegraphics[width=0.3\textwidth]{exit.jpg}
	\caption{Đèn chỉ lối thoát hiểm}
\end{wrapfigure}
- Hệ thống chỉ dẫn lối thoát nạn và chiếu sáng sự cố chỉ dẫn cho người thoát ra khỏi công trình nhanh chóng khi có sự cố cháy xảy ra nhằm giảm thương vong về con người. Đèn hoạt động theo nguyên tắc: Khi chưa có sự cố mất điện, đèn hoạt động nhờ nguồn điện cấp từ tủ điện ánh sáng của tầng 220VAC. Ngoài ra các hộp đèn chỉ dẫn thoát nạn (EXIT) đều có nguồn ắc quy dự phòng, tự cung cấp điện cho đường chỉ dẫn khi mất hai nguồn trên trong một thời gian tối thiểu là 2 giờ.

- Đèn chiếu sáng sự cố lắp đặt trên lối thoát nạn: hành lang, cầu thang, chỗ khó di chuyển, chỗ rẽ. Khoảng cách không quá 30m.

- Đèn chiếu sáng sự cố có cường độ chiếu sáng ban đầu là 10 lux, và cường độ chiếu sáng tại bất kỳ điểm nào trên lối thoát nạn không nhỏ hơn 1 lux.

\subsection{ĐÈN BÁO CHÁY}
Được đặt bên trên công tắc khẩn của mỗi tầng. Đèn báo cháy sẽ sáng lên mỗi khi công t khẩn hoạt động, đồng thời đây cũng là đèn báo khẩn cấp cho những người hiện diện trong tòa nhà được biết. Điều này có ý nghĩa quan trọng, vì trong lúc bối rối do sự cố cháy, thì người sử dụng cần phân biệt rõ ràng công tác khẩn nào còn hiệu lực được kích hoạt máy bơm chữa cháy.
\begin{figure}[H]
	\centering
	\includegraphics[width=0.6\textwidth]{denbaochay.jpg}
	\caption{Đèn báo cháy}
\end{figure}




