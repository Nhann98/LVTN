\documentclass[a4paper]{report}

\usepackage{times} %dung de su dung font chu Times New Roman

\usepackage{multicol} %dung de su dung khi dung excel sang table Latex
\setlength{\columnsep}{1cm}

\usepackage{multirow} %dung de su dung khi dung excel sang table latex

\usepackage{multido} %dung de ve may dotted lines

\usepackage[utf8]{vietnam} %dung de viet tieng Viet

\usepackage{fancyhdr} %dung de tao ra cac style kho giay

\usepackage[Rejne]{fncychap}

\usepackage[a4paper,left=3cm, right=2cm, top=2cm, bottom=2cm, includefoot, includehead]{geometry} %dung de dinh dang kho giay

\usepackage{amsmath} %dung de tao ra cac phuong trinh toan hoc

\usepackage{fontawesome5} %package dùng để sử dụng icon để trang trí này nọ cho đẹp

\usepackage{wasysym} %dùng để sử dụng icon wasysysm

\usepackage{graphicx} %dung de su dung khi insert image 
<<<<<<< HEAD
\graphicspath{{E:/K22 Nhiet Lanh/LVTN/Files Latex LVTN/Pic/}}
=======
\graphicspath{{D:/DOCUMENTS/TOT NGHIEP/LVTN/Pic/}}
>>>>>>> 5bd27b94e1e9f4a66ad0fe62dbdc20c2adae49b4

\usepackage[titles]{tocloft} %Dùng để chỉnh mục lục - khoảng cách tiêu đề các thể loại
\cftsetindents{chapter}{0cm}{1cm} %chinh khoang cach trong TOC
\cftsetindents{section}{1cm}{1cm}
\cftsetindents{subsection}{1.5cm}{1.5cm}
\addtocontents{toc}{~\hfill\textbf{Trang}\par}
\usepackage{tocloft}
\setlength{\cftbeforechapskip}{1em}
\setlength{\cftbeforesecskip}{0.5em}

\usepackage[hidelinks, unicode]{hyperref} %dùng để làm toc có thể click được

\usepackage{float} %dung de tao, di chuyen cac object nhu table, image...

\usepackage{adjustbox} %dung de dieu chinh scale nhung table qua to

\usepackage{subcaption} %dùng để tạo subfigure

\usepackage{lscape} %dung de xoay ngang to giay

\usepackage{scrextend} %dung de dieu chinh size font chu
<<<<<<< HEAD
\changefontsizes{13pt}
=======
\label{key}\changefontsizes{12pt}
>>>>>>> 5bd27b94e1e9f4a66ad0fe62dbdc20c2adae49b4

\usepackage{indentfirst} %dùng để sau khi section thì dòng đầu tiên thục vào n cm
\setlength{\parindent}{1cm}
\renewcommand{\baselinestretch}{1.5} %dùng để giãn dòng 1.5

\widowpenalty=1000
\clubpenalty=1000

\usepackage{wrapfig} %dùng để wrap các text xung quanh table hoặc figure(image)

\usepackage{enumitem} %dùng để thiết kế lại các định dạng khi sử dụng list hoặc item, enumerate

\usepackage{booktabs} %dùng cho excel2latex

\usepackage{anyfontsize} %dùng cho mọi kích cỡ chữ

\usepackage[dvipsnames]{xcolor} %dung de highlight chu viet hoac doi mau chu

\usepackage{tcolorbox}

\usepackage[backend=bibtex, style=verbose-trad2]{biblatex} %dùng để tạo "TÀI LIỆU THAM KHẢO"
\citetrackerfalse
\interfootnotelinepenalty=10000 %dung de tranh tinh trang footcite qua trang khac
\addbibresource{thamkhao.bib}

%Phan nay la su dung package fancyhdr
\pagestyle{fancy}
\fancyhf{}
\renewcommand{\footrulewidth}{1pt}
\renewcommand*{\thesection}{\Alph{section}.}
\renewcommand*{\thesubsection}{\hspace{0.75cm}\Roman{subsection}.}
\renewcommand*{\thesubsubsection}{\hspace{1cm}\arabic{subsubsection}.}

% Đầu trang có số trang, số chương & chân trang có tên đề tài
\fancyhead[L]{\leftmark}
\fancyhead[R]{Trang \thepage}
\fancyfoot[C]{IoT HVAC}

%Tạo 1 lệnh để tạo chương mới, trình độ này thượng thừa vl ra
\newcommand{\chapmoi}[1]{
	\chapter{\textbf{#1}}
	\begin{center} %cái này là tạo biểu tượng dưới cái chương
		{\Huge \APLminus}
		{\Huge \faBookOpen}
		{\Huge \APLminus}
	\end{center}
	\newpage}

\newcommand{\dottedline}[1]{
	\par\nobreak
	\noindent\rule{0pt}{1.5\baselineskip}
	\multido{}{#1}{\noindent\makebox[\linewidth]{\dotfill}\endgraf}
	\bigskip}

%Đây chính là phần soạn thảo Latex, add các file vào để compile
%Khi compile thì compile file chính là đủ.
\begin{document}
	\fancyhead[L]{LỜI NÓI ĐẦU}
\begin{center}
	\textbf{{\Large \textcolor{red}{LỜI NÓI ĐẦU}}}
\end{center}

Nước ta hiện đang là một trong những quốc gia có mức độ công nghiệp hoá và hiện đại hoá đang phát triển rất mạnh mẽ trong những năm gần đây. Khi mà đời sống tăng cao thì nhu cầu về điều hoà càng cao, có thể nói hầu như trong tất cả các cao ốc, văn phòng, khách sạn, bệnh viện, nhà hàng, một số phân xưởng,… đã và đang xây dựng đều trang bị hệ thống điều hòa không khí. Mục đích của việc điều hòa không khí là tạo ra môi trường vi khí hậu thích hợp cho điều kiện sinh lý của con người và nâng cao độ tin cậy hoạt động của các trang thiết bị công nghệ.

Tuy nhiên, nếu chỉ đơn thuần là việc điều hoà không khí và vận hành thôi thì trong thời đại kỹ thuật số hiện nay là chưa đủ. Nhu cầu cấp tiến hơn cho các hệ thống lớn là ngoài việc vận hành phải đảm bảo được an toàn, ổn định thì còn phải đạt được tiêu chí tiết kiệm điện, giảm giá thành vận hành hệ thống. Hệ thống càng lớn thì mức độ tiết kiệm càng được để ý nhiều hơn. 

Với đề tài ``IoT HVAC''. Để thực nhiện đề tài này, chúng em đã vận dụng kiến thức, kinh nghiệp làm việc và các tài liệu liên quan để tính toán, thiết kế dưới sự hướng dẫn tận tình của thầy Đỗ Trí Nhựt.

Vì đây là lần đầu tiên thực hiện việc tính toán, thiết kế cho một công trình lớn, hơn nữa kiến thức chuyên môn còn hạn chế và chưa có kinh nghiệm thực tế nên sẽ có nhiều thiếu sót. Kính mong được sự chỉ bảo cũng như góp ý quý báo của quý thầy cô để chúng em có được kinh nghiệm và tiến bộ sau này. Chúng em xin chân thành cảm ơn!

\begin{flushright}
	Hồ Chí Minh, tháng 02 năm 2020
	
	Sinh viên thực hiện
	
	Nguyễn Phúc Thịnh
	
	Nguyễn Thành Nhân
	
	Nguyễn Vũ Trường
\end{flushright}

	\include{nhanxet&loicamon}
	\tableofcontents %đây là lệnh tạo mục lục
\setcounter{secnumdepth}{3} %report không có đánh số cho subsubsection nên lệnh này dùng đánh số cho subsubsection

<<<<<<< HEAD
	\fancyhead[L]{\leftmark}
%Trang bìa
\chapmoi{TỔNG QUAN VỀ CÔNG TRÌNH THIẾT KẾ}

%nội dung
	\section{KHÁI QUÁT VỀ CÔNG TRÌNH}
	\subsection{VỊ TRÍ CÔNG TRÌNH}
	- Nằm toạ lạc trên đường Trần Duy Hưng (\emph{địa chỉ cụ thể là: \textbf{117 Trần Duy Hưng, Trung Hoà, Cầu Giấy, Hà Nội}}) và có vị trí chiến lược trong việc phát triển Hà Nội.
	
	 - Nằm gần khu dân cư và trung tâm trọng yếu như Khu đô thị Trung Hoà - Nhân Chính, trung tâm Hội nghị Quốc gia, trung tâm Triển lãm Quốc gia, trường THPT Chuyên Amsterdam ...

\begin{figure}[H]
	\centering
	\includegraphics[width=0.7\textheight]{Google_maps.png}
	\caption{Vị trí công trình thông qua Google Maps}
\end{figure}
	
	\subsection{MỤC ĐÍCH SỬ DỤNG}
	- Với diện tích khu đất 19 689m$^{2}$, diện tích xây dựng là 7 799m$^{2}$, do Tập đoàn Charm Vit, Hàn Quốc phát triển với mức phát triển với mức đầu tư trên 120 triệu USD, bao gồm một toà tháp văn phòng hạng A 27 tầng, một toà tháp khách sạn 5 sao 27 tầng và một khu trung tâm thương mại cao cấp 5 tầng.

\begin{figure}[H]
	\centering
	\includegraphics[width=0.7\textheight]{ban-ve-khach-san-5-sao-Ha-noi-Plaza_3.jpg}
	\caption{Phối cảnh công trình}
\end{figure}

	- Khách sạn có quy mô 27 tầng và 2 tầng hầm, chiều cao trên 100m với tổng diện tích đất 19 689m$^{2}$, diện tích sàn xây dựng là 150 000m$^{2}$.
	
	- Tầng 1 đến tầng 4 dùng vào các hoạt động dịch vụ như: trung tâm thương mại, siêu thị và nhà hàng ăn uống Âu và Á,...
	
	- Ngoài ra, Ha Noi Plaza còn có hệ thống 1 phòng họp có chứa được khoảng 800 người, có sân golf tập, bể bơi trong nhà và ngoài trời và một số dịch vụ công cộng...
		
	- Từ tầng 5 đến tầng 27 với diện tích 101 104m$^{2}$ và được làm văn phòng cho thuê với diện tích 53 443m$^{2}$ (riêng tầng 26 và tầng 27 được dùng làm nhà hàng).	
		
	- Khu mua sắm 5 tầng này có diện tích lên tới 15 000m$^{2}$, sẽ là trung tâm thương mại cao cấp đầu tiên trong khu vực.
	
	\subsection{TÓM TẮT CÁC TẦNG CỦA TOÀ NHÀ}
	\begin{itemize}
	\setlength\itemsep{1mm}
		\item \emph{Tầng 1}: Là không gian sang trọng với đồ trang sức cao cấp, mỹ phẩm, nước hoa và đồng hồ.
	
		\item \emph{Tầng M}: Là nơi tập trung các gian hàng thời trang, phụ kiện thời trang, đồ da thương hiệu quốc tế cho cả nam lẫn nữ.
	
		\item \emph{Tầng 2}: Là các gian hàng thời trang công sở, trang phục hàng ngày thương hiệu mạnh Việt Nam, đồ lưu niệm, trang phục và dụng cụ thể thao.
	
		\item \emph{Tầng 3}: Là khu mua sắm cho mẹ và bé, đồ trang trí nội thất, đồ gia dụng, chăn nệm, đồ điện tử cao cấp.
	
		\item \emph{Tầng 4}: Là khu ẩm thực với 17 quầy food court đa dạng, 2 quán cà phê và một nhà hàng rộng 400m$^{2}$.
	\end{itemize}
	
	- \textbf{Đặc biệt}, tại tầng M sẽ là một siêu thị mini rộng gần 400m$^{2}$, tầng 2 là khu vui chơi giải trí dành cho các bé và các máy trò chơi cho thanh thiếu niên, tầng sẽ là một showroom trang trí nội thất sang trọng.
	
	\section{KHÍ HẬU}
	- Hà Nội là khí hậu nhiệt đới gió mùa ẩm, mùa hè nóng, mưa nhiều và mùa đông lạnh, ít mưa. Thời tiết tại đây được chia làm 2 mùa: \textbf{mùa mưa} (\emph{từ tháng 4 đến tháng 10}) và \textbf{mùa khô} (\emph{từ tháng 11 đến tháng 3}).
	
	- Mùa nóng bắt đầu từ tháng 5 đến tháng 8, khí hậu nóng ẩm vào đầu mùa và cuối mùa mưa nhiều, khô ráo vào tháng 9 và tháng 10, mùa lạnh bắt đầu từ tháng 11 đến tháng 3 năm sau.
	
	- Từ cuối tháng 11 đến nửa đầu tháng 2 rét và hành khô, từ nửa cuối tháng 2 đến hết tháng 3 lạnh và mưa phùn kéo dài từng đợt, trong khoảng tháng 9 đến giữa tháng 11, Hà Nội có những ngày thu với tiết trời mát mẻ.
	
	- Nhiệt độ trung bình mùa đông là 16.4$^{\circ}$C, trung bình mùa hạ 29.2$^{\circ}$C (lúc cao nhất lên tới 42.8$^{\circ}$C). Nhiệt độ trung bình cả năm 23.6$^{\circ}$C, lượng mưa trung bình hàng năm vào mức 1800mm đến 2000mm, do chịu ảnh hưởng của hiệu ứng đô thị và là vùng khí hậu có độ ẩm cao nên những đợt nắng nóng, nhiệt độ cảm nhận thực tế luôn cao hơn mức đo đạc, có thể lên tới 50$^{\circ}$C.
	
	- Lượng bức xạ tổng cộng trung bình hằng năm ở Hà Nội là 122.8 $ kcal/cm^{2} $ với 1641 giờ nắng và nhiệt độ không khí trung bình hằng năm là 23.6$^{\circ}$C, cao nhất là tháng 6 (29.8$^{\circ}$C), thấp nhất là tháng 1 (17.2$^{\circ}$C). Hà Nội có độ ẩm \& lượng mưa khá lớn. Độ ẩm tương đối lớn trung bình hàng năm là 79\%. Lượng mưa trung bình hàng năm là 1800mm và mỗi năm có khoảng 144 ngày mưa.
	
	\section{CẤP ĐIỆN - NĂNG LƯỢNG CHO TOÀ NHÀ}
	Hệ thống điện nặng là hệ thống điện chính của tòa nhà bao gồm hệ thống Điện Động Lực và hệ thống Điện Điều Khiển. Sử dụng nguồn điện chính 3 Pha 380 Volt hoặc 1 pha 220 Volt.
	
	- Nguồn Cấp Điện Chính:
	Trạm Biến Áp Điện Lực + Tủ Tụ Bù ==> ATS + Máy Phát ==> UPS lưu Điện ==> Tải sử dụng Trực tiếp.
	
	- Tải Sử Dụng Trực Tiếp: Từng căn hộ sử dụng điện 1 pha, Máy Bơm Cấp Thoát Nước, Thang Máy, Điều Hòa v.v...
	
	\begin{figure}[H]
		\centering
		\caption{Sơ Đồ Hệ Thống Cơ Điện cho toà nhà}
		\includegraphics[scale=0.6]{sodohethongcodien.jpg}
	\end{figure}
	
	%Trang bìa
\chapmoi{SƠ LƯỢC VỀ HỆ THỐNG CƠ ĐIỆN}

	\section{KỸ THUẬT LẠNH ỨNG DỤNG ĐIỀU HOÀ KHÔNG KHÍ}
	\subsection{GIỚI THIỆU VỀ KỸ THUẬT LẠNH}
	Kỹ thuật lạnh đã ra đời từ rất lâu và được sử dụng trong rất nhiều ngành nghề kỹ thuật khác nhau: trong công nghiệp chế biến \& bảo quản thực phẩm, công nghiệp hoá chất, công nghiệp rượu bia, kỹ thuật sấy nhiệt độ thấp, công nghiệp dầu mỏ, chế tạo vật liệu, dụng cụ, xử lý hạt giống, v.v...
	
	\subsection{ĐIỀU HOÀ KHÔNG KHÍ}
	\subsubsection{Giới thiệu về Hệ Thống Điều Hoà Không Khí}
	Hệ thống điều hòa không khí hiện đại đầu tiên được phát triển vào năm 1902 bởi một kỹ sư trẻ tên là \textbf{Willis Haviland Carrier}. 
	
	\begin{wrapfigure}{r}{0.3\textwidth}
		\includegraphics[width=0.9\linewidth]{WillisHavilandCarrier.jpg}
		\caption{Chân dung ông Willis Haviland Carrier}
	\end{wrapfigure}

	Ban đầu hệ thống được thiết kế để làm giảm độ ẩm của không khí trong xưởng in của một công ty xuất bằng cách thổi nó qua ống ướp lạnh. Không khí được làm mát khi nó đi qua các đường ống lạnh và trở lên khô hơn. Quá trình làm giảm độ ẩm trong nhà máy đã tạo ra một lợi ích phụ là giảm nhiệt độ không khí và một công nghệ mới đã được sinh ra. Đó là công nghệ điều hòa không khí.
	
	Khi một chất lỏng chuyển thành khí (trong một quá trình được gọi là \emph{chuyển đổi pha}), nó hấp thụ nhiệt của môi trường xung quanh. Điều hòa không khí khai thác tính năng này của giai đoạn chuyển đổi bằng cách buộc các hợp chất hóa học đặc biệt để bay hơi và ngưng tụ hơn và hơn nữa trong một hệ thống khép kín của cuộn dây.
	
	\subsubsection{Nguyên lý hoạt động}
	Hệ thống làm lạnh không khí gồm có một máy nén khí bơm gas (môi chất lạnh) áp suất cao đến dàn nóng (outdoor), tại đây khí gas dưới áp suất lớn sẽ hóa lỏng và tỏa nhiệt ra môi trường bên ngoài nhờ quạt gió (gia dụng) hoặc tháp nước (công nghiệp), hoặc bình ngưng.

	Sau đó gas dưới dạng lỏng tuần hoàn đến van tiết lưu (van này có tác dụng tạo chênh lệch áp suất cần thiết cho hệ thống). Ở đây gas từ dạng lỏng ấp suất cao sẽ được tiết lưu về dạng khí áp suất thấp, nhiệt độ thấp phun vào dàn lạnh (indoor) và thu nhiệt từ môi trường cần làm lạnh nhờ hệ thống quạt và các tấm lược gió trên các ống dẫn gas. Sau đó gas ở trạng thái khí được máy nén hút về để bơm tiếp một chu trình mới.	
	
	\begin{figure}[H]
		\centering
		\includegraphics[width=0.9\linewidth]{cau-tao-va-nguyen-ly-hoat-dong-cua-may-lanh-4.jpg} 
		\caption{Sơ đồ mô tả nguyên lý hoạt động của hệ thống điều hoà}
	\end{figure}
	
	\subsection{PHÂN LOẠI HỆ THỐNG ĐIỀU HOÀ}
	\subsubsection{PHÂN LOẠI THEO ĐẶC TÍNH}
	\textbf{Theo mức độ quan trọng của hệ thống điều hòa không khí}
	\begin{enumerate}
		\setlength\itemsep{1mm}
		\item \textbf{Hệ thống điều hòa không khí cấp I:} Là hệ thống điều hoà có khả năng duy trì các thông số vi khí hậu trong nhà với mọi phạm vi thông số ngoài trời, ngay tại cả ở những thời điểm khắc nghiệt nhất trong năm về mùa Hè lẫn mùa Đông.
		\item \textbf{Hệ thống điều hòa không khí cấp II:} Là hệ thống điều hoà có khả năng duy trì các thông số vi khí hậu trong nhà với sai số không qúa 200 giờ trong 1 năm, tức tương đương khoảng 8 ngày trong 1 năm. Điều đó có nghĩa trong 1 năm ở những ngày khắc nghiệt nhất về mùa Hè và mùa Đông hệ thống có thể có sai số nhất định, nhưng số lượng những ngày đó cũng chỉ xấp xỉ 4 ngày trong một mùa.
		\item \textbf{Hệ thống điều hòa không khí cấp III:} Là hệ thống điều hoà có khả năng duy trì các thông số tính toán trong nhà với sai số không quá 400 giờ trong 1 năm, tương đương 17 ngày.
	\end{enumerate}
	
	\textbf{Theo phương pháp xử lý nhiệt ẩm}
	\begin{enumerate}
		\setlength\itemsep{1mm}
		\item \textbf{Hệ thống điều hòa kiểu khô:} Không khí được xử lý nhiệt ẩm nhờ các thiết bị trao đổi nhiệt kiểu bề mặt. Đặc điểm của việc xử lý không khí qua các thiết bị trao đổi nhiệt kiểu bề mặt là không có khả năng làm tăng dung ẩm của không khí. Quá trình xử lý không khí qua các thiết bị trao đổi nhiệt kiểu bề mặt tuỳ thuộc vào nhiệt độ bề mặt mà dung ẩm không đổi hoặc giảm. Khi nhiệt độ bề mặt thiết bị nhỏ hơn nhiệt độ đọng sương ts của không khí đi qua thì hơi ẩm trong nó sẽ ngưng tụ lại trên bề mặt của thiết bị, kết quả dung ẩm giảm. Trên thực tế, quá trình xử lý luôn luôn làm giảm dung ẩm của không khí.
		\item \textbf{Hệ thống điều hòa không khí kiểu ướt:} Không khí được xử lý qua các thiết bị trao đổi nhiệt hỗn hợp. Trong thiết bị này không khí sẽ hỗn hợp với nước phun đã qua xử lý để trao đổi nhiệt ẩm. Kết quả quá trình trao đổi nhiệt ẩm có thể làm tăng, giảm hoặc duy trì không đổi dung ẩm không khí.
	\end{enumerate}
	
	\textbf{Theo đặc điểm của khâu xử lý nhiệt}
	\begin{enumerate}
		\setlength\itemsep{1mm}
		\item \textbf{Hệ thống điều hòa cục bộ:} Là hệ thống điều hoà không khí trong một không gian hẹp, thường là phòng. Kiểu điều hoà cục bộ trên thực tế chủ yếu sử dụng các máy điều hoà dạng cửa sổ, máy điều hoà kiểu rời (2 mảnh) và máy điều hoà ghép.
		\item \textbf{Hệ thống điều hòa phân tán:} Máy điều hoà VRV do hãng Daikin của Nhật phát minh đầu tiên. Hiện nay hầu hết các hãng đã sản xuất các máy điều hoà VRV và đặt dưới các tên gọi khác nhau , nhưng về mặt bản chất thì không có gì khác.
		
	+ Tên gọi VRV xuất phát từ các chữ đầu tiếng Anh : \textit{Variable Refrigerant Volume}, nghĩa là hệ thống điều hoà có khả năng điều chỉnh lưu lượng môi chất tuần hoàn và qua đó có thể thay đổi công suất theo phụ tải bên ngoài.

	+ Máy điều hoà VRV ra đời nhằm khắc phục nhược điểm của máy điều hoà dạng rời là độ dài đường ống dẫn ga, chênh lệch độ cao giữa dàn nóng, dàn lạnh và công suất lạnh bị hạn chế. Với máy điều hoà VRV cho phép có thể kéo dài khoảng cách giữa dàn nóng và dàn lạnh lên đến 100m và chênh lệch độ cao đạt 50m. Công suất máy điều hoà VRV cũng đạt giá trị công suất trung bình.
	
		\item \textbf{Hệ thống điều hòa trung tâm:} Hệ thống điều hoà trung tâm là hệ thống mà khâu xử lý không khí thực hiện tại một trung tâm sau đó được dẫn theo hệ thống kênh dẫn gió đến các hộ tiêu thụ. Hệ thống điều hoà trung tâm trên thực tế là máy điều hoà dạng tủ, ở đó không khí được xử lý nhiệt ẩm tại tủ máy điều hoà rồi được dẫn theo hệ thống kênh dẫn đến các phòng.
		
		\begin{figure}[H]
			\centering
			\includegraphics[scale=0.55]{VRV.png}
			\caption{Điều hoà trung tâm VRV}
		\end{figure}

	Tuy nhiên hệ thống này có kênh gió quá lớn (80.000 $BTU/h$ trở lên) nên chỉ có thể sử dụng trong các toà nhà có không gian lắp đặt lớn. Đối với hệ thống điều hoà trung tâm do xử lý nhiệt ẩm tại một nơi duy nhất nên chỉ thích hợp cho các phòng lớn, đông người. Đối với các toà nhà làm việc, khách sạn, công sở,... là các đối tượng có nhiều phòng nhỏ với các chế độ hoạt động khác nhau, không gian lắp đặt bé, tính đồng thời làm việc không cao thì hệ thống này không thích hợp.
	\end{enumerate}
	
	\textbf{Theo đặc điểm của môi chất giải nhiệt}
	\begin{enumerate}
		\setlength\itemsep{1mm}
		\item \textbf{Giải nhiệt bằng gió:} 
		
		$ \bullet $ Tất cả các máy điều hoà công suất nhỏ đều giải nhiệt bằng không khí, các máy điều hoà công suất trung bình có thể giải nhiệt bằng gió hoặc nước, hầu hết các máy công suất lớn đều giải nhiệt bằng nước.
		
		$ \bullet $ Máy lạnh trung tâm sản xuất ra gió lạnh và cấp tới các không gian điều hoà qua các kênh dẫn gió. Lúc này, gió đóng vai trò trao đổi và thực hiện quá trình tăng giảm nhiệt, ẩm của không gian kín. Kết thúc công đoạn này, gió lạnh lại tuần hoàn về máy lạnh trung tâm qua một kênh dẫn gió khác (hoặc hồi trực tiếp về buồng máy) và tiếp tục một chu trình mới.
		
		$ \bullet $ Hệ thống này bao gồm: máy lạnh trung tâm, các kênh dẫn gió và phân phối gió lạnh, thiết bị giải nhiệt dàn ngưng…
		
		\item \textbf{Giải nhiệt bằng nước:} 
		
		$ \bullet $ Để nâng cao hiệu quả giải nhiệt các máy công suất lớn sử dụng nước để giải nhiệt cho thiết bị ngưng tụ. Đối với các hệ thống này đòi hỏi trang bị đi kèm là hệ thống bơm, tháp giải nhiệt và đường ống dẫn nước.
		
		$ \bullet $ Nước lạnh sản xuất ra tại các máy lạnh trung tâm được cấp tới các dàn trao đổi nhiệt đặt tại các không gian điều hoà. Lúc này, nước đóng vai trò trao đổi nhiệt, thực hiện quá trình tăng giảm giảm nhiệt và độ ẩm theo bên trong không gian. Kết thúc công đoạn này, nước lại tuần hoàn về máy lạnh trung tâm và tiếp tục một chu trình mới.
		
		$ \bullet $ Hệ thống này phù hợp với những yêu cầu điều hoà cho các không gian khác nhau có chế độ nhiệt độ – độ ẩm khác nhau.( ở mỗi không gian riêng biệt ta có thể lựa chọn một nhiệt độ – độ ẩm tuỳ thích, tuỳ thuộc vào cách khống chế tại không gian đó)
		
		$ \bullet $ Yêu cầu về không gian lắp đặt cho hệ thống này không cao lắm. Khoảng cách giữa trần giả và đáy dầm khoảng từ 100 – 200 $ mm $ là có thể thực hiện được.
	\end{enumerate}

\begin{figure}[H]
\begin{subfigure}{0.5\textwidth}
\begin{center}
		\includegraphics[width=0.9\linewidth]{429_thap_giai_nhiet_liang_chi_lbc_15rt_.jpg} 
	\caption{Tháp giải nhiệt bằng nước}
\end{center}
\end{subfigure}
\begin{subfigure}{0.5\textwidth}
\begin{center}
		\includegraphics[width=0.9\linewidth]{Chiller_giai_nhiet_gio.jpg}
	\caption{Chiller sử dụng giải nhiệt gió}
\end{center}
\end{subfigure}
\caption{Các loại giải nhiệt}
\end{figure}
	
	\textbf{Theo khả năng xử lý nhiệt}
	\begin{enumerate}
		\setlength\itemsep{1mm}
		\item \textbf{Điều hoà một chiều lạnh:} Máy chỉ có khả năng làm lạnh về mùa Hè về mua đông không có khả năng sưởi ấm.
		
		\item \textbf{Điều hoà hai chiều lạnh:} Máy có hệ thống van đảo chiều cho phép hoán đổi chức năng của các dàn nóng và lạnh về các mùa khác nhau. Mùa Hè bên trong nhà là dàn lạnh, bên ngoài là dàn nóng về mùa đông sẽ hoán đổi ngược lại.
	\end{enumerate}
	
\begin{figure}[H]
	\centering
	\includegraphics[width=0.6\linewidth]{vi-vn-daikin-fthf25ravmv-1.jpg}
	\caption{Mô tả máy điều hoà dân dụng 2 chiều}
\end{figure}
	
	\textbf{Theo đặc điểm máy nén}
	\begin{enumerate}
		\setlength\itemsep{1mm}
		\item \textbf{Máy nén PISTON}
		
		$ \bullet $ \textbf{Phân loại theo số lượng piston:} 1 piston, 2 piston, 3 piston, 4 piston v.v...
		
		$ \bullet $ \textbf{Phân loại theo hình dạng:} 1 piston thường là loại kín đặt ngập trong dầu, sơn màu đen, hình dáng hinh trụ tròn và mập lùn hơn so với loại xoắn ốc. Loại 2 piston trở lên thì công suất lớn hơn, nữa kín, thường piston đặt lệch nhau trong một mặt phẳng, ta phân biệt qua số lượng mặt bích hình thoi tròn và dẹp trên thân máy. Thường là gia công đúc, nên máy có hình thoi theo kiểu khối.
		
		$ \ast $ \textbf{Ưu điểm:} tỉ số nén cao, dùng cho máy nén nhiều cấp độ bay hơi sâu. Loại năng suất nhỏ dùng cho hầu hết là: tủ lạnh, máy điều hòa cục bộ v.v...
		
		$ \ast $ \textbf{Khuyết điểm:} hiệu suất thấp, ồn không hiệu quả đối với dãy công suất lớn.
		
\begin{figure}[H]
	\centering
	\includegraphics[width=0.9\linewidth]{bitzer1.jpg}
	\caption{Bên trong máy nén PISTON}
\end{figure}		
		
		\item \textbf{Máy nén Xoắn Ốc} 
		
		Năng suất lạnh \textit{trung bình} và \textit{vừa} (3 $ HP $ đến 60 $ HP $ = 4 máy xoắn ốc ghép song song).
		
		$ \bullet $ \textbf{Phân loại theo kích cỡ máy nén:} 3, 4, 5, 8, 10, 12, 15 $ HP $.
		
		$ \bullet $ \textbf{Phân loại theo hình dáng:} máy theo kiểu kín, đặt ngập trong dầu, máy hình trụ đứng và tròn 2 đầu, sơn đen hình thon và cao gấp đôi máy nén piston một cấp.
		
		$ \bullet $ \textbf{Cấu tạo của máy nén xoắn ốc gồm:} 1 scroll đứng yên và một phần scroll chuyển động theo đĩa lệnh tâm.
		
		– Không có van hút và van đẩy nên tạo được ưu điểm lọai được áp suất rơi gây ra bởi các van nên tăng hiệu suất năng lượng của chu trình.
		
		– Không tồn tại không gian chết – Hiệu suất thể tích tăng gần 100%.
		
		– Rất ít chi tiết chuyển động – Tỉ lệ hư hỏng máy nén giảm tối đa.
		
		– Việc nén gas liên tục trong các túi của scroll – Vận tốc xoay luôn được giữ ở mức thấp.
		
		– Dải công suất rộng lớn dễ dàng cho khách hàng lựa chọn phù hợp với yêu cầu.
		
		– Hệ số hiệu quả làm lạnh COP lớn.
		
		$ \bullet $ \textbf{Đặc tính khởi động tải tối ưu:}
		
		– Máy nén scroll có ưu điểm khởi động giảm tải ngay cả khi áp suất hệ thống không cân bằng.
		
		– Khi máy nén ngừng thì các scroll được tách ra và áp suất lúc này cân bằng.
		
		– Khi máy nén khởi động trở lại, nó không ở điều kiện giảm tải. Vì áp suất sẽ tăng dần cho đến khi vượt quá áp suất đẩy làm van mở và thiết lập lại sự liên tục của hệ thống.

\begin{figure}[H]
	\begin{subfigure}{0.5\textwidth}
		\includegraphics[width=0.9\linewidth]{may-nen-xoan-oc.jpg} 
		\caption{Máy nén xoắn ốc}
	\end{subfigure}
	\begin{subfigure}{0.5\textwidth}
		\includegraphics[width=0.9\linewidth]{may-nen-khi-xoan-oc-ben-trong.jpg}
		\caption{Cấu tạo bên trong của máy nén xoắn ốc}
	\end{subfigure}
\end{figure}
		
		$ \ast $ \textbf{Ưu điểm \& đặc điểm vận hành:} 
		
		– Có khả năng tránh được hiện tượng ngập lỏng và cho phép một lượng nhỏ chất bẩn rắn đi qua mà không làm hư hỏng phần SCROLL.
		
		– Để tránh được hiện tượng ngập lỏng và cặn bẩn được là nhờ vào khả năng tương thích trục và tương thích bán kính trong máy nén SCROLL. Trong nhiều trường hợp không cần bình tách lỏng hoặc bình chứa lỏng lắp trên đường hút. Khi cần thiết thì máy nén scroll chỉ cần sấy cacte.
		
		– Khả năng tương thích theo bán kính. Khi có lỏng hay chất bẩn thí scroll tách ra cho phép chúng đi qua tự do không làm hư hỏng máy nén.
		
		– Khả năng tương thích theo trục. Khi quá tải, scroll cố định tách lên phía trên scroll quay để làm sạch khỏi máy nén bất kì lượng lỏng thừa nào.
		
		– Hạn chế tối đa sự rung động.
		
		\item \textbf{Máy nén Trục Vít}
		
		Hiệu suất cao ở dãy năng suất lạnh lớn (40 $ tons $ đến 900 $ tons $).
		
		$ \bullet $ \textbf{Phân loại máy nén trục vít:} Vít đơn, vít đôi.
		
		Trong ngành lạnh thường chọn theo máy nén vít đôi, các công ty như Daikin (Nhật), vilter (USA) v.v... thì sử dụng vít đơn.

\begin{figure}[H]
	\begin{subfigure}{0.5\textwidth}
		\includegraphics[width=0.9\linewidth]{maynentrucvitcautao.jpg}
		\caption{Cấu tạo bên trong của máy nén trục vít}
	\end{subfigure}
	\begin{subfigure}{0.5\textwidth}
	\includegraphics[width=0.9\linewidth]{maynentrucvit.png} 
	\caption{Máy nén trục vít}
	\end{subfigure}
\end{figure}			
		
		$ \ast $ \textbf{Ưu điểm:} 
		
		– Độ tin cậy cao, tuổi thọ cao
		
		– Kích thước nhỏ gọn
		
		– Không có các chi tiết chuyển động tịnh tiến và quán tính kèm theo
		
		– Hầu như không có hiện tượng va đập thủy lực
		
		– Trong cùng một máy nén có thể thực hiện 2 hay nhiều cấp nén
		
		– Các chỉ tiêu năng lượng và thể tích ổn định trong thời gian vận hành lâu dài…
		
		$ \ast $ \textbf{Khuyết điểm:} 
		
		– Việc chế tạo đòi hỏi phải có trình độ cao
		
		– Dầu bôi trơn cho may nén phải là dâu chuyên dụng
		
		– Để phun dầu vào máy nén cần phải tiêu tốn 1 công nhất định
		
		\item \textbf{Máy nén Ly Tâm} 
		
		- Máy nén ly tâm nhỏ gọn với cấu trúc \textit{đơn giản, ít bộ phận chuyển động, đáng tin cậy, bền, chi phí hoạt động thấp}; dễ dàng để thực hiện nhiều mức độ nén, và hàng loạt các quá trình bay hơi nhiệt độ làm mát trung gian, dễ dàng, \textit{cho phép tiêu thụ điện năng thấp hơn}; dầu bôi trơn tua-bin ly tâm có rất ít quá trình trao đổi nhiệt vì hiệu quả sinh nhiệt giữa các chi tiết máy nhỏ, nên máy có hiệu quả năng suất lạnh cao.
		
		$ \bullet $ \textbf{Phân loại máy nén ly tâm:} Có 2 loại (COP thường 6.0 trở lên)
		
		\textbf{Loại ly tâm thường} dùng một cánh quạt ly tâm, ứng dụng lực ly tâm (giống với máy bơm nước ly tâm), với cánh quạt lớn hơn rất nhiều so với máy nén turbocor ly tâm, ứng dụng với dãy công suất rất lớn từ 500 tons trở lên, thường những tập đoàn lớn sử dụng công nghệ này với tải rất lớn: Trane, carrier, york, climaveneta, dunham bush v.v...
		
		\textbf{Loại ly tâm turbo} thường dùng 2 cánh quạt ly tâm với động cơ một chiều và dể dàng thay đổi tốc độ nhờ điều chỉnh điện áp cấp tùy theo năng suất lạnh tương ứng, hảng danfoss đang sử dụng công nghệ này đó là máy nén Frictionless centrifugal (Danfoss Turbocor), sử dụng năng lượng hiệu quả cao nhờ thiết kế công nghệ từ. Công suất từ 60 đến 300 tons. Loại turborco: công suất nhỏ. Loại ly tâm trực tiếp công suất lớn.

\begin{figure}[H]
	\centering
	\includegraphics[width=0.6\linewidth]{maynenlytam.jpg}
	\caption{Máy nén ly tâm trong công nghiệp}
\end{figure}			
		
		$ \ast $ \textbf{Ưu điểm:} 
		
		– Kích thước và trọng lượng nhỏ , đặc biệt với năng suất lạng rất lớn
		
		– Cấu tạo đơn giản vận hành tin cậy và tuổi thọ kéo dài
		
		– Có thể truyền động trực tiếp từ động cơ quay nhanh
		
		– Cân bằng tốt cho nền móng nhẹ nhàng có thể đặt trực tiếp lên các thiết bị khác
		
		– Dòng tác nhân lạnh ra khỏi máy nén một cách đồng đều , không có dầu bôi trơn trong máy nén tăng hệ số truỵền nhiệt
		
		– Có thể nén tiết lưu nhiều cấp 
		
		$ \ast $ \textbf{Khuyết điểm:} 
		
		– Hiệu suất thấp hơn đối với các máy nhỏ và trung bình
		
		– Cần có bộ tăng tốc khi có sử dụng động cơ điện
		
		Các công ty sử dụng công nghệ turborco như danfoss, sử dụng động cơ một chiều để thay đổi tốc độ và tăng hiệu suất máy nén.
		
	\end{enumerate}
	
	\subsubsection{PHÂN LOẠI THEO CÁCH LẮP ĐẶT}
	\begin{enumerate}
		\item \textbf{Điều hoà không khí dạng cửa sổ (WINDOW TYPE)} 
		
		Máy lạnh dạng cửa sổ thường được lắp đặt trên các tường trông giống như các cửa sổ nên được gọi là máy lạnh dạng cửa sổ. Dàn lạnh và dàn nóng được nằm chung trong 1 khối. Máy lạnh dạng cửa sổ là máy lạnh có công suất nhỏ nằm trong khoảng 7.000 $ BTU/h $ đến 24.000 $ BTU/h $ với các model chủ yếu sau có công suất sau 7.000 $ BTU/h $, 9.000 $ BTU/h $, 12.000 $ BTU/h $ và 18.000 $ BTU/h $. 
		
		Tuỳ theo hãng máy mà số model có thể nhiều hay ít. Hiện nay loại máy này không còn sản xuất nữa và thay vào đó là loại máy treo tường 2 khối.
		
\begin{figure}[H]
	\centering
	\includegraphics[width=0.75\linewidth]{may-dieu-hoa-loai-cua-so.jpg}
	\caption{Điều hoà cửa sổ}
\end{figure}
		
		\item \textbf{Điều hoà không khí loại 2 cục}
		
		Máy lạnh treo tường gồm 2 cụm dàn nóng và dàn lạnh được bố trí tách rời nhau. Giữa dàn nóng và dàn lạnh được nối với nhau bằng các ống đồng dẫn gas và dây điện điều khiển. Máy nén thường đặt ở bên trong cụm dàn nóng, và được điều khiển làm việc từ dàn lạnh của máy thông qua bộ điều khiển Remote. Máy lạnh loại treo tường thường có công suất từ 9.000 $ BTU/h $ đến 48.000 $ BTU/h $, bao gồm chủ yếu các model sau : 9.000 $ BTU/h $, 12.000 $ BTU/h $,18.000 $ BTU/h $, 24.000 $ BTU/h $, 36.000 $ BTU/h $, 48.000 $ BTU/h $. 
		
		Tuỳ theo từng hãng chế tạo máy mà số model mỗi chủng loại có khác nhau. Nhưng theo thị trường hiện nay thì chỉ có loại từ 9000 $ BTU/h $ đến 24.000 $ BTU/h $.
		
\begin{figure}[H]
	\centering
	\includegraphics[width=0.9\linewidth]{phan-loai-he-thong-dieu-hoa-khong-khi-2-.jpg}
	\caption{Điều hoà loại cục nóng \& cục lạnh}
\end{figure}
		
		\item \textbf{Điều hoà không khí loại Multi - Split}
		
		Máy điều hòa loại Multi-Split về thực chất là máy điều hoà gồm \textbf{1 dàn nóng} và từ \textbf{2 - 4 dàn lạnh}. Mỗi cụm dàn lạnh được gọi là một hệ thống. Thường các hệ thống hoạt động độc lập. Mỗi dàn lạnh hoạt động không phụ thuộc vào các dàn lạnh khác. 
		
		Các máy điều hoà ghép có thể có các dàn lạnh chủng loại khác nhau. Máy điều hòa dạng ghép có những đặc điểm và cấu tạo tương tự máy điều hòa kiểu rời.
		
		Tuy nhiên do dàn nóng chung nên tiết kiệm diện tích lắp đặt.

\begin{figure}[H]
	\centering
	\includegraphics[width=0.9\linewidth]{multi_split.png}
	\caption{Điều hoà loại Multi - Split}
\end{figure}		
		
		\item \textbf{Điều hoà không khí dạng tủ đứng}
		
		Máy lạnh tủ đứng là máy có công suất trung bình. Đây là dạng máy rất hay được lắp đặt ở các nhà hàng và các sảnh của các cơ quan. Công suất của máy từ 36.000 $ BTU/h $ đến 100.000 $ BTU/h $. 
		
		Về nguyên lý lắp đặt cũng giống như máy lạnh treo tường 2 khối gồm dàn nóng, dàn lạnh và hệ thống ống đồng, dây điện nối giữa chúng. 
		
		Ưu điểm của máy là gió lạnh được tuần hoàn và thổi trực tiếp vào không gian điều hoà nên tổn thất nhiệt rất ít.
\begin{figure}[H]
	\centering
	\includegraphics[width=0.3\linewidth]{image_10.png}
	\caption{Điều hoà loại tủ đứng}
\end{figure}		
	\end{enumerate}
	
	\section{HỆ THỐNG CẤP THOÁT NƯỚC VÀ CHỮA CHÁY}
	\subsection{GIỚI THIỆU VỀ HỆ THỐNG CẤP THOÁT NƯỚC}
	Nước thì không thể thiếu với cuộc sống mọi người. Việc xây dựng hệ thống cáp thoát nước đòi hỏi tính kỹ thuật cao và luôn đảm bảo mọi người có đủ nước sinh hoạt tại mọi thời điểm.
	
	\subsubsection{\emph{Hệ thống cấp nước:}}
	Hầu hết hệ thống cấp nước của các tòa nhà chung cư sử dụng tích hợp của ba loại hệ thống: hệ thống cấp nước trực tiếp, hệ thống cấp nước gián tiếp và hệ thống bơm nước thải.
		
	+ Đối với hệ thống cấp nước trực tiếp, nước sạch được cấp trực tiếp từ đường ống nước công cộng đến các hộ gia đình ở các tầng thấp bằng áp suất thủy lực bên trong đường ống chính;
	
	+ Đối với hệ thống cấp nước gián tiếp, sử dụng máy bơm nước để lấy nước từ các bể chứa ở tầng trệt của tòa nhà, và hút nước sạch vào bể trên mái nhà, sau đó dẫn nước đến từng hộ gia đình thông qua mạng lưới đường ống phụ;
		
	* Đối với hệ thống bơm nước thải, nước được truyền kết thúc nhận được bằng cách lắp máy bơm áp lực để cấp nước: đường ống cứu hỏa cũng có chức năng tương tự;
	
	Hệ thống cấp nước bao gồm: \emph{máy bơm nước, đường ống đứng, bể chứa, thiết bị phao tự ngắt} và \emph{các đường ống phụ}. Tất cả các phần cố định của hệ thống cấp nước phải được thường xuyên kiểm tra và duy trì hoạt động đúng cách và tất cả các bể nước phải được làm sạch theo định kỳ để kiểm soát chất lượng tốt nhất.
	\begin{figure}[H]
		\centering
		\includegraphics[width=0.8\linewidth]{so-luoc-he-thong-cap-thoat-nuoc-nha-cao-tang_2.jpg}	
		\caption{Sơ đồ hệ thống cấp nước toà nhà}
	\end{figure}
	
	\subsubsection{\emph{Hệ thống thoát nước:}}
	Hệ thống thoát nước có thể được chia thành \emph{hệ thống đường ống thoát nước mưa} và \emph{hệ thống đường ống nước thải}. Các phần cố định của hệ thống thoát nước bao gồm các đường ống nước thải, xi phông, hố ga. Các đường ống nước thải phải nối sao cho phù hợp nhất, chẳng hạn như nước thải từ bồn rửa không được xả ra theo đường ống nước mưa. Ngoài ra, phải đảm bảo đầu thoát nước thải không bị rác chặn hoặc phải có lưới để ngăn  rác khỏi tắc đường ống.
	
	Tất cả các đường ống nước thải bao gồm đường ống chôn dưới đất, ống dẫn chất thải, ống thông gió và ống cống ngầm phải luôn ở trong tình trạng hoạt động tốt. Cần phải kiểm tra định kỳ tất cả các đường ống trên; nếu phát hiện rò rỉ, tắc nghẽn hoặc hư hỏng, cần tiến hành sửa chữa ngay.
	
	Để ngăn chặn khí thải và côn trùng trong đường ống xâm nhập vào khu dân cư, các thiết bị vệ sinh bao gồm bồn rửa tay, chậu rửa, bồn tắm và vòi sen, nhà vệ sinh và nắp thoát nước ở sàn phải được gắn với ống xi phông (ống xi phông hình chữ U, ống xi phông hình chai hoặc loại chống chảy ngược).
	
	Cần kiểm tra các cửa cống  thường xuyên, nếu phát hiện tắc nghẽn thì phải xử lý ngay. Các cửa cống phải được bố trí sao cho việc bảo trì  được thực hiện dễ dàng và thường xuyên. Không nên để các vật cản như đồ đạc hay cây cảnh ở khu vực này. Có thể ngăn chặn khí thải do rò rỉ từ các hố ga bằng cách sử dụng loại nắp cống hai lớp, hoặc sửa chữa ở các cạnh của lỗ cống hoặc các vết nứt ở các miệng cống.
	
	Trách nhiệm sửa chữa và bảo trì hệ thống thoát nước được xác định dựa trên hư hỏng của đường ống công cộng hoặc đường ống của từng căn hộ. Ví dụ, nếu như xảy ra nổ đường ống thoát nước mưa, hoặc tất cả các chủ sở hữu phải chịu trách nhiệm sửa chữa. Tuy nhiên, một nhánh của đường ống được nối đến một căn hộ bị hư hỏng, chủ sở hữu hoặc người cư trú trong đó căn hộ phải có trách nhiệm sửa chữa.
	
	\subsection{TỔNG QUAN VỀ HỆ THỐNG CẤP THOÁT NƯỚC}
	- Toà nhà được thiết kế với 27 tầng. Số lượng nhà vệ sinh được thống kê trong bảng bên dưới sau đây:
	
\begin{table}[H]
		\vspace{-0.5cm}
		\centering
		\textbf{\caption{Số lượng nhà vệ sinh theo từng tầng}}
		\begin{tabular}{|c|c|c|c|c|c|c|}
			\hline
			Tầng  & 1     & 2     & 3     & 4     & 5     & 6-27 \\
			\hline
			Số lượng nhà vệ sinh & 2     & 2     & 2     & 2     & 2     & 44 \\
			\hline
		\end{tabular}
		\label{tab:soluongnvs}
\end{table}
	
	$\Rightarrow$ Như vậy thì theo bảng thống kê, có tổng cộng \textbf{54 nhà vệ sinh}.  
	
	- Trong đó, mỗi nhà vệ sinh có đúng số lượng thiết bị sau:
	
\begin{table}[H]
	\vspace{-0.25cm}
	\centering
	\begin{minipage}[t]{.5\textwidth}
		\centering
		\caption{Nhà vệ sinh nam} 
		\begin{tabular}{|c|c|c|}
    	\hline
    	\multicolumn{1}{|c|}{\textbf{STT}} & \multicolumn{1}{c|}{\textbf{Tên thiết bị}} & \multicolumn{1}{c|}{\textbf{Số lượng}} \\
    	\hline
    	1     & Bồn tiểu nam & 4 \\
    	\hline
    	2     & Chậu rửa mặt & 2 \\
    	\hline
    	3     & Bồn cầu & 4 \\
    	\hline
    	\end{tabular} 	
  		\label{tab:tb_nvs_nam}
	\end{minipage}
	\hspace{-0.5cm}
	\begin{minipage}[t]{.5\textwidth}
		\centering
		\caption{Nhà vệ sinh nữ}
		\begin{tabular}{|c|c|c|}
    	\hline
    	\multicolumn{1}{|c|}{\textbf{STT}} & \multicolumn{1}{c|}{\textbf{Tên thiết bị}} & \multicolumn{1}{c|}{\textbf{Số lượng}} \\
    	\hline
    	1     & Chậu rửa mặt & 3 \\
    	\hline
    	2     & Bồn cầu & 4 \\
    	\hline
    	\end{tabular}
  		\label{tab:tb_nvs_nữ}	
	\end{minipage}		
\end{table}
	
\begin{table}[H]
		\vspace{-0.5cm}
  		\centering
  		\caption{Tổng thiết bị có trong \textbf{54 nhà vệ sinh}}
    	\begin{tabular}{|c|c|c|}
    	\hline
    	\multicolumn{1}{|c|}{\textbf{STT}} & \multicolumn{1}{c|}{\textbf{Tên thiết bị}} & \multicolumn{1}{c|}{\textbf{Số lượng}} \\
    	\hline
    	1     & Bồn tiểu nam & 4 \\
    	\hline
    	2     & Chậu rửa mặt & 5 \\
    	\hline
    	3     & Bồn cầu & 8 \\
   	 	\hline
    	\end{tabular}%
  		\label{tab:total_equipment_toilet}
\end{table}
		
		- Đối với công trình, hệ thống cấp nước sẽ được cung cấp gián tiếp từ nhà máy vì dù đây là building lớn, nằm ngay trung tâm thành phố Hà Nội - Rất ít xảy ra hiện tượng cúp nước. Nhưng không thể sử dụng áp lực từ chính đường ống nước để đẩy nước lên các tầng cao hơn được. Đồng thời, nếu xảy ra tình huống cúp nước thì toà nhà vẫn đảm bảo được một lượng nước để vận hành trong khoảng thời gian chờ khắc phục sự cố.
		
	\subsection{TỔNG QUAN VỀ HỆ THỐNG THOÁT NƯỚC}
	- Đối với hệ thống thoát nước, ngoại trừ việc phải tính toán thoát nước cho các nhà vệ sinh thì còn phải tính tới cả thoát nước mưa, nước ngưng.
	
	- Hệ thống thoát nước phải đảm bảo được các nguyên tắc sau đây:
	\begin{itemize}[leftmargin=2.2cm]
		\setlength\itemsep{1mm}
		\item Hệ thống thoát nước phân và hệ thống thoát nước sàn riêng biệt. 
		\item Hệ thống ống đứng thoát nước mưa được bố trí trong các hộp gen thông tầng.
		\item Hệ thống đảm bảo thoát nước tốt. 
		\item Có tổng chiều dài ngắn nhất. 
		\item Dể dàng kiểm tra sữa chửa thay thế. 
		\item Tránh đi qua phòng khách, phòng ngủ. 
		\item Dễ phân biệt khi sửa chữa. 
		\item Thuận tiện trong quá trình thi công. 
	\end{itemize}
	\subsubsection{Hệ thống thoát nước nhà vệ sinh:}	
	- Hệ thống thoát nước đầu tiên và dễ thấy nhất trong toà nhà đó chính là hệ thống thoát nước cho các nhà vệ sinh, sinh hoạt chung khác.
	
\begin{table}[H]
		\vspace{-0.5cm}
  		\centering
  		\caption{Tổng thiết bị có trong \textbf{54 nhà vệ sinh}}
    	\begin{tabular}{|c|c|c|}
    	\hline
    	\multicolumn{1}{|c|}{\textbf{STT}} & \multicolumn{1}{c|}{\textbf{Tên thiết bị}} & \multicolumn{1}{c|}{\textbf{Số lượng}} \\
    	\hline
    	1     & Bồn tiểu nam & 4 \\
    	\hline
    	2     & Chậu rửa mặt & 5 \\
    	\hline
    	3     & Bồn cầu & 8 \\
   	 	\hline
    	\end{tabular}%
  		\label{tab:tong_thoat_nvs}
\end{table}
	 
	\subsubsection{Vai trò của hệ thống thoát nước mưa:}
	\begin{itemize}
		\vspace{-2mm}
		\setlength\itemsep{1mm}
		\item Hệ thống thoát nước mưa trên mái sẽ làm cho nước mưa không tồn đọng lại trên mái và không bị thấm ngược vào trong nhà.
		\item Nếu thoát nước không tốt thì nước mưa sẽ thấm vào mái sẽ gây ra ẩm mốc làm ảnh hưởng kết cấu công trình cũng như thẩm mỹ công trình.
		\item Trong trường hợp không có hệ thống thoát nước mưa thì rác sẽ bị ùn ứ, đọng lâu ngày sinh ra mất vệ sinh làm cho các sinh vật sinh sôi như muỗi, ruồi, các loại vi khuẩn, nấm mốc gây bệnh.
	\end{itemize}
	
	$\ast$ Tuỳ từng loại mái của toà nhà mà sẽ có các hệ thống thoát nước mưa khác nhau, nhưng 2 loại mái thường gặp nhất là \emph{mái dốc} và \emph{mái bằng}.
	
	\subsubsection{Tại sao phải có hệ thống thoát nước ngưng:}
	- Trong quá trình làm lạnh khí (ở dàn lạnh của điều hòa), hơi nước ngưng tụ và hoá lỏng, vì vậy phải có đường thoát nước từ dàn lạnh ra. Nhiều người không chú ý, thậm chí không biết đến vấn đề này nên khi lắp đặt điều hoà, không biết thoát nước đi đâu – nhất là khi dàn lạnh ở phía tường trong, không tiếp cận với hệ thống thoát nước nào hoặc đi ra ngoài mặt thoáng rất xa. Vì vậy các gia đình nên chú ý khi lắp điều hòa phải lắp đặt thêm ống thoát nước điều hòa.

- Ống thoát nước điều hòa là một phần quan trọng nằm trong các loại vật tư khi lắp điều hòa, khi điều hòa hoạt động đặc biệt là mùa đông chạy chiều nóng, lượng nước thải trong một ngày đêm trung bình từ 6-12 lít tùy thuộc từng dòng và công suất điều hòa. Nước chảy qua ống thoát có độ lạnh khoảng 10-15 độ C do đó cần có các bộ phận dẫn hướng nước tránh rò rỉ, thoát ra sàn nhà, trần nhà.
	
	\vspace{0.5cm}$\pmb{\Rightarrow}$ Vì vậy hệ thống thoát nước toà nhà cũng là một yếu tố vô cùng quan trọng trong tính toán cấp thoát nước.
	
	\subsubsection{Tiêu chuẩn và quy chuẩn sẽ áp dụng cho hệ thống:}
	\hspace{1cm}- Hệ thống sẽ được tính toán dựa theo những \emph{tiêu chuẩn} và \emph{quy chuẩn} sau:
	\begin{itemize}[leftmargin=2cm]
		\item[\textbf{1.}]Quy chuẩn QCVN 07-2:2016/BXD
		\item[\textbf{2.}]Tiêu chuẩn TCVN 4519 : 1988
		\item[\textbf{3.}]Tiêu chuẩn TCVN 5576 : 1991
	\end{itemize}
	\subsection{TỔNG QUAN VỀ HỆ THỐNG CHỮA CHÁY}	
	- Hệ thống chữa cháy là tổng hợp các thiết bị kỹ thuật chuyên dùng, đường ống dẫn và các chất chữa cháy dùng để dập tắt đám cháy.
	
	- Hệ thống chữa cháy vách tường là hệ thống chữa cháy được lắp đặt ở trên tường bên trong các công trình.
	
	- Thiết bị chủ yếu trong hệ thống chữa cháy vách tường gồm: \emph{máy bơm nước chữa cháy}, \emph{đường ống cấp nước chữa cháy} và các phương tiện khác như \emph{van}, \emph{lăng phun nước}, \emph{cuộn vòi dẫn nước}…
	
	\subsubsection{Hệ thống chữa cháy tự động Sprinkler đường ống ướt:}
	- Toà nhà sẽ sử dụng hệ thống chữa cháy tự động Sprinkler đường ống ướt. Hệ thống đường ống ướt là hệ thống sprinkler tiêu chuẩn thường xuyên nạp đầy nước có áp lực ở cả phía trên và phía dưới van báo động đường ống ướt.
	\subsubsection{Tiêu chuẩn và quy chuẩn sẽ áp dụng cho hệ thống:}
	- Hệ thống sẽ được tính toán dựa theo những \emph{tiêu chuẩn} và \emph{quy chuẩn} sau:
	\begin{itemize}[leftmargin=2cm]
		\item[\textbf{1.}]Tiêu chuẩn TCVN 6305 
		\item[\textbf{2.}]Tiêu chuẩn TCVN 7735 – 2003
		\item[\textbf{3.}]Tiêu chuẩn TCVN 5040 – 1990
		\item[\textbf{4.}]Tiêu chuẩn TCVN 5760 – 1993
	\end{itemize}
	
	\section{HỆ THỐNG ĐIỆN TRONG CÔNG TRÌNH}
	\subsection{GIỚI THIỆU VỀ HỆ THỐNG CHIẾU SÁNG}	
	- Thiết kế hệ thống chiếu sáng trong toà nhà đòi hỏi sự hiểu biết về kỹ thuật điện, nguồn sáng và tầm nhìn, đồng thời cũng nhạy cảm với các vấn đề về kiến trúc và thẩm mỹ. Thiết kế cuối cùng cần đáp ứng nhu cầu trực quan cho mắt người thực hiện vô số công việc trong khi vẫn đáp ứng đc các hình dạng kiến trúc và môi trường ngay lập tức.
	
	- Hệ thống chiếu sáng do đó cũng có thể xem là một trong những hệ thống thuộc diện ``sống còn'' của toà nhà mà đòi hỏi rất gắt gao ở việc thiết kế cũng như tính thẩm mỹ.
	
	- Các nhà thiết kế về ánh sáng hiểu rằng hầu hết những người ở trong tòa nhà không nhất thiết muốn có đèn LED hoặc một chủng loại đèn bất kỳ nào - họ muốn thoải mái nhìn thấy những gì họ đang làm. Làm thế nào để cung cấp cho họ tầm nhìn này là vai trò của nhà thiết kế ánh sáng. Làm thế nào để cung cấp cho điều này trong khi vẫn thiết kế hài hòa với kiến trúc, tích hợp với ánh sáng ban ngày sẵn có, giảm thiểu việc sử dụng năng lượng xây dựng, phù hợp với quá trình xây dựng tổng thể và ngân sách là tất cả những việc phải làm của nhà thiết kế ánh sáng trong toàn bộ quá trình thiết kế tòa nhà.
	
	- Một hệ thống chiếu sáng đủ tiêu chuẩn phải đáp ứng đc các tiêu chí sau:
	\begin{itemize}[leftmargin=2.5cm]
		\item Độ rọi chiếu sáng.
		\item Thiết kế không gian.
		\item Nhiệt độ màu.
		\item Điều kiện tiện nghi.
		\item Hệ thống điều khiển hợp lý có thể sử dụng điều khiển từ xa để tránh gây lãng phí điện năng và dể quản lý khâu bật tắt.
		\item Tính thẩm mỹ cao tôn lên vẻ đẹp không gian, sang trọng, hiện đại, phù hợp thời đại công nghệ mới.
	\end{itemize}
	\subsubsection{Tiêu chuẩn và quy chuẩn sẽ áp dụng cho hệ thống:}		
	- Hệ thống sẽ được tính toán dựa theo những \emph{tiêu chuẩn} và \emph{quy chuẩn} sau:
	\begin{itemize}[leftmargin=2cm]
		\item[\textbf{1.}]Tiêu chuẩn TCVN 7114 : 2008 
		\item[\textbf{2.}]Quy chuẩn QCXDVN 09 : 2005
	\end{itemize}	
	
	\section{HỆ THỐNG TỰ ĐỘNG HOÁ TRONG CÔNG TRÌNH}
	- Sự phát triển bền vững của kinh tế, chính trị ở mỗi quốc gia trên thế giới làm cho nhu cầu đòi hỏi về vật chất, sự sang trọng tiện nghi và đảm bảo an ninh,an toàn trong cả nơi làm việc cũng như nhà ở ngày càng có nhu cầu cao hơn.

	- Sự ra đời của các toà nhà, khách sạn, các trung tâm thương mại, các cao ốc văn phòng… với mức độ tự động hóa và bảo mật cao ngày càng nhiều hơn. Nhu cầu về nhân lực cũng như thiết bị vật tư, các giải pháp thiết kế và thi công cao. Đó là lĩnh vực có thể nghiên cứu đầu tư kinh doanh khả thi trong tương lai không xa.
	
	- Trong toà nhà thông minh, đồ dùng trong nhà từ các phòng chức năng, các căn phòng làm việc, phòng ngủ, phòng khách đến toilet đều gắn các bộ điều khiển điện tử có thể kết nối với mạng Internet và điện thoại di động, cho phép chủ nhân có thể điều khiển tại chỗ, điều khiển vật dụng từ xa hoặc lập trình cho thiết bị ở nhà hoạt động tự động theo lịch với chương trình có sẵn.
	
	{\large $\pmb{\Rightarrow}$} Như vậy, toà nhà thông minh là một toà nhà có một hệ thống kỹ thuật hoàn hảo, được lập trình tối ưu hóa cho việc điều khiển, giám sát, vận hành thiết bị,vật dụng trong toà nhà.	
	\subsection{GIỚI THIỆU VỀ HỆ THỐNG BMS CỦA TOÀ NHÀ}
	- Hệ thống BMS (Building Management System) là hệ thống đồng bộ cho phép điều khiển và quản lý mọi hệ thống kỹ thuật trong tòa nhà như \emph{hệ thống điện, hệ thống cung cấp nước sinh hoạt, điều hòa thông gió, cảnh báo môi trường, an ninh, báo cháy – chữa cháy,}… đảm bảo cho việc vận hành các thiết bị trong tòa nhà được chính xác, kịp thời, hiệu quả, tiết kiệm năng lượng và tiết kiệm chi phí vận hành. Hệ thống BMS là hệ thống đồng bộ mang tính thời gian thực, trực tuyến, đa phương tiện, nhiều người dung, hệ thống vi xử lý bao gồm các bộ vi xử lý trung tâm với tất cả các phần mềm và phần cứng máy tính, các thiết bị vào/ra, các bộ vi xử lý khu vực, các bộ cảm biến và điều khiển qua các ma trận điểm.
	
	\textbf{Hệ thống quản lý tòa nhà (hệ thống BMS) điều khiển và giám sát các hệ thống sau:}
\begin{multicols}{2}
	\begin{itemize}
		\item Trạm phân phối điện
		\item Máy phát điện dự phòng
		\item Hệ thống chiếu sáng
		\item Hệ thống điều hoà \& thông gió
		\item Hệ thống cấp nước sinh hoạt
		\item Hệ thống báo cháy
		\item Hệ thống chữa cháy
		\item Hệ thống thang máy
		\item Hệ thống âm thanh công cộng
		\item Hệ thống thẻ kiểm soát ra vào
		\item Hệ thống an ninh
		\item[\vspace{\fill}] %thêm vào để cho đủ item/2 đặng các cột có khoảng cách = nhau
	\end{itemize}
\end{multicols}
	
	\begin{figure}[H]
		\centering
		\includegraphics[scale=0.8]{hao_phuong_so_do_he_thong_bms.jpg}	
		\caption{Sơ đồ hệ thống BMS}
	\end{figure}

	\textbf{Tính năng của BMS}
\begin{itemize}[leftmargin=2cm]
	\item Cho phép các tiện ích (thiết bị thông minh) trong tòa nhà hoạt động một cách đồng bộ, chính xác theo đúng yêu cầu của người điều hành
	
	\item Cho phép điều khiển các ứng dụng trong tòa nhà thông qua cáp điều khiển và giao thức mạng
	
	\item Kết nối các hệ thống kỹ thuật như an ninh, báo cháy… qua cổng giao diện mở của hệ thống với các ngôn ngữ giao diện theo tiêu chuẩn quốc tế
	
	\item Giám sát được môi trường không khí, môi trường làm việc của con người
	
	\item Tổng hợp, báo cáo thông tin
	
	\item Cảnh báo sự cố, đưa ra những tín hiệu cảnh báo kịp thời trước khi có những sự cố
	
	\item Quản lý dữ liệu gồm soạn thảo chương trình, quản lý cơ sở dữ liệu, chương trình soạn thảo đồ hoạ, lưu trữ và sao lưu dữ liệu
	
	\item Hệ thống BMS linh hoạt, có khả năng mở rộng với các giải pháp sẵn sàng đáp ứng với mọi yêu cầu
\end{itemize}

\begin{figure}[H]
	\centering
	\includegraphics[width=0.95\linewidth]{he_thong_bms_hao_phuong.png}	
	\caption{BMS tính năng}
\end{figure}
	
	\textbf{Lợi ích mang lại từ BMS}
	\begin{itemize}[leftmargin=2cm]
		\item Đơn giản hóa và tự động hóa vận hành các thủ tục, chức năng có tính lặp đi lặp lại
		\item Quản lý tốt hơn các thiết bị trong tòa nhà nhờ hệ thống lưu trữ dữ liệu, chương trình bảo trì bảo dưỡng và hệ thống tự động báo cáo cảnh báo
		\item Giảm sự cố và phản ứng nhanh đối với các yêu cầu của khách hàng hay khi xảy ra sự cố
		\item Giảm chi phí năng lượng nhờ tính năng quản lý tập trung điều khiển và quản lý năng lượng
		\item Giảm chi phí nhân công và thời gian đào tạo nhân viên vận hành – cách sử dụng dễ hiểu, mô hình quản lý được thể hiện trực quan trên máy tính cho phép giảm tối đa chi phí dành cho nhân sự và đào tạo
		\item Dễ dàng nâng cấp, linh hoạt trong việc lập trình theo nhu cầu, kích thước, tổ chức và các yêu cầu mở rộng khác nhau
	\end{itemize}
	
	\textbf{Cấu trúc của hệ thống BMS gồm 4 phần:}
	\begin{enumerate}
		\setlength{\itemindent}{2cm}
		\item Phần mềm điều khiển trung tâm
		\item Thiết bị cấp quản lý
		\item Bộ điều khiển cấp trường
		\item Cảm biến và các thiết bị chấp hành
	\end{enumerate}

	\subsection{IoT VÀ TÁC ĐỘNG CỦA NÓ LÊN BMS}
	\subsubsection{IoT là gì?}
	- Internet of Things, hay IoT, internet vạn vật là đề cập đến hàng tỷ thiết bị vật lý trên khắp thế giới hiện được kết nối với internet, thu thập và chia sẻ dữ liệu. Nhờ bộ xử lý giá rẻ và mạng không dây, có thể biến mọi thứ, từ viên thuốc sang máy bay, thành một phần của IoT. Điều này bổ sung sự “thông minh kỹ thuật số” cho các thiết bị, cho phép chúng giao tiếp mà không cần có con người tham gia và hợp nhất thế giới kỹ thuật số và vật lý.

	- Một bóng đèn có thể được bật bằng ứng dụng điện thoại thông minh là một thiết bị IoT, như một cảm biến chuyển động hoặc một bộ điều chỉnh nhiệt thông minh trong văn phòng của bạn hoặc đèn đường được kết nối. Một thiết bị IoT có thể đơn giản như đồ chơi của trẻ em hoặc nghiêm trọng như một chiếc xe tải không người lái, hoặc phức tạp như một động cơ phản lực hiện chứa hàng ngàn cảm biến thu thập và truyền dữ liệu trở lại để đảm bảo nó hoạt động hiệu quả. Ở quy mô lớn hơn, các dự án thành phố thông minh đang được lấp đầy bằng các cảm biến để giúp chúng ta hiểu và kiểm soát môi trường.

	- Thuật ngữ IoT chủ yếu được sử dụng cho các thiết bị thường không được mong đợi có kết nối internet và có thể giao tiếp với mạng độc lập với hành động của con người. Vì lý do này, PC thường không được coi là thiết bị IoT và cũng không phải là điện thoại thông minh – mặc dù thiết bị này được nhồi nhét bằng cảm biến. Tuy nhiên, một chiếc smartwatch hoặc một fitness band hoặc thiết bị đeo khác có thể được tính là một thiết bị IoT.
	
	\subsubsection{IoT thay đổi cách mà BMS vận hành thế nào?}
	- Kết nối chính là điểm mạnh của IoT. Với IoT, các nhà sản xuất khác nhau đều có thể có mặt trong hệ thống BMS được nhờ vào các tiêu chuẩn giao thức như \emph{Ethernet/IP, XML, KNX, BACnet, Modbus và LonWorks}. 

	- Khả năng kết nối được đó sẽ giúp hệ thống tiến đến thứ gọi là ``Đám mây''. Với các ``đám mây'' này thì việc điều khiển từ xa và khả năng tiên đoán lỗi hệ thống và giám sát có thể thực hiện từ khắp mọi nơi hoặc hơn nữa là xử lý và hiệu chỉnh tạm thời bằng các thuật toán trước khi nhân viên kỹ thuật xuống sửa chữa. Điều này làm tăng khả năng vận hành lâu dài của hệ thống.
	
	- IoT cũng sẽ thay đổi mức giá của hệ thống BMS bởi các thành phần linh kiện của hệ thống giờ đây sẽ không còn độc quyền từ một hãng nữa mà sẽ là mức giá cạnh tranh tới từ nhiều hãng.
	\subsection{LÀM THẾ NÀO ĐỂ TIẾP CẬN IoT ĐÚNG CÁCH?}
	- Internet Of Things viết tắt là IoT chính là internet trong mọi thứ. Và theo WikiPedia định nghĩa thì IoT chính là mạng lưới vạn vật kết nối Internet hoặc mạng lưới kết nối thiết bị Internet . Là một kịch bản của thế giới, khi mà mỗi đồ vật, con người được cung cấp một định danh riêng của nó và tất cả có khả năng truyền tải, trao đổi thông tin, dữ liệu qua một mạng duy nhất mà không cần đến sự tương tác trực tiếp giữa người với người, hay người với máy tính.
	
	- Khi mà vạn vật đều có chung một mạng kết nối thì việc liên lạc và làm việc trở nên rất dễ dàng. Con người có thể hiện thực hóa mục đích của mình trong tương lai. Chúng ta hoàn toàn có thể kiểm soát mọi thứ. Giả sử 1 chiếc ví mà các bạn đang sử dụng có tích hợp công nghệ IoT. Chúng có nhiệm vụ kiểm tra số lượng tiền trong ví, kiểm tra ngày hết hạn của các giấy tờ mà các bạn để trong đó như: bảo hiểm y tế, hạn nộp học phí,.. và thông báo tình trạng của nó đến cho chúng ta biết thông qua các ứng dụng tin nhắn SMS, facebook, skype, zalo,…
	
	- Hay như một hệ thống tưới nước tự động cây cối trong gia đình bạn được tích hợp công nghệ IoT. Giúp bạn điều khiển qui trình chăm sóc cây, tưới nước cây, thậm chí là bắt sâu bọ,… khi bạn có chuyến đi công tác xa vài ngày hay vài tháng mà không thể thực hiện được các chức năng đó. Điều đó sẽ trở nên rất đơn giải khi giả sử mà hệ thống tưới cây tự động và điện thoại hoặc laptop, PC,.. của bạn được kết nối và mạng lưới Internet và qua đó có thể trao đổi thông tin cũng như thực thi các câu lệnh mà bạn mong muốn.
	
	- Điều đó thật mới mẻ và tiện dụng phải không nào? Chúng ta có thể tiết kiệm được rất nhiều thời gian cũng như tránh gặp phải những trường hợp khó khăn khi không làm chủ và quản lý được tất cả mọi vật xung quanh ta.
	\subsubsection{Hệ thống nhúng - Bước đầu tiên IoT}
	- IoT là một hệ thống tự trị, để làm được điều này, nền tảng kiến thức của IoT bắt buộc phải xuất phát từ hệ thống nhúng. 
	
	- Hệ thống nhúng thường có một số đặc điểm chung như sau:
\begin{itemize}[leftmargin=2.2cm]
		\item Các hệ thống nhúng được thiết kế để thực hiện một số nhiệm vụ chuyên dụng chứ không phải đóng vai trò là các hệ thống máy tính đa chức năng. Một số hệ thống đòi hỏi ràng buộc về tính hoạt động thời gian thực để đảm bảo độ an toàn và tính ứng dụng; một số hệ thống không đòi hỏi hoặc ràng buộc chặt chẽ, cho phép đơn giản hóa hệ thống phần cứng để giảm thiểu chi phí sản xuất.
		
		\item Một hệ thống nhúng thường không phải là một khối riêng biệt mà là một hệ thống phức tạp nằm trong thiết bị mà nó điều khiển.
		
		\item Phần mềm được viết cho các hệ thống nhúng được gọi là firmware và được lưu trữ trong các chip bộ nhớ ROM hoặc bộ nhớ flash chứ không phải là trong một ổ đĩa. Phần mềm thường chạy với số tài nguyên phần cứng hạn chế: không có bàn phím, màn hình hoặc có nhưng với kích thước nhỏ, dung lượng bộ nhớ thấp.
\end{itemize}

\begin{figure}[H]
	\centering
	\includegraphics[width=0.6\linewidth]{embedded_system.jpg}
	\caption{Một board mạch nhúng}
\end{figure}

	- Các hệ thống nhúng thường nằm trong các cỗ máy được kỳ vọng là sẽ chạy hàng năm trời liên tục mà không bị lỗi hoặc có thể khôi phục hệ thống khi gặp lỗi. Vì thế, các phần mềm hệ thống nhúng được phát triển và kiểm thử một cách cẩn thận hơn là phần mềm cho máy tính cá nhân. Ngoài ra, các thiết bị rời không đáng tin cậy như ổ đĩa, công tắc hoặc nút bấm thường bị hạn chế sử dụng. Việc khôi phục hệ thống khi gặp lỗi có thể được thực hiện bằng cách sử dụng các kỹ thuật như watchdog timer – nếu phần mềm không đều đặn nhận được các tín hiệu watchdog định kì thì hệ thống sẽ bị khởi động lại.
	
	- Một số vấn đề cụ thể về độ tin cậy \& hướng giải quyết cho hệ thống nhúng như:
\begin{itemize}[leftmargin=2.2cm]
		\item Hệ thống không thể ngừng để sửa chữa một cách an toàn, ví dụ như ở các hệ thống không gian, hệ thống dây cáp dưới đáy biển, các đèn hiệu dẫn đường,... Giải pháp đưa ra là chuyển sang sử dụng các hệ thống con dự trữ hoặc các phần mềm cung cấp một phần chức năng.
		
		\item Hệ thống phải được chạy liên tục vì tính an toàn, ví dụ như các thiết bị dẫn đường máy bay, thiết bị kiểm soát độ an toàn trong các nhà máy hóa chất,... Giải pháp đưa ra là lựa chọn backup hệ thống.
		
		\item Nếu hệ thống ngừng hoạt động sẽ gây tổn thất rất nhiều tiền của ví dụ như các dịch vụ buôn bán tự động, hệ thống chuyển tiền, hệ thống kiểm soát trong các nhà máy...
\end{itemize}

	- Hệ thống nhúng tương tác với thế giới bên ngoài với nhiều cách:
\begin{itemize}[leftmargin=2.2cm]
	\item Cảm nhận môi trường: cảm biến nhiệt độ, độ ẩm, ánh sáng, trọng lượng…, cảm nhận bằng tín hiệu điện (máy dò nhiễu điện từ)
	
	\item Tác động trở lại môi trường (hú còi báo động khi phát hiện khói trong tòa nhà…)
	
	\item Tốc độ tương tác phải đáp ứng thời gian thực (hệ thống còi báo hỏa, hệ thống chống trộm trên ô tô,…)
	
	\item Có thể có hoặc không có giao diện giao tiếp với người dùng như máy tính cá nhân. Với những hệ thống đơn giản, thiết bị nhúng sử dụng LCD nhỏ, Joystick, LED, nút bấm, chỉ thị chữ hoặc số và thường đi kèm với một menu đơn giản. Hiện nay chúng ta cũng có thể kết nối đến hệ thống nhúng thông qua giao diện Web, việc này cho phép giảm thiểu chi phí cho màn hình nhưng vẫn cung cấp khả năng hiển thị và nhập liệu thuận tiện thuận tiện thông qua mạng và máy tính khác.
\end{itemize}	
	
	\subsubsection{Hệ thống nhúng \& ngôn ngữ C/C++}
	- Thiết kế hệ thống nhúng đòi hỏi 1 quá trình phức tạp rất nhiều khâu cũng như quá trình. Nếu phát triển một cách đầy đủ thì sẽ tốn rất nhiều thời gian, công sức cũng như đòi hỏi kiến thức chuyên môn sâu. Nên ở giới hạn trong tập luận văn này, chúng ta sẽ chỉ sử dụng những board mạch có sẵn trên thị trường \textit{(như STM32 hoặc Arduino hoặc ESP8266,...)} để sử dụng \& tạo model thử nghiệm, kiểm chứng.
	
	- Ngôn ngữ mà chúng ta sẽ sử dụng chính là C/C++ vì đây là loại ngôn ngữ thường được sử dụng trong các hệ thống nhúng do tính chất vừa là ngôn ngữ lập trình bậc cao vừa có khả năng tiếp cận các thanh ghi. 
	

\begin{figure}[H]
\begin{subfigure}{0.5\textwidth}
\begin{center}
	\includegraphics[width=0.7\linewidth]{esp8266.jpg} 
	\caption{Board ESP8266}
\end{center}
\end{subfigure}
\begin{subfigure}{0.5\textwidth}
\begin{center}
		\includegraphics[width=0.7\linewidth]{stm32f407_20discovery.jpg}
	\caption{Board STM32F407}
\end{center}
\end{subfigure}
\caption{Các loại kit có mặt trên thị trường}	
\end{figure}

	
	\newpage
\cfoot{CHƯƠNG 3}
\lhead{TÍNH TOÁN PHỤ TẢI LẠNH CHO ĐHKK}

%Trang bìa
\newpage
\chapter{\textbf{TÍNH TOÁN PHỤ TẢI LẠNH CHO ĐHKK}}
\newpage

%Phần nội dung
\section{XÂY DỰNG MÔ HÌNH 3D BẰNG \emph{REVIT}}
\hspace{1cm}$ \clubsuit $ Nhân làm phần này. Sẽ add vào sau. $ \clubsuit $

+++++++++++++++++++++++++++++++++++++++++

\section{PHƯƠNG PHÁP TRUYỀN THỐNG}
\subsection{ĐẠI CƯƠNG}
\hspace{1cm}Các bước chủ yếu để tính toán cân bằng nhiệt ẩm truyền thống gồm \textbf{7 bước} để tính toán như sau:
\subsubsection{Xác định các nguồn nhiệt toả ra}
\hspace{1cm}- Các nguồn nhiệt này có thể xuất phát từ nhiều nguồn khác nhau, điển hình như: Do người, do máy móc, chiếu sáng, rò rỉ không khí, bức xạ mặt trời, thẩm thấu qua kết cấu,...

- Phương trình cân bằng nhiệt tổng quát có dạng:

\begin{center}
	\textit{Q{\footnotesize t}} = \textit{Q{\footnotesize toả} + Q{\footnotesize tt}}
\end{center}

\begin{itemize}[leftmargin = 3cm, label = $\star$]
	\item \textit{Q{\footnotesize t}} - nhiệt thừa trong phòng, \textit{W};
	
	\item \textit{Q{\footnotesize toả}} - nhiệt toả ra trong phòng, \textit{W};
	
	\item \textit{Q{\footnotesize tt}} - nhiệt thẩm thấu từ ngoài vào qua kết cấu bao che do chênh lệch nhiệt độ, \textit{W};
\end{itemize}


- Trong đó, nhiệt lượng \textbf{Q{\footnotesize toả}} có thể được phân thành \textit{8 phần nhiệt} như sau:

\begin{center}
	\textit{Q{\footnotesize toả}} = \textit{Q$ _{1} $ + Q$ _{2} $ + Q$ _{3} $ + Q$ _{4} $ + Q$ _{5} $ + Q$ _{6} $ + Q$ _{7} $ + Q$ _{8} $}
\end{center}

\begin{itemize}[leftmargin = 3cm, label = $\ast$]
	\item \textit{Q$ _{1} $} - Nhiệt toả từ máy móc;
	
	\item \textit{Q$ _{2} $} - Nhiệt toả từ đèn chiếu;
	
	\item \textit{Q$ _{3} $} - Nhiệt toả từ người;
	
	\item \textit{Q$ _{4} $} - Nhiệt toả từ bán thành phẩm;
	
	\item \textit{Q$ _{5} $} - Nhiệt toả từ bề mặt thiết bị trao đổi nhiệt;
	
	\item \textit{Q$ _{6} $} - Nhiệt toả do bức xạ mặt trời qua cửa kính;
	
	\item \textit{Q$ _{7} $} - Nhiệt toả do bức xạ mặt trời qua bao che;
	
	\item \textit{Q$ _{8} $} - Nhiệt toả do lò rọt không khí qua cửa;
\end{itemize}

- Lượng nhiệt từ \textbf{Q{\footnotesize tt}} có thể phân được phân thành 4 lượng nhiệt sau:

\begin{center}
	\textit{Q{\footnotesize tt}} = \textit{Q$ _{9} $ + Q$ _{10} $ + Q$ _{11} $ + Q{\footnotesize bs}}, \textit{W}
\end{center}

\begin{itemize}[leftmargin = 3cm, label = $\ast$]
	\item \textit{Q$ _{9} $} - Nhiệt thẩm thấu qua vách;
	
	\item \textit{Q$ _{10} $} - Nhiệt thẩm thấu qua trần (mái);
	
	\item \textit{Q$ _{11} $} - Nhiệt thẩm thấu qua nền;
	
	\item \textit{Q{\footnotesize bs}} - Nhiệt tổn thất bổ sung do gió và hướng vách;
\end{itemize}


\subsubsection{Xác định nguồn ẩm thừa trong phòng điều hoà W{\footnotesize t}:}

\begin{center}
	\textit{W{\footnotesize t}} = \textit{W$ _{1} $ + W$ _{2} $ + W$ _{3} $ + W$ _{4} $}, kg/s
\end{center}

\begin{itemize}[leftmargin = 3cm, label = $\ast$]
	\item \textit{W$ _{9} $} - Lượng ẩm do người toả vào phòng, \textit{kg/s};
	
	\item \textit{W$ _{10} $} - Lượng ẩm bay hơi từ bán thành phẩm, \textit{kg/s};

	\item \textit{W$ _{11} $} - Lượng ẩm do bay hơi từ sàn ẩm, \textit{kg/s};
	
	\item \textit{W{\footnotesize bs}} - Lượng ẩm do hơi nước nóng toả vào phòng, \textit{kg/s};	
\end{itemize}
	
\subsubsection{Xác định tia quá trình {\Large $\varepsilon$} (còn gọi là hệ số góc tia quá trình)}
\begin{center}
	{\Large $\varepsilon_{t}$} = $ \dfrac{Q_{t}}{W_{t}} $, \textit{kJ/kg}
\end{center}

\subsubsection{Xác định sơ đồ điều hoà không khí}
\hspace{1cm}- Trong bước này, chúng ta cần phải xác định được sơ đồ điều hoà không khí với các thông số trạng thái không khí trong nhà T, ngoài nhà N, hoà trộn H và thổi vào V ví dụ như entanpi I$_{T}$, I$_{N}$, I$_{H}$, I$_{V}$, nhiệt độ t$_{T}$, t$_{N}$, t$_{H}$, t$_{V}$, lưu lượng không khí G$_{T}$, G$_{N}$, G$_{H}$, G$_{V}$ (\textit{kg/s}), L$_{T}$, L$_{N}$, L$_{H}$, L$_{V}$ (\textit{$m^3/s$}), khối lượng riêng không khí $\rho_{T}$, $\rho_{N}$, $\rho_{H}$, $\rho_{V}$, ẩm dung của không khí d$_{T}$, d$_{N}$, d$_{H}$, d$_{V}$...

\subsubsection{Xác định năng suất gió của hệ thống}
\hspace{1cm}- Để có thể tải được hết nhiệt thừa ra khỏi phòng điều hoà cần một lượng gió G$_{t}$ là:

\begin{center}
	$ G_{t} = \dfrac{Q_{t}}{I_{T} - I_{V}} $, \textit{kg/s}
\end{center}

- Để có thể tải được hết ẩm thừa ra khỏi phòng điều hoà cần một lượng gió G$_{W}$ là:

\begin{center}
	$ G_{W} = \dfrac{W_{t}}{d_{T} - d_{V}} $, \textit{kg/s}
\end{center}

Năng suất gió của hệ thống G phải bằng G$_{t}$ và G$_{M}$ do đó:

\begin{center}
	G = G$_{t}$ = G$_{M}$
\end{center}

hay:

\begin{center}
	$ \dfrac{Q_{t}}{I_{T} - I_{V}} $ = $\dfrac{W_{t}}{d_{T} - d_{V}} $
\end{center}

rút ra:

\begin{center}
	$ \dfrac{Q_{t}}{W_{t}} $ = $ \dfrac{I_{T} - I_{V}}{d_{T} - d_{V}} $ = {\Large $\varepsilon_{t}$}
\end{center}

{\Large $\varepsilon_{t}$} chính là hệ số góc tia quá trình.

\subsubsection{Tính năng suất lạnh}
\hspace{1cm}- Năng suất lạnh của hệ thống điều hoà không khí Q$ _{0} $ có thể được tính như sau:

\begin{center}
	$Q _{0} = G_{V}(I_{H} - I_{V})$, \textit{kW}
\end{center}
\begin{center}
	$ Q _{0} = Q_{t}\times\dfrac{I_{H} - I_{V}}{I_{T} - I_{V}}$, \textit{kW}
\end{center}

\subsubsection{Lượng ẩm ngưng tụ trên dàn bay hơi W}
\begin{center}
	$W = G_{V}(d_{H}-d_{V})$, \textit{kg/s}
\end{center}

\subsection{TÍNH TOÁN NHIỆT ẨM CHO TOÀ NHÀ}




\section{PHƯƠNG PHÁP CARRIER}
\hspace{1cm}- 

\section{TÍNH TẢI LẠNH BẰNG \emph{REVIT MEP}}
\hspace{1cm}- 

\end{document}

=======
	\include{LVTN_chap1&2}
	%\newpage
\cfoot{CHƯƠNG 3}
\lhead{TÍNH TOÁN PHỤ TẢI LẠNH CHO ĐHKK}

%Trang bìa
\newpage
\chapter{\textbf{TÍNH TOÁN PHỤ TẢI LẠNH CHO ĐHKK}}
\newpage

%Phần nội dung
\section{XÂY DỰNG MÔ HÌNH 3D BẰNG \emph{REVIT}}
\hspace{1cm}$ \clubsuit $ Nhân làm phần này. Sẽ add vào sau. $ \clubsuit $

+++++++++++++++++++++++++++++++++++++++++

\section{PHƯƠNG PHÁP TRUYỀN THỐNG}
\subsection{ĐẠI CƯƠNG}
\hspace{1cm}Các bước chủ yếu để tính toán cân bằng nhiệt ẩm truyền thống gồm \textbf{7 bước} để tính toán như sau:
\subsubsection{Xác định các nguồn nhiệt toả ra}
\hspace{1cm}- Các nguồn nhiệt này có thể xuất phát từ nhiều nguồn khác nhau, điển hình như: Do người, do máy móc, chiếu sáng, rò rỉ không khí, bức xạ mặt trời, thẩm thấu qua kết cấu,...

- Phương trình cân bằng nhiệt tổng quát có dạng:

\begin{center}
	\textit{Q{\footnotesize t}} = \textit{Q{\footnotesize toả} + Q{\footnotesize tt}}
\end{center}

\begin{itemize}[leftmargin = 3cm, label = $\star$]
	\item \textit{Q{\footnotesize t}} - nhiệt thừa trong phòng, \textit{W};
	
	\item \textit{Q{\footnotesize toả}} - nhiệt toả ra trong phòng, \textit{W};
	
	\item \textit{Q{\footnotesize tt}} - nhiệt thẩm thấu từ ngoài vào qua kết cấu bao che do chênh lệch nhiệt độ, \textit{W};
\end{itemize}


- Trong đó, nhiệt lượng \textbf{Q{\footnotesize toả}} có thể được phân thành \textit{8 phần nhiệt} như sau:

\begin{center}
	\textit{Q{\footnotesize toả}} = \textit{Q$ _{1} $ + Q$ _{2} $ + Q$ _{3} $ + Q$ _{4} $ + Q$ _{5} $ + Q$ _{6} $ + Q$ _{7} $ + Q$ _{8} $}
\end{center}

\begin{itemize}[leftmargin = 3cm, label = $\ast$]
	\item \textit{Q$ _{1} $} - Nhiệt toả từ máy móc;
	
	\item \textit{Q$ _{2} $} - Nhiệt toả từ đèn chiếu;
	
	\item \textit{Q$ _{3} $} - Nhiệt toả từ người;
	
	\item \textit{Q$ _{4} $} - Nhiệt toả từ bán thành phẩm;
	
	\item \textit{Q$ _{5} $} - Nhiệt toả từ bề mặt thiết bị trao đổi nhiệt;
	
	\item \textit{Q$ _{6} $} - Nhiệt toả do bức xạ mặt trời qua cửa kính;
	
	\item \textit{Q$ _{7} $} - Nhiệt toả do bức xạ mặt trời qua bao che;
	
	\item \textit{Q$ _{8} $} - Nhiệt toả do lò rọt không khí qua cửa;
\end{itemize}

- Lượng nhiệt từ \textbf{Q{\footnotesize tt}} có thể phân được phân thành 4 lượng nhiệt sau:

\begin{center}
	\textit{Q{\footnotesize tt}} = \textit{Q$ _{9} $ + Q$ _{10} $ + Q$ _{11} $ + Q{\footnotesize bs}}, \textit{W}
\end{center}

\begin{itemize}[leftmargin = 3cm, label = $\ast$]
	\item \textit{Q$ _{9} $} - Nhiệt thẩm thấu qua vách;
	
	\item \textit{Q$ _{10} $} - Nhiệt thẩm thấu qua trần (mái);
	
	\item \textit{Q$ _{11} $} - Nhiệt thẩm thấu qua nền;
	
	\item \textit{Q{\footnotesize bs}} - Nhiệt tổn thất bổ sung do gió và hướng vách;
\end{itemize}


\subsubsection{Xác định nguồn ẩm thừa trong phòng điều hoà W{\footnotesize t}:}

\begin{center}
	\textit{W{\footnotesize t}} = \textit{W$ _{1} $ + W$ _{2} $ + W$ _{3} $ + W$ _{4} $}, kg/s
\end{center}

\begin{itemize}[leftmargin = 3cm, label = $\ast$]
	\item \textit{W$ _{9} $} - Lượng ẩm do người toả vào phòng, \textit{kg/s};
	
	\item \textit{W$ _{10} $} - Lượng ẩm bay hơi từ bán thành phẩm, \textit{kg/s};

	\item \textit{W$ _{11} $} - Lượng ẩm do bay hơi từ sàn ẩm, \textit{kg/s};
	
	\item \textit{W{\footnotesize bs}} - Lượng ẩm do hơi nước nóng toả vào phòng, \textit{kg/s};	
\end{itemize}
	
\subsubsection{Xác định tia quá trình {\Large $\varepsilon$} (còn gọi là hệ số góc tia quá trình)}
\begin{center}
	{\Large $\varepsilon_{t}$} = $ \dfrac{Q_{t}}{W_{t}} $, \textit{kJ/kg}
\end{center}

\subsubsection{Xác định sơ đồ điều hoà không khí}
\hspace{1cm}- Trong bước này, chúng ta cần phải xác định được sơ đồ điều hoà không khí với các thông số trạng thái không khí trong nhà T, ngoài nhà N, hoà trộn H và thổi vào V ví dụ như entanpi I$_{T}$, I$_{N}$, I$_{H}$, I$_{V}$, nhiệt độ t$_{T}$, t$_{N}$, t$_{H}$, t$_{V}$, lưu lượng không khí G$_{T}$, G$_{N}$, G$_{H}$, G$_{V}$ (\textit{kg/s}), L$_{T}$, L$_{N}$, L$_{H}$, L$_{V}$ (\textit{$m^3/s$}), khối lượng riêng không khí $\rho_{T}$, $\rho_{N}$, $\rho_{H}$, $\rho_{V}$, ẩm dung của không khí d$_{T}$, d$_{N}$, d$_{H}$, d$_{V}$...

\subsubsection{Xác định năng suất gió của hệ thống}
\hspace{1cm}- Để có thể tải được hết nhiệt thừa ra khỏi phòng điều hoà cần một lượng gió G$_{t}$ là:

\begin{center}
	$ G_{t} = \dfrac{Q_{t}}{I_{T} - I_{V}} $, \textit{kg/s}
\end{center}

- Để có thể tải được hết ẩm thừa ra khỏi phòng điều hoà cần một lượng gió G$_{W}$ là:

\begin{center}
	$ G_{W} = \dfrac{W_{t}}{d_{T} - d_{V}} $, \textit{kg/s}
\end{center}

Năng suất gió của hệ thống G phải bằng G$_{t}$ và G$_{M}$ do đó:

\begin{center}
	G = G$_{t}$ = G$_{M}$
\end{center}

hay:

\begin{center}
	$ \dfrac{Q_{t}}{I_{T} - I_{V}} $ = $\dfrac{W_{t}}{d_{T} - d_{V}} $
\end{center}

rút ra:

\begin{center}
	$ \dfrac{Q_{t}}{W_{t}} $ = $ \dfrac{I_{T} - I_{V}}{d_{T} - d_{V}} $ = {\Large $\varepsilon_{t}$}
\end{center}

{\Large $\varepsilon_{t}$} chính là hệ số góc tia quá trình.

\subsubsection{Tính năng suất lạnh}
\hspace{1cm}- Năng suất lạnh của hệ thống điều hoà không khí Q$ _{0} $ có thể được tính như sau:

\begin{center}
	$Q _{0} = G_{V}(I_{H} - I_{V})$, \textit{kW}
\end{center}
\begin{center}
	$ Q _{0} = Q_{t}\times\dfrac{I_{H} - I_{V}}{I_{T} - I_{V}}$, \textit{kW}
\end{center}

\subsubsection{Lượng ẩm ngưng tụ trên dàn bay hơi W}
\begin{center}
	$W = G_{V}(d_{H}-d_{V})$, \textit{kg/s}
\end{center}

\subsection{TÍNH TOÁN NHIỆT ẨM CHO TOÀ NHÀ}




\section{PHƯƠNG PHÁP CARRIER}
\hspace{1cm}- 

\section{TÍNH TẢI LẠNH BẰNG \emph{REVIT MEP}}
\hspace{1cm}- 

\end{document}

>>>>>>> 5bd27b94e1e9f4a66ad0fe62dbdc20c2adae49b4
	%Trang bìa
\chapmoi{PHÂN TÍCH NĂNG LƯỢNG}




	%trang bìa
\chapmoi{PHÂN TÍCH VÀ CHỌN PHƯƠNG ÁN}
	\chapmoi{THIẾT KẾ DÀN BAY HƠI}


\section{TÍNH TOÁN NHIỆT - THIẾT KẾ - TRỞ KHÁNG}


\section{MÔ HÌNH 3D SỬ DỤNG \textit{INVENTOR}}


	\chapmoi{TÍNH TOÁN CHU TRÌNH LẠNH}

\section{LỰA CHỌN VÀ TÍNH TOÁN CHU TRÌNH LẠNH}


\section{TÍNH VÀ CHỌN MÁY NÉN}



	\chapmoi{THIẾT KẾ THIẾT BỊ NGƯNG TỤ}
	\chapmoi{THIẾT KẾ THIẾT BỊ PHỤ}


\section{BÌNH CHỨA CAO ÁP}


\section{BÌNH TÁCH DẦU}


\section{PHIN LỌC}


\section{PHIN SẤY}


\section{THÁP GIẢI NHIỆT}


\section{MÔ HÌNH 3D \textit{INVENTOR}}




	\chapmoi{THIẾT KẾ HỆ THỐNG ĐƯỜNG ỐNG}
	\chapmoi{THIẾT KẾ HỆ THỐNG CẤP THOÁT NƯỚC}
	\chapmoi{HỆ THỐNG ĐIỆN}
	\chapmoi{HỆ THỐNG TỰ ĐỘNG HOÁ}
	\chapmoi{TÍNH TOÁN KINH TẾ}
	
\newpage
\end{document}
\end{document}