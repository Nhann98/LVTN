\documentclass[a4paper]{report}

\usepackage{multicol} %dung de su dung khi dung excel sang table Latex

\usepackage[utf8]{vietnam} %dung de viet tieng Viet

\usepackage{fancyhdr} %dung de tao ra cac style kho giay

\usepackage[left=3cm, right=2cm, top=2.5cm, bottom=2cm, includefoot, includehead]{geometry} %dung de dinh dang kho giay

\usepackage{booktabs} %dung de chuyen excel sang table Latex

\usepackage{amsmath} %dung de tao ra cac phuong trinh toan hoc

\usepackage{graphicx} %dung de su dung khi insert image 

\usepackage{tocloft} %Dùng để chỉnh mục lục - khoảng cách tiêu đề các thể loại
\cftsetindents{section}{1cm}{1cm}
\cftsetindents{subsection}{1.5cm}{1.5cm}

\usepackage[hidelinks]{hyperref} %dùng để làm toc có thể click được

\usepackage{float} %dung de tao, di chuyen cac object nhu table, image...

\usepackage{adjustbox} %dung de dieu chinh scale nhung table qua to

\usepackage{subcaption} %dùng để tạo subfigure

\usepackage{lscape} %dung de xoay ngang to giay

\usepackage{scrextend} %dung de dieu chinh size font chu
\changefontsizes{12pt}

\usepackage{indentfirst} %dùng để sau khi section thì dòng đầu tiên thục vào n cm
\setlength{\parindent}{1cm}
\renewcommand{\baselinestretch}{1.5} %dùng để giãn dòng 1.5

\usepackage[Conny]{fncychap} %dùng để tạo style chapter

\usepackage{wrapfig} %dùng để wrap các text xung quanh table hoặc figure(image)

\usepackage{enumitem} %dùng để thiết kế lại các định dạng khi sử dụng list hoặc item, enumerate

%\usepackage{tgbonum} %dùng để điều chỉnh font chữ
%\fontfamily{\sfdefault}
%\selectfont

%Phan nay la su dung package fancyhdr
\pagestyle{fancy}
\fancyhf{}
\renewcommand{\footrulewidth}{1pt}
\renewcommand*{\thesection}{\Alph{section}.}
\renewcommand*{\thesubsection}{\hspace{0.75cm}\Roman{subsection}.}
\renewcommand*{\thesubsubsection}{\hspace{1cm}\arabic{subsubsection}.}
\graphicspath{{E:/K22 Nhiet Lanh/LVTN/Files Latex LVTN/Pic/}}

%Phan nay la input cac file vao de compile, build ra pdf.
%This is main file of LVTN
\tableofcontents %đây là lệnh tạo mục lục
\setcounter{secnumdepth}{3} %report không có đánh số cho subsubsection nên lệnh này dùng đánh số cho subsubsection

\input{LVTN_chap1&2}
\input{LVTN_Chap3}


\newpage
\end{document}
